%% Generated by Sphinx.
\def\sphinxdocclass{jupyterBook}
\documentclass[letterpaper,10pt,english]{jupyterBook}
\ifdefined\pdfpxdimen
   \let\sphinxpxdimen\pdfpxdimen\else\newdimen\sphinxpxdimen
\fi \sphinxpxdimen=.75bp\relax
\ifdefined\pdfimageresolution
    \pdfimageresolution= \numexpr \dimexpr1in\relax/\sphinxpxdimen\relax
\fi
%% let collapsible pdf bookmarks panel have high depth per default
\PassOptionsToPackage{bookmarksdepth=5}{hyperref}
%% turn off hyperref patch of \index as sphinx.xdy xindy module takes care of
%% suitable \hyperpage mark-up, working around hyperref-xindy incompatibility
\PassOptionsToPackage{hyperindex=false}{hyperref}
%% memoir class requires extra handling
\makeatletter\@ifclassloaded{memoir}
{\ifdefined\memhyperindexfalse\memhyperindexfalse\fi}{}\makeatother

\PassOptionsToPackage{warn}{textcomp}

\catcode`^^^^00a0\active\protected\def^^^^00a0{\leavevmode\nobreak\ }
\usepackage{cmap}
\usepackage{fontspec}
\defaultfontfeatures[\rmfamily,\sffamily,\ttfamily]{}
\usepackage{amsmath,amssymb,amstext}
\usepackage{polyglossia}
\setmainlanguage{english}



\setmainfont{FreeSerif}[
  Extension      = .otf,
  UprightFont    = *,
  ItalicFont     = *Italic,
  BoldFont       = *Bold,
  BoldItalicFont = *BoldItalic
]
\setsansfont{FreeSans}[
  Extension      = .otf,
  UprightFont    = *,
  ItalicFont     = *Oblique,
  BoldFont       = *Bold,
  BoldItalicFont = *BoldOblique,
]
\setmonofont{FreeMono}[
  Extension      = .otf,
  UprightFont    = *,
  ItalicFont     = *Oblique,
  BoldFont       = *Bold,
  BoldItalicFont = *BoldOblique,
]



\usepackage[Bjarne]{fncychap}
\usepackage[,numfigreset=1,mathnumfig]{sphinx}

\fvset{fontsize=\small}
\usepackage{geometry}


% Include hyperref last.
\usepackage{hyperref}
% Fix anchor placement for figures with captions.
\usepackage{hypcap}% it must be loaded after hyperref.
% Set up styles of URL: it should be placed after hyperref.
\urlstyle{same}

\addto\captionsenglish{\renewcommand{\contentsname}{Intro}}

\usepackage{sphinxmessages}



        % Start of preamble defined in sphinx-jupyterbook-latex %
         \usepackage[Latin,Greek]{ucharclasses}
        \usepackage{unicode-math}
        % fixing title of the toc
        \addto\captionsenglish{\renewcommand{\contentsname}{Contents}}
        \hypersetup{
            pdfencoding=auto,
            psdextra
        }
        % End of preamble defined in sphinx-jupyterbook-latex %
        

\title{My Jupyter Book}
\date{Oct 14, 2022}
\release{}
\author{Daniel Mo Houshmand}
\newcommand{\sphinxlogo}{\vbox{}}
\renewcommand{\releasename}{}
\makeindex
\begin{document}

\pagestyle{empty}
\sphinxmaketitle
\pagestyle{plain}
\sphinxtableofcontents
\pagestyle{normal}
\phantomsection\label{\detokenize{index::doc}}


\begin{sphinxVerbatim}[commandchars=\\\{\}]
```\PYGZob{}nb\PYGZhy{}exec\PYGZhy{}table\PYGZcb{}
```
\end{sphinxVerbatim}

\sphinxAtStartPar
which produces:


\begin{savenotes}\sphinxattablestart
\centering
\begin{tabulary}{\linewidth}[t]{|T|T|T|T|T|}
\hline
\sphinxstyletheadfamily 
\sphinxAtStartPar
Document
&\sphinxstyletheadfamily 
\sphinxAtStartPar
Modified
&\sphinxstyletheadfamily 
\sphinxAtStartPar
Method
&\sphinxstyletheadfamily 
\sphinxAtStartPar
Run Time (s)
&\sphinxstyletheadfamily 
\sphinxAtStartPar
Status
\\
\hline
\sphinxAtStartPar
index
&
\sphinxAtStartPar
2022\sphinxhyphen{}10\sphinxhyphen{}14 00:14
&
\sphinxAtStartPar
cache
&
\sphinxAtStartPar
0.79
&
\sphinxAtStartPar
✅
\\
\hline
\sphinxAtStartPar
intro
&
\sphinxAtStartPar
2022\sphinxhyphen{}10\sphinxhyphen{}14 00:14
&
\sphinxAtStartPar
cache
&
\sphinxAtStartPar
\sphinxhyphen{}
&
\sphinxAtStartPar
❌
\\
\hline
\sphinxAtStartPar
overview
&
\sphinxAtStartPar
2022\sphinxhyphen{}10\sphinxhyphen{}14 00:14
&
\sphinxAtStartPar
cache
&
\sphinxAtStartPar
2.9
&
\sphinxAtStartPar
✅
\\
\hline
\sphinxAtStartPar
ph
&
\sphinxAtStartPar
2022\sphinxhyphen{}10\sphinxhyphen{}14 00:14
&
\sphinxAtStartPar
cache
&
\sphinxAtStartPar
0.8
&
\sphinxAtStartPar
✅
\\
\hline
\sphinxAtStartPar
prenb
&
\sphinxAtStartPar
2022\sphinxhyphen{}10\sphinxhyphen{}14 00:14
&
\sphinxAtStartPar
cache
&
\sphinxAtStartPar
1.7
&
\sphinxAtStartPar
✅
\\
\hline
\sphinxAtStartPar
prereq
&
\sphinxAtStartPar
2022\sphinxhyphen{}10\sphinxhyphen{}14 00:14
&
\sphinxAtStartPar
cache
&
\sphinxAtStartPar
0.95
&
\sphinxAtStartPar
✅
\\
\hline
\sphinxAtStartPar
src/dsforum/fake\_synth/DSF\_intro
&
\sphinxAtStartPar
2022\sphinxhyphen{}10\sphinxhyphen{}14 00:14
&
\sphinxAtStartPar
cache
&
\sphinxAtStartPar
1.52
&
\sphinxAtStartPar
✅
\\
\hline
\sphinxAtStartPar
src/dsforum/fake\_synth/ano
&
\sphinxAtStartPar
2022\sphinxhyphen{}10\sphinxhyphen{}14 00:14
&
\sphinxAtStartPar
cache
&
\sphinxAtStartPar
0.84
&
\sphinxAtStartPar
✅
\\
\hline
\sphinxAtStartPar
src/dsforum/fake\_synth/fake
&
\sphinxAtStartPar
2022\sphinxhyphen{}10\sphinxhyphen{}14 00:14
&
\sphinxAtStartPar
cache
&
\sphinxAtStartPar
0.84
&
\sphinxAtStartPar
✅
\\
\hline
\sphinxAtStartPar
src/dsforum/fake\_synth/simulations
&
\sphinxAtStartPar
2022\sphinxhyphen{}10\sphinxhyphen{}14 00:14
&
\sphinxAtStartPar
cache
&
\sphinxAtStartPar
0.84
&
\sphinxAtStartPar
✅
\\
\hline
\sphinxAtStartPar
src/dsforum/fake\_synth/synthetic
&
\sphinxAtStartPar
2022\sphinxhyphen{}10\sphinxhyphen{}14 00:14
&
\sphinxAtStartPar
cache
&
\sphinxAtStartPar
\sphinxhyphen{}
&
\sphinxAtStartPar
❌
\\
\hline
\sphinxAtStartPar
src/effect/intro
&
\sphinxAtStartPar
2022\sphinxhyphen{}10\sphinxhyphen{}14 00:14
&
\sphinxAtStartPar
cache
&
\sphinxAtStartPar
1.52
&
\sphinxAtStartPar
✅
\\
\hline
\sphinxAtStartPar
src/effect/ventetid
&
\sphinxAtStartPar
2022\sphinxhyphen{}10\sphinxhyphen{}14 00:14
&
\sphinxAtStartPar
cache
&
\sphinxAtStartPar
0.84
&
\sphinxAtStartPar
✅
\\
\hline
\sphinxAtStartPar
src/intro/0intro
&
\sphinxAtStartPar
2022\sphinxhyphen{}10\sphinxhyphen{}14 00:14
&
\sphinxAtStartPar
cache
&
\sphinxAtStartPar
1.52
&
\sphinxAtStartPar
✅
\\
\hline
\sphinxAtStartPar
src/intro/devman
&
\sphinxAtStartPar
2022\sphinxhyphen{}10\sphinxhyphen{}14 00:14
&
\sphinxAtStartPar
cache
&
\sphinxAtStartPar
0.84
&
\sphinxAtStartPar
✅
\\
\hline
\sphinxAtStartPar
src/intro/install
&
\sphinxAtStartPar
2022\sphinxhyphen{}10\sphinxhyphen{}14 00:14
&
\sphinxAtStartPar
cache
&
\sphinxAtStartPar
0.75
&
\sphinxAtStartPar
✅
\\
\hline
\sphinxAtStartPar
src/intro/setup
&
\sphinxAtStartPar
2022\sphinxhyphen{}10\sphinxhyphen{}14 00:14
&
\sphinxAtStartPar
cache
&
\sphinxAtStartPar
0.84
&
\sphinxAtStartPar
✅
\\
\hline
\sphinxAtStartPar
src/opp/bp
&
\sphinxAtStartPar
2022\sphinxhyphen{}10\sphinxhyphen{}14 00:14
&
\sphinxAtStartPar
cache
&
\sphinxAtStartPar
1.52
&
\sphinxAtStartPar
✅
\\
\hline
\sphinxAtStartPar
src/opp/data
&
\sphinxAtStartPar
2022\sphinxhyphen{}10\sphinxhyphen{}14 00:14
&
\sphinxAtStartPar
cache
&
\sphinxAtStartPar
0.84
&
\sphinxAtStartPar
✅
\\
\hline
\sphinxAtStartPar
src/opp/ml
&
\sphinxAtStartPar
2022\sphinxhyphen{}10\sphinxhyphen{}14 00:14
&
\sphinxAtStartPar
cache
&
\sphinxAtStartPar
0.75
&
\sphinxAtStartPar
✅
\\
\hline
\sphinxAtStartPar
src/opp/ph0
&
\sphinxAtStartPar
2022\sphinxhyphen{}10\sphinxhyphen{}14 00:14
&
\sphinxAtStartPar
cache
&
\sphinxAtStartPar
1.52
&
\sphinxAtStartPar
✅
\\
\hline
\sphinxAtStartPar
src/opp/stats
&
\sphinxAtStartPar
2022\sphinxhyphen{}10\sphinxhyphen{}14 00:14
&
\sphinxAtStartPar
cache
&
\sphinxAtStartPar
0.84
&
\sphinxAtStartPar
✅
\\
\hline
\sphinxAtStartPar
src/opp/viz
&
\sphinxAtStartPar
2022\sphinxhyphen{}10\sphinxhyphen{}14 00:14
&
\sphinxAtStartPar
cache
&
\sphinxAtStartPar
0.84
&
\sphinxAtStartPar
✅
\\
\hline
\sphinxAtStartPar
src/test/SynthNAV0
&
\sphinxAtStartPar
2022\sphinxhyphen{}10\sphinxhyphen{}14 00:14
&
\sphinxAtStartPar
cache
&
\sphinxAtStartPar
\sphinxhyphen{}
&
\sphinxAtStartPar
❌
\\
\hline
\sphinxAtStartPar
src/test/SyntheticNAV
&
\sphinxAtStartPar
2022\sphinxhyphen{}10\sphinxhyphen{}14 00:14
&
\sphinxAtStartPar
cache
&
\sphinxAtStartPar
1.52
&
\sphinxAtStartPar
✅
\\
\hline
\sphinxAtStartPar
src/test/markdown\sphinxhyphen{}notebooks
&
\sphinxAtStartPar
2022\sphinxhyphen{}10\sphinxhyphen{}14 00:14
&
\sphinxAtStartPar
cache
&
\sphinxAtStartPar
0.77
&
\sphinxAtStartPar
✅
\\
\hline
\sphinxAtStartPar
src/test/notebooks
&
\sphinxAtStartPar
2022\sphinxhyphen{}10\sphinxhyphen{}14 00:14
&
\sphinxAtStartPar
cache
&
\sphinxAtStartPar
\sphinxhyphen{}
&
\sphinxAtStartPar
❌
\\
\hline
\end{tabulary}
\par
\sphinxattableend\end{savenotes}

\sphinxAtStartPar
https://logo.com/

\begin{sphinxVerbatim}[commandchars=\\\{\}]
\PYG{k+kn}{import} \PYG{n+nn}{plotly}\PYG{n+nn}{.}\PYG{n+nn}{io} \PYG{k}{as} \PYG{n+nn}{pio}
\PYG{k+kn}{import} \PYG{n+nn}{plotly}\PYG{n+nn}{.}\PYG{n+nn}{express} \PYG{k}{as} \PYG{n+nn}{px}
\PYG{k+kn}{import} \PYG{n+nn}{plotly}\PYG{n+nn}{.}\PYG{n+nn}{offline} \PYG{k}{as} \PYG{n+nn}{py}

\PYG{n}{df} \PYG{o}{=} \PYG{n}{px}\PYG{o}{.}\PYG{n}{data}\PYG{o}{.}\PYG{n}{iris}\PYG{p}{(}\PYG{p}{)}
\PYG{n}{fig} \PYG{o}{=} \PYG{n}{px}\PYG{o}{.}\PYG{n}{scatter}\PYG{p}{(}\PYG{n}{df}\PYG{p}{,} \PYG{n}{x}\PYG{o}{=}\PYG{l+s+s2}{\PYGZdq{}}\PYG{l+s+s2}{sepal\PYGZus{}width}\PYG{l+s+s2}{\PYGZdq{}}\PYG{p}{,} \PYG{n}{y}\PYG{o}{=}\PYG{l+s+s2}{\PYGZdq{}}\PYG{l+s+s2}{sepal\PYGZus{}length}\PYG{l+s+s2}{\PYGZdq{}}\PYG{p}{,} \PYG{n}{color}\PYG{o}{=}\PYG{l+s+s2}{\PYGZdq{}}\PYG{l+s+s2}{species}\PYG{l+s+s2}{\PYGZdq{}}\PYG{p}{,} \PYG{n}{size}\PYG{o}{=}\PYG{l+s+s2}{\PYGZdq{}}\PYG{l+s+s2}{sepal\PYGZus{}length}\PYG{l+s+s2}{\PYGZdq{}}\PYG{p}{)}
\PYG{n}{fig}
\end{sphinxVerbatim}

\begin{sphinxVerbatim}[commandchars=\\\{\}]
\PYG{k+kn}{import} \PYG{n+nn}{sympy} \PYG{k}{as} \PYG{n+nn}{sym}
\PYG{n}{f} \PYG{o}{=} \PYG{n}{sym}\PYG{o}{.}\PYG{n}{Function}\PYG{p}{(}\PYG{l+s+s1}{\PYGZsq{}}\PYG{l+s+s1}{f}\PYG{l+s+s1}{\PYGZsq{}}\PYG{p}{)}
\PYG{n}{y} \PYG{o}{=} \PYG{n}{sym}\PYG{o}{.}\PYG{n}{Function}\PYG{p}{(}\PYG{l+s+s1}{\PYGZsq{}}\PYG{l+s+s1}{y}\PYG{l+s+s1}{\PYGZsq{}}\PYG{p}{)}
\PYG{n}{n} \PYG{o}{=} \PYG{n}{sym}\PYG{o}{.}\PYG{n}{symbols}\PYG{p}{(}\PYG{l+s+sa}{r}\PYG{l+s+s1}{\PYGZsq{}}\PYG{l+s+s1}{\PYGZbs{}}\PYG{l+s+s1}{alpha}\PYG{l+s+s1}{\PYGZsq{}}\PYG{p}{)}
\PYG{n}{f} \PYG{o}{=} \PYG{n}{y}\PYG{p}{(}\PYG{n}{n}\PYG{p}{)}\PYG{o}{\PYGZhy{}}\PYG{l+m+mi}{2}\PYG{o}{*}\PYG{n}{y}\PYG{p}{(}\PYG{n}{n}\PYG{o}{\PYGZhy{}}\PYG{l+m+mi}{1}\PYG{o}{/}\PYG{n}{sym}\PYG{o}{.}\PYG{n}{pi}\PYG{p}{)}\PYG{o}{\PYGZhy{}}\PYG{l+m+mi}{5}\PYG{o}{*}\PYG{n}{y}\PYG{p}{(}\PYG{n}{n}\PYG{o}{\PYGZhy{}}\PYG{l+m+mi}{2}\PYG{p}{)}
\PYG{n}{glue}\PYG{p}{(}\PYG{l+s+s2}{\PYGZdq{}}\PYG{l+s+s2}{sym\PYGZus{}eq}\PYG{l+s+s2}{\PYGZdq{}}\PYG{p}{,} \PYG{n}{sym}\PYG{o}{.}\PYG{n}{rsolve}\PYG{p}{(}\PYG{n}{f}\PYG{p}{,}\PYG{n}{y}\PYG{p}{(}\PYG{n}{n}\PYG{p}{)}\PYG{p}{,}\PYG{p}{[}\PYG{l+m+mi}{1}\PYG{p}{,}\PYG{l+m+mi}{4}\PYG{p}{]}\PYG{p}{)}\PYG{p}{)}
\end{sphinxVerbatim}

\begin{sphinxuseclass}{sd-tab-set}
\begin{sphinxuseclass}{sd-tab-item}\subsubsection*{python}

\begin{sphinxuseclass}{sd-tab-content}
\begin{sphinxVerbatim}[commandchars=\\\{\}]
\PYG{k}{def} \PYG{n+nf}{main}\PYG{p}{(}\PYG{p}{)}\PYG{p}{:}
    \PYG{k}{return}
\end{sphinxVerbatim}

\end{sphinxuseclass}
\end{sphinxuseclass}
\begin{sphinxuseclass}{sd-tab-item}\subsubsection*{java}

\begin{sphinxuseclass}{sd-tab-content}
\begin{sphinxVerbatim}[commandchars=\\\{\}]
\PYG{k+kd}{class} \PYG{n+nc}{Main}\PYG{+w}{ }\PYG{p}{\PYGZob{}}
\PYG{+w}{    }\PYG{k+kd}{public}\PYG{+w}{ }\PYG{k+kd}{static}\PYG{+w}{ }\PYG{k+kt}{void}\PYG{+w}{ }\PYG{n+nf}{main}\PYG{p}{(}\PYG{n}{String}\PYG{o}{[}\PYG{o}{]}\PYG{+w}{ }\PYG{n}{args}\PYG{p}{)}\PYG{+w}{ }\PYG{p}{\PYGZob{}}
\PYG{+w}{    }\PYG{p}{\PYGZcb{}}
\PYG{p}{\PYGZcb{}}
\end{sphinxVerbatim}

\end{sphinxuseclass}
\end{sphinxuseclass}
\begin{sphinxuseclass}{sd-tab-item}\subsubsection*{julia}

\begin{sphinxuseclass}{sd-tab-content}
\begin{sphinxVerbatim}[commandchars=\\\{\}]
\PYG{k}{function}\PYG{+w}{ }\PYG{n}{main}\PYG{p}{(}\PYG{p}{)}
\PYG{k}{end}
\end{sphinxVerbatim}

\end{sphinxuseclass}
\end{sphinxuseclass}
\begin{sphinxuseclass}{sd-tab-item}\subsubsection*{fortran}

\begin{sphinxuseclass}{sd-tab-content}
\begin{sphinxVerbatim}[commandchars=\\\{\}]
\PYG{k}{PROGRAM }\PYG{n}{main}
\PYG{k}{END }\PYG{k}{PROGRAM }\PYG{n}{main}
\end{sphinxVerbatim}

\end{sphinxuseclass}
\end{sphinxuseclass}
\begin{sphinxuseclass}{sd-tab-item}\subsubsection*{c++}

\begin{sphinxuseclass}{sd-tab-content}
\begin{sphinxVerbatim}[commandchars=\\\{\}]
\PYG{k+kt}{int}\PYG{+w}{ }\PYG{n+nf}{main}\PYG{p}{(}\PYG{k}{const}\PYG{+w}{ }\PYG{k+kt}{int}\PYG{+w}{ }\PYG{n}{argc}\PYG{p}{,}\PYG{+w}{ }\PYG{k}{const}\PYG{+w}{ }\PYG{k+kt}{char}\PYG{+w}{ }\PYG{o}{*}\PYG{o}{*}\PYG{n}{argv}\PYG{p}{)}\PYG{+w}{ }\PYG{p}{\PYGZob{}}
\PYG{+w}{  }\PYG{k}{return}\PYG{+w}{ }\PYG{l+m+mi}{0}\PYG{p}{;}
\PYG{p}{\PYGZcb{}}
\end{sphinxVerbatim}

\end{sphinxuseclass}
\end{sphinxuseclass}
\end{sphinxuseclass}
\begin{sphinxVerbatim}[commandchars=\\\{\}]
\PYG{n}{pip} \PYG{n}{install} \PYG{n}{sweetviz}
\PYG{n}{pip} \PYG{n}{install} \PYG{n}{jupyter}\PYG{o}{\PYGZhy{}}\PYG{n}{cache}
\PYG{n+nb}{print}\PYG{p}{(}\PYG{l+s+s1}{\PYGZsq{}}\PYG{l+s+s1}{this is python}\PYG{l+s+s1}{\PYGZsq{}}\PYG{p}{)}
\end{sphinxVerbatim}

\begin{sphinxuseclass}{cell}\begin{sphinxVerbatimInput}

\begin{sphinxuseclass}{cell_input}
\begin{sphinxVerbatim}[commandchars=\\\{\}]
\PYG{n}{a} \PYG{o}{=} \PYG{l+m+mi}{1}
\PYG{n}{b} \PYG{o}{=} \PYG{l+m+mi}{2}
\PYG{n}{c} \PYG{o}{=} \PYG{l+m+mi}{1}
\end{sphinxVerbatim}

\end{sphinxuseclass}\end{sphinxVerbatimInput}

\end{sphinxuseclass}
\begin{sphinxadmonition}{important}{Important:}
\begin{sphinxadmonition}{note}{Note:}
\sphinxAtStartPar
This text is \sphinxstylestrong{standard} \sphinxstyleemphasis{Markdown}
\end{sphinxadmonition}
\end{sphinxadmonition}

\sphinxAtStartPar
pip install sweetviz
You can install Jupyter Book via pip:

\begin{sphinxVerbatim}[commandchars=\\\{\}]
\PYG{n}{pip} \PYG{n}{install} \PYG{o}{\PYGZhy{}}\PYG{n}{U} \PYG{n}{jupyter}\PYG{o}{\PYGZhy{}}\PYG{n}{book}
\end{sphinxVerbatim}

\sphinxAtStartPar
or via conda\sphinxhyphen{}forge:

\sphinxAtStartPar
conda install \sphinxhyphen{}c conda\sphinxhyphen{}forge jupyter\sphinxhyphen{}book
This will install everything you need to build a Jupyter Book locally.

\sphinxAtStartPar
from myst\_nb import glue
my\_variable = “here is some text!”
glue(“cool\_text”, my\_variable)

\begin{sphinxVerbatim}[commandchars=\\\{\}]
\PYG{n}{pip} \PYG{n}{install} \PYG{n}{sweetviz}
\PYG{n+nb}{print}\PYG{p}{(}\PYG{l+s+s1}{\PYGZsq{}}\PYG{l+s+s1}{this is python}\PYG{l+s+s1}{\PYGZsq{}}\PYG{p}{)}
\end{sphinxVerbatim}

\begin{sphinxShadowBox}
\sphinxstylesidebartitle{}

\begin{sphinxadmonition}{note}{Note:}
\sphinxAtStartPar
Here’s my note!
\end{sphinxadmonition}
\end{sphinxShadowBox}
\begin{equation*}
\begin{split}
  \int_0^\infty \frac{x^3}{e^x-1}\,dx = \frac{\pi^4}{15}
\end{split}
\end{equation*}
\begin{sphinxadmonition}{note}{Note:}
\sphinxAtStartPar
Here is a note
\end{sphinxadmonition}

\begin{DUlineblock}{0em}
\item[] \sphinxstylestrong{\Large Notebooks with MyST Markdown}
\end{DUlineblock}

\sphinxAtStartPar
Jupyter Book also lets you write text\sphinxhyphen{}based notebooks using MyST Markdown.
See \sphinxhref{https://jupyterbook.org/file-types/myst-notebooks.html}{the Notebooks with MyST Markdown documentation} for more detailed instructions.
This page shows off a notebook written in MyST Markdown.

\begin{DUlineblock}{0em}
\item[] \sphinxstylestrong{\large An example cell}
\end{DUlineblock}

\sphinxAtStartPar
With MyST Markdown, you can define code cells with a directive like so:

\begin{sphinxuseclass}{cell}\begin{sphinxVerbatimInput}

\begin{sphinxuseclass}{cell_input}
\begin{sphinxVerbatim}[commandchars=\\\{\}]
\PYG{n+nb}{print}\PYG{p}{(}\PYG{l+m+mi}{2} \PYG{o}{+} \PYG{l+m+mi}{2}\PYG{p}{)}
\end{sphinxVerbatim}

\end{sphinxuseclass}\end{sphinxVerbatimInput}
\begin{sphinxVerbatimOutput}

\begin{sphinxuseclass}{cell_output}
\begin{sphinxVerbatim}[commandchars=\\\{\}]
4
\end{sphinxVerbatim}

\end{sphinxuseclass}\end{sphinxVerbatimOutput}

\end{sphinxuseclass}
\sphinxAtStartPar
When your book is built, the contents of any \sphinxcode{\sphinxupquote{\{code\sphinxhyphen{}cell\}}} blocks will be
executed with your default Jupyter kernel, and their outputs will be displayed
in\sphinxhyphen{}line with the rest of your content.


\sphinxstrong{See also:}
\nopagebreak


\sphinxAtStartPar
Jupyter Book uses \sphinxhref{https://jupytext.readthedocs.io/en/latest/}{Jupytext} to convert text\sphinxhyphen{}based files to notebooks, and can support \sphinxhref{https://jupyterbook.org/file-types/jupytext.html}{many other text\sphinxhyphen{}based notebook files}.



\begin{DUlineblock}{0em}
\item[] \sphinxstylestrong{\large Create a notebook with MyST Markdown}
\end{DUlineblock}

\sphinxAtStartPar
MyST Markdown notebooks are defined by two things:
\begin{enumerate}
\sphinxsetlistlabels{\arabic}{enumi}{enumii}{}{.}%
\item {} 
\sphinxAtStartPar
YAML metadata that is needed to understand if / how it should convert text files to notebooks (including information about the kernel needed).
See the YAML at the top of this page for example.

\item {} 
\sphinxAtStartPar
The presence of \sphinxcode{\sphinxupquote{\{code\sphinxhyphen{}cell\}}} directives, which will be executed with your book.

\end{enumerate}

\sphinxAtStartPar
That’s all that is needed to get started!

\begin{DUlineblock}{0em}
\item[] \sphinxstylestrong{\large Quickly add YAML metadata for MyST Notebooks}
\end{DUlineblock}

\sphinxAtStartPar
If you have a markdown file and you’d like to quickly add YAML metadata to it, so that Jupyter Book will treat it as a MyST Markdown Notebook, run the following command:

\begin{sphinxVerbatim}[commandchars=\\\{\}]
\PYG{n}{jupyter}\PYG{o}{\PYGZhy{}}\PYG{n}{book} \PYG{n}{myst} \PYG{n}{init} \PYG{n}{path}\PYG{o}{/}\PYG{n}{to}\PYG{o}{/}\PYG{n}{markdownfile}\PYG{o}{.}\PYG{n}{md}
\end{sphinxVerbatim}

\begin{DUlineblock}{0em}
\item[] \sphinxstylestrong{\large Markdown Files}
\end{DUlineblock}

\sphinxAtStartPar
Whether you write your book’s content in Jupyter Notebooks (\sphinxcode{\sphinxupquote{.ipynb}}) or
in regular markdown files (\sphinxcode{\sphinxupquote{.md}}), you’ll write in the same flavor of markdown
called \sphinxstylestrong{MyST Markdown}.
This is a simple file to help you get started and show off some syntax.

\begin{DUlineblock}{0em}
\item[] \sphinxstylestrong{\large What is MyST?}
\end{DUlineblock}

\sphinxAtStartPar
MyST stands for “Markedly Structured Text”. It
is a slight variation on a flavor of markdown called “CommonMark” markdown,
with small syntax extensions to allow you to write \sphinxstylestrong{roles} and \sphinxstylestrong{directives}
in the Sphinx ecosystem.

\sphinxAtStartPar
For more about MyST, see \sphinxhref{https://jupyterbook.org/content/myst.html}{the MyST Markdown Overview}.

\begin{DUlineblock}{0em}
\item[] \sphinxstylestrong{\large Sample Roles and Directives}
\end{DUlineblock}

\sphinxAtStartPar
Roles and directives are two of the most powerful tools in Jupyter Book. They
are kind of like functions, but written in a markup language. They both
serve a similar purpose, but \sphinxstylestrong{roles are written in one line}, whereas
\sphinxstylestrong{directives span many lines}. They both accept different kinds of inputs,
and what they do with those inputs depends on the specific role or directive
that is being called.

\sphinxAtStartPar
Here is a “note” directive:

\begin{sphinxadmonition}{note}{Note:}
\sphinxAtStartPar
Here is a note
\end{sphinxadmonition}

\sphinxAtStartPar
It will be rendered in a special box when you build your book.

\sphinxAtStartPar
Here is an inline directive to refer to a document: \DUrole{xref,std,std-doc}{markdown\sphinxhyphen{}notebooks}.

\begin{DUlineblock}{0em}
\item[] \sphinxstylestrong{\large Citations}
\end{DUlineblock}

\sphinxAtStartPar
You can also cite references that are stored in a \sphinxcode{\sphinxupquote{bibtex}} file. For example,
the following syntax: \sphinxcode{\sphinxupquote{\{cite\}`holdgraf\_evidence\_2014`}} will render like
this: {[}\hyperlink{cite.src/test/markdown:id3}{HdHPK14}{]}.

\sphinxAtStartPar
Moreover, you can insert a bibliography into your page with this syntax:
The \sphinxcode{\sphinxupquote{\{bibliography\}}} directive must be used for all the \sphinxcode{\sphinxupquote{\{cite\}}} roles to
render properly.
For example, if the references for your book are stored in \sphinxcode{\sphinxupquote{references.bib}},
then the bibliography is inserted with:

\begin{DUlineblock}{0em}
\item[] \sphinxstylestrong{\large Learn more}
\end{DUlineblock}

\sphinxAtStartPar
This is just a simple starter to get you started.
You can learn a lot more at \sphinxhref{https://jupyterbook.org}{jupyterbook.org}.
\begin{itemize}
\item {} 
\sphinxAtStartPar
Intro

\begin{itemize}
\item {} 
\sphinxAtStartPar
{\hyperref[\detokenize{intro::doc}]{\sphinxcrossref{Welcome to your Jupyter Book}}}

\item {} 
\sphinxAtStartPar
{\hyperref[\detokenize{overview::doc}]{\sphinxcrossref{Store code outputs and insert into content}}}

\item {} 
\sphinxAtStartPar
{\hyperref[\detokenize{prereq::doc}]{\sphinxcrossref{EffektTeam Effekt}}}

\end{itemize}
\end{itemize}
\begin{itemize}
\item {} 
\sphinxAtStartPar
Configurations

\begin{itemize}
\item {} 
\sphinxAtStartPar
{\hyperref[\detokenize{src/intro/0intro::doc}]{\sphinxcrossref{Introduksjon}}}

\item {} 
\sphinxAtStartPar
{\hyperref[\detokenize{src/intro/setup::doc}]{\sphinxcrossref{Setup}}}

\end{itemize}
\end{itemize}
\begin{itemize}
\item {} 
\sphinxAtStartPar
Oppfriskning \& Oppslag

\begin{itemize}
\item {} 
\sphinxAtStartPar
{\hyperref[\detokenize{src/opp/stats::doc}]{\sphinxcrossref{Statistics}}}

\item {} 
\sphinxAtStartPar
{\hyperref[\detokenize{src/opp/data::doc}]{\sphinxcrossref{Data Handling}}}

\item {} 
\sphinxAtStartPar
{\hyperref[\detokenize{src/opp/ml::doc}]{\sphinxcrossref{Machine Learning}}}

\item {} 
\sphinxAtStartPar
{\hyperref[\detokenize{src/opp/viz::doc}]{\sphinxcrossref{Visualisation Tools \& Techniques}}}

\item {} 
\sphinxAtStartPar
{\hyperref[\detokenize{src/opp/bp::doc}]{\sphinxcrossref{Best Practices}}}

\end{itemize}
\end{itemize}
\begin{itemize}
\item {} 
\sphinxAtStartPar
Data Science Forum

\begin{itemize}
\item {} 
\sphinxAtStartPar
{\hyperref[\detokenize{src/dsforum/fake_synth/DSF_intro::doc}]{\sphinxcrossref{Intro Data Science Forum \sphinxhyphen{} Fake \& Synthetic Data}}}

\end{itemize}
\end{itemize}
\begin{itemize}
\item {} 
\sphinxAtStartPar
Team Effekt

\begin{itemize}
\item {} 
\sphinxAtStartPar
{\hyperref[\detokenize{src/effect/intro::doc}]{\sphinxcrossref{Intro POAO \sphinxhyphen{} Team Effekt}}}

\item {} 
\sphinxAtStartPar
{\hyperref[\detokenize{src/effect/ventetid::doc}]{\sphinxcrossref{Ventetid}}}

\end{itemize}
\end{itemize}
\begin{itemize}
\item {} 
\sphinxAtStartPar
Testing

\begin{itemize}
\item {} 
\sphinxAtStartPar
{\hyperref[\detokenize{src/test/markdown-notebooks::doc}]{\sphinxcrossref{Notebooks with MyST Markdown}}}

\item {} 
\sphinxAtStartPar
{\hyperref[\detokenize{src/test/SynthNAV0::doc}]{\sphinxcrossref{DataSynthesizer (random mode)}}}

\item {} 
\sphinxAtStartPar
{\hyperref[\detokenize{src/test/SyntheticNAV::doc}]{\sphinxcrossref{SyntheticNAV}}}

\end{itemize}
\end{itemize}

\sphinxstepscope


\part{Intro}

\sphinxstepscope


\chapter{Welcome to your Jupyter Book}
\label{\detokenize{intro:welcome-to-your-jupyter-book}}\label{\detokenize{intro::doc}}

\section{Quickly add YAML metadata for MyST Notebooks}
\label{\detokenize{intro:quickly-add-yaml-metadata-for-myst-notebooks}}
\sphinxAtStartPar
If you have a markdown file and you’d like to quickly add YAML metadata to it, so that Jupyter Book will treat it as a MyST Markdown Notebook, run the following command:

\begin{sphinxuseclass}{cell}\begin{sphinxVerbatimInput}

\begin{sphinxuseclass}{cell_input}
\begin{sphinxVerbatim}[commandchars=\\\{\}]
\PYG{n}{jupyter}\PYG{o}{\PYGZhy{}}\PYG{n}{book} \PYG{n}{myst} \PYG{n}{init} \PYG{n}{path}\PYG{o}{/}\PYG{n}{to}\PYG{o}{/}\PYG{n}{markdownfile}\PYG{o}{.}\PYG{n}{md}
\end{sphinxVerbatim}

\end{sphinxuseclass}\end{sphinxVerbatimInput}

\end{sphinxuseclass}

\bigskip\hrule\bigskip



\section{substitutions:
key1: “I’m a \sphinxstylestrong{substitution}”
key2: |
\sphinxstyleliteralintitle{\sphinxupquote{\{note\}     \{\{ key1 \}\}     }}
fishy: |
\sphinxstyleliteralintitle{\sphinxupquote{\{image\} img/fun\sphinxhyphen{}fish.png     :alt: fishy     :width: 200px     }}}
\label{\detokenize{intro:substitutions-key1-im-a-substitution-key2-note-key1-fishy-image-img-fun-fish-png-alt-fishy-width-200px}}
\sphinxAtStartPar
\sphinxhref{https://jupyterbook.org}{the Jupyter Book documentation}

\sphinxstepscope


\chapter{Store code outputs and insert into content}
\label{\detokenize{overview:store-code-outputs-and-insert-into-content}}\label{\detokenize{overview:content-code-outputs-glue}}\label{\detokenize{overview::doc}}
\sphinxAtStartPar
You often wish to run analyses in one notebook and insert them in your
documents elsewhere. For example, if you’d like to include a figure,
or if you want to cite an analysis that you have run.

\begin{sphinxShadowBox}
\sphinxstylesidebartitle{}

\sphinxAtStartPar
Currently, \sphinxcode{\sphinxupquote{glue}} only works with Python.
\end{sphinxShadowBox}

\sphinxAtStartPar
The \sphinxcode{\sphinxupquote{glue}} tool from \sphinxhref{https://myst-nb.readthedocs.io/}{MyST\sphinxhyphen{}NB}
allows you to add a key to variables in a notebook,
then display those variables in your book by referencing the key. It
follows a two\sphinxhyphen{}step process:
\begin{itemize}
\item {} 
\sphinxAtStartPar
\sphinxstylestrong{Glue a variable to a name}. Do this by using
the \sphinxcode{\sphinxupquote{myst\_nb.glue}} function on a variable
that you’d like to re\sphinxhyphen{}use elsewhere in the book. You’ll give the variable
a name that can be referenced later.

\item {} 
\sphinxAtStartPar
\sphinxstylestrong{Reference that variable from your page’s content}. Then, when you are
writing your content, insert the variable into your text by using a
\sphinxcode{\sphinxupquote{\{glue:\}}} role.

\end{itemize}

\sphinxAtStartPar
We’ll cover each step in more detail below.

\begin{sphinxShadowBox}
\sphinxstylesidebartitle{}

\sphinxAtStartPar
For more information about roles, see \DUrole{xref,myst}{}.
\end{sphinxShadowBox}


\section{Gluing variables in your notebook}
\label{\detokenize{overview:gluing-variables-in-your-notebook}}\label{\detokenize{overview:glue-gluing}}
\sphinxAtStartPar
You can use \sphinxcode{\sphinxupquote{myst\_nb.glue()}} to assign the value of a variable to
a key of your choice. \sphinxcode{\sphinxupquote{glue}} will store all of the information that is normally used to \sphinxstylestrong{display}
that variable (i.e., whatever happens when you display the variable by putting it at the end of a
code cell). Choose a key that you will remember, as you will use it later.

\sphinxAtStartPar
The following code glues a variable inside the notebook to the key \sphinxcode{\sphinxupquote{"cool\_text"}}:

\begin{sphinxuseclass}{cell}\begin{sphinxVerbatimInput}

\begin{sphinxuseclass}{cell_input}
\begin{sphinxVerbatim}[commandchars=\\\{\}]
\PYG{k+kn}{from} \PYG{n+nn}{myst\PYGZus{}nb} \PYG{k+kn}{import} \PYG{n}{glue}
\PYG{n}{my\PYGZus{}variable} \PYG{o}{=} \PYG{l+s+s2}{\PYGZdq{}}\PYG{l+s+s2}{here is some text!}\PYG{l+s+s2}{\PYGZdq{}}
\PYG{n}{glue}\PYG{p}{(}\PYG{l+s+s2}{\PYGZdq{}}\PYG{l+s+s2}{cool\PYGZus{}text}\PYG{l+s+s2}{\PYGZdq{}}\PYG{p}{,} \PYG{n}{my\PYGZus{}variable}\PYG{p}{)}
\end{sphinxVerbatim}

\end{sphinxuseclass}\end{sphinxVerbatimInput}
\begin{sphinxVerbatimOutput}

\begin{sphinxuseclass}{cell_output}
\begin{sphinxVerbatim}[commandchars=\\\{\}]
\PYGZsq{}here is some text!\PYGZsq{}
\end{sphinxVerbatim}

\end{sphinxuseclass}\end{sphinxVerbatimOutput}

\end{sphinxuseclass}
\sphinxAtStartPar
You can then insert it into your text. Adding
\sphinxcode{\sphinxupquote{\{glue:\}`cool\_text`}} to your content results in the
following: \DUrole{pasted-inline}{\sphinxcode{\sphinxupquote{'here is some text!'}}}.


\subsection{Gluing numbers, plots, and tables}
\label{\detokenize{overview:gluing-numbers-plots-and-tables}}
\sphinxAtStartPar
You can glue anything in your notebook and display it later with \sphinxcode{\sphinxupquote{\{glue:\}}}. Here
we’ll show how to glue and paste \sphinxstylestrong{numbers and images}. We’ll simulate some
data and run a simple bootstrap on it. We’ll hide most of this process below,
to focus on the glueing part.

\sphinxAtStartPar
In the cell below, \sphinxcode{\sphinxupquote{data}} contains our data, and \sphinxcode{\sphinxupquote{bootstrap\_indices}} is a collection of sample indices in each bootstrap. Below we’ll calculate a few statistics of interest, and
\sphinxstylestrong{\sphinxcode{\sphinxupquote{glue()}}} them into the notebook.

\begin{sphinxuseclass}{cell}\begin{sphinxVerbatimInput}

\begin{sphinxuseclass}{cell_input}
\begin{sphinxVerbatim}[commandchars=\\\{\}]
\PYG{c+c1}{\PYGZsh{} Calculate the mean of a bunch of random samples}
\PYG{n}{means} \PYG{o}{=} \PYG{n}{data}\PYG{p}{[}\PYG{n}{bootstrap\PYGZus{}indices}\PYG{p}{]}\PYG{o}{.}\PYG{n}{mean}\PYG{p}{(}\PYG{l+m+mi}{0}\PYG{p}{)}
\PYG{c+c1}{\PYGZsh{} Calculate the 95\PYGZpc{} confidence interval for the mean}
\PYG{n}{clo}\PYG{p}{,} \PYG{n}{chi} \PYG{o}{=} \PYG{n}{np}\PYG{o}{.}\PYG{n}{percentile}\PYG{p}{(}\PYG{n}{means}\PYG{p}{,} \PYG{p}{[}\PYG{l+m+mf}{2.5}\PYG{p}{,} \PYG{l+m+mf}{97.5}\PYG{p}{]}\PYG{p}{)}

\PYG{c+c1}{\PYGZsh{} Store the values in our notebook}
\PYG{n}{glue}\PYG{p}{(}\PYG{l+s+s2}{\PYGZdq{}}\PYG{l+s+s2}{boot\PYGZus{}mean}\PYG{l+s+s2}{\PYGZdq{}}\PYG{p}{,} \PYG{n}{means}\PYG{o}{.}\PYG{n}{mean}\PYG{p}{(}\PYG{p}{)}\PYG{p}{)}
\PYG{n}{glue}\PYG{p}{(}\PYG{l+s+s2}{\PYGZdq{}}\PYG{l+s+s2}{boot\PYGZus{}clo}\PYG{l+s+s2}{\PYGZdq{}}\PYG{p}{,} \PYG{n}{clo}\PYG{p}{)}
\PYG{n}{glue}\PYG{p}{(}\PYG{l+s+s2}{\PYGZdq{}}\PYG{l+s+s2}{boot\PYGZus{}chi}\PYG{l+s+s2}{\PYGZdq{}}\PYG{p}{,} \PYG{n}{chi}\PYG{p}{)}
\end{sphinxVerbatim}

\end{sphinxuseclass}\end{sphinxVerbatimInput}
\begin{sphinxVerbatimOutput}

\begin{sphinxuseclass}{cell_output}
\begin{sphinxVerbatim}[commandchars=\\\{\}]
3.0004698361476696
\end{sphinxVerbatim}

\begin{sphinxVerbatim}[commandchars=\\\{\}]
2.9880501551321603
\end{sphinxVerbatim}

\begin{sphinxVerbatim}[commandchars=\\\{\}]
3.012816614734734
\end{sphinxVerbatim}

\end{sphinxuseclass}\end{sphinxVerbatimOutput}

\end{sphinxuseclass}
\sphinxAtStartPar
By default, \sphinxcode{\sphinxupquote{glue}} will display the value of the variable you are gluing. This
is useful for sanity\sphinxhyphen{}checking its value at glue\sphinxhyphen{}time. If you’d like to \sphinxstylestrong{prevent display},
use the \sphinxcode{\sphinxupquote{display=False}} option. Note that below, we also \sphinxstyleemphasis{overwrite} the value of
\sphinxcode{\sphinxupquote{boot\_chi}} (but using the same value):

\begin{sphinxuseclass}{cell}\begin{sphinxVerbatimInput}

\begin{sphinxuseclass}{cell_input}
\begin{sphinxVerbatim}[commandchars=\\\{\}]
\PYG{n}{glue}\PYG{p}{(}\PYG{l+s+s2}{\PYGZdq{}}\PYG{l+s+s2}{boot\PYGZus{}chi\PYGZus{}notdisplayed}\PYG{l+s+s2}{\PYGZdq{}}\PYG{p}{,} \PYG{n}{chi}\PYG{p}{,} \PYG{n}{display}\PYG{o}{=}\PYG{k+kc}{False}\PYG{p}{)}
\end{sphinxVerbatim}

\end{sphinxuseclass}\end{sphinxVerbatimInput}
\begin{sphinxVerbatimOutput}

\begin{sphinxuseclass}{cell_output}
\end{sphinxuseclass}\end{sphinxVerbatimOutput}

\end{sphinxuseclass}
\sphinxAtStartPar
You can also glue visualizations, such as Matplotlib figures (here we use \sphinxcode{\sphinxupquote{display=False}} to ensure that the figure isn’t plotted twice):

\begin{sphinxuseclass}{cell}\begin{sphinxVerbatimInput}

\begin{sphinxuseclass}{cell_input}
\begin{sphinxVerbatim}[commandchars=\\\{\}]
\PYG{c+c1}{\PYGZsh{} Visualize the histogram with the intervals}
\PYG{n}{fig}\PYG{p}{,} \PYG{n}{ax} \PYG{o}{=} \PYG{n}{plt}\PYG{o}{.}\PYG{n}{subplots}\PYG{p}{(}\PYG{p}{)}
\PYG{n}{ax}\PYG{o}{.}\PYG{n}{hist}\PYG{p}{(}\PYG{n}{means}\PYG{p}{)}
\PYG{k}{for} \PYG{n}{ln} \PYG{o+ow}{in} \PYG{p}{[}\PYG{n}{clo}\PYG{p}{,} \PYG{n}{chi}\PYG{p}{]}\PYG{p}{:}
    \PYG{n}{ax}\PYG{o}{.}\PYG{n}{axvline}\PYG{p}{(}\PYG{n}{ln}\PYG{p}{,} \PYG{n}{ls}\PYG{o}{=}\PYG{l+s+s1}{\PYGZsq{}}\PYG{l+s+s1}{\PYGZhy{}\PYGZhy{}}\PYG{l+s+s1}{\PYGZsq{}}\PYG{p}{,} \PYG{n}{c}\PYG{o}{=}\PYG{l+s+s1}{\PYGZsq{}}\PYG{l+s+s1}{r}\PYG{l+s+s1}{\PYGZsq{}}\PYG{p}{)}
\PYG{n}{ax}\PYG{o}{.}\PYG{n}{set\PYGZus{}title}\PYG{p}{(}\PYG{l+s+s2}{\PYGZdq{}}\PYG{l+s+s2}{Bootstrap distribution and 95}\PYG{l+s+s2}{\PYGZpc{}}\PYG{l+s+s2}{ CI}\PYG{l+s+s2}{\PYGZdq{}}\PYG{p}{)}

\PYG{c+c1}{\PYGZsh{} And a wider figure to show a timeseries}
\PYG{n}{fig2}\PYG{p}{,} \PYG{n}{ax} \PYG{o}{=} \PYG{n}{plt}\PYG{o}{.}\PYG{n}{subplots}\PYG{p}{(}\PYG{n}{figsize}\PYG{o}{=}\PYG{p}{(}\PYG{l+m+mi}{6}\PYG{p}{,} \PYG{l+m+mi}{2}\PYG{p}{)}\PYG{p}{)}
\PYG{n}{ax}\PYG{o}{.}\PYG{n}{plot}\PYG{p}{(}\PYG{n}{np}\PYG{o}{.}\PYG{n}{sort}\PYG{p}{(}\PYG{n}{means}\PYG{p}{)}\PYG{p}{,} \PYG{n}{lw}\PYG{o}{=}\PYG{l+m+mi}{3}\PYG{p}{,} \PYG{n}{c}\PYG{o}{=}\PYG{l+s+s1}{\PYGZsq{}}\PYG{l+s+s1}{r}\PYG{l+s+s1}{\PYGZsq{}}\PYG{p}{)}
\PYG{n}{ax}\PYG{o}{.}\PYG{n}{set\PYGZus{}axis\PYGZus{}off}\PYG{p}{(}\PYG{p}{)}

\PYG{n}{glue}\PYG{p}{(}\PYG{l+s+s2}{\PYGZdq{}}\PYG{l+s+s2}{boot\PYGZus{}fig}\PYG{l+s+s2}{\PYGZdq{}}\PYG{p}{,} \PYG{n}{fig}\PYG{p}{,} \PYG{n}{display}\PYG{o}{=}\PYG{k+kc}{False}\PYG{p}{)}
\PYG{n}{glue}\PYG{p}{(}\PYG{l+s+s2}{\PYGZdq{}}\PYG{l+s+s2}{sorted\PYGZus{}means\PYGZus{}fig}\PYG{l+s+s2}{\PYGZdq{}}\PYG{p}{,} \PYG{n}{fig2}\PYG{p}{,} \PYG{n}{display}\PYG{o}{=}\PYG{k+kc}{False}\PYG{p}{)}
\end{sphinxVerbatim}

\end{sphinxuseclass}\end{sphinxVerbatimInput}
\begin{sphinxVerbatimOutput}

\begin{sphinxuseclass}{cell_output}
\noindent\sphinxincludegraphics{{overview_10_2}.png}

\noindent\sphinxincludegraphics{{overview_10_3}.png}

\end{sphinxuseclass}\end{sphinxVerbatimOutput}

\end{sphinxuseclass}
\sphinxAtStartPar
The same can be done for \sphinxcode{\sphinxupquote{DataFrame}}s (or other table\sphinxhyphen{}like objects) as well.

\begin{sphinxuseclass}{cell}\begin{sphinxVerbatimInput}

\begin{sphinxuseclass}{cell_input}
\begin{sphinxVerbatim}[commandchars=\\\{\}]
\PYG{n}{bootstrap\PYGZus{}subsets} \PYG{o}{=} \PYG{n}{data}\PYG{p}{[}\PYG{n}{bootstrap\PYGZus{}indices}\PYG{p}{]}\PYG{p}{[}\PYG{p}{:}\PYG{l+m+mi}{3}\PYG{p}{,} \PYG{p}{:}\PYG{l+m+mi}{5}\PYG{p}{]}\PYG{o}{.}\PYG{n}{T}
\PYG{n}{df} \PYG{o}{=} \PYG{n}{pd}\PYG{o}{.}\PYG{n}{DataFrame}\PYG{p}{(}\PYG{n}{bootstrap\PYGZus{}subsets}\PYG{p}{,} \PYG{n}{columns}\PYG{o}{=}\PYG{p}{[}\PYG{l+s+s2}{\PYGZdq{}}\PYG{l+s+s2}{first}\PYG{l+s+s2}{\PYGZdq{}}\PYG{p}{,} \PYG{l+s+s2}{\PYGZdq{}}\PYG{l+s+s2}{second}\PYG{l+s+s2}{\PYGZdq{}}\PYG{p}{,} \PYG{l+s+s2}{\PYGZdq{}}\PYG{l+s+s2}{third}\PYG{l+s+s2}{\PYGZdq{}}\PYG{p}{]}\PYG{p}{)}
\PYG{n}{glue}\PYG{p}{(}\PYG{l+s+s2}{\PYGZdq{}}\PYG{l+s+s2}{df\PYGZus{}tbl}\PYG{l+s+s2}{\PYGZdq{}}\PYG{p}{,} \PYG{n}{df}\PYG{p}{)}
\end{sphinxVerbatim}

\end{sphinxuseclass}\end{sphinxVerbatimInput}
\begin{sphinxVerbatimOutput}

\begin{sphinxuseclass}{cell_output}
\begin{sphinxVerbatim}[commandchars=\\\{\}]
      first    second     third
0  3.386805  2.897177  3.088443
1  2.847377  3.098285  3.257944
2  2.944767  2.997578  2.790158
3  2.956163  2.911256  3.332858
4  3.069492  2.979205  3.340269
\end{sphinxVerbatim}

\end{sphinxuseclass}\end{sphinxVerbatimOutput}

\end{sphinxuseclass}
\begin{sphinxadmonition}{tip}{Tip:}
\sphinxAtStartPar
Since we are going to paste this figure into our document at a later point,
you may wish to remove the output here, using the \sphinxcode{\sphinxupquote{remove\sphinxhyphen{}output}} tag
(see \DUrole{xref,std,std-ref}{hiding/remove\sphinxhyphen{}content}).
\end{sphinxadmonition}


\section{Pasting glued variables into your page}
\label{\detokenize{overview:pasting-glued-variables-into-your-page}}\label{\detokenize{overview:glue-pasting}}
\sphinxAtStartPar
Once you have glued variables to their names, you can then \sphinxstylestrong{paste}
those variables into your text in your book anywhere you like (even on other pages).
These variables can be pasted using one of the roles or directives in the \sphinxcode{\sphinxupquote{glue}} \sphinxstyleemphasis{family}.


\subsection{The \sphinxstyleliteralintitle{\sphinxupquote{glue}} role/directive}
\label{\detokenize{overview:the-glue-role-directive}}
\sphinxAtStartPar
The simplest role and directive is \sphinxcode{\sphinxupquote{glue:any}},
which pastes the glued output in\sphinxhyphen{}line or as a block respectively,
with no additional formatting.
Simply add this:

\begin{sphinxVerbatim}[commandchars=\\\{\}]
```\PYGZob{}glue:\PYGZcb{} your\PYGZhy{}key
```
\end{sphinxVerbatim}

\sphinxAtStartPar
For example, we’ll paste the plot we generated above with the following text:

\begin{sphinxVerbatim}[commandchars=\\\{\}]
```\PYGZob{}glue:\PYGZcb{} boot\PYGZus{}fig
```
\end{sphinxVerbatim}

\sphinxAtStartPar
Here’s how it looks:
\begin{sphinxVerbatimOutput}

\begin{sphinxuseclass}{cell_output}
\noindent\sphinxincludegraphics{{overview_10_0}.png}

\end{sphinxuseclass}\end{sphinxVerbatimOutput}

\sphinxAtStartPar
Or we can paste in\sphinxhyphen{}line objects like so:

\begin{sphinxVerbatim}[commandchars=\\\{\}]
In\PYGZhy{}line text; \PYGZob{}glue:\PYGZcb{}`boot\PYGZus{}mean`, and a figure: \PYGZob{}glue:\PYGZcb{}`boot\PYGZus{}fig`.
\end{sphinxVerbatim}

\sphinxAtStartPar
In\sphinxhyphen{}line text; \DUrole{pasted-inline}{\sphinxcode{\sphinxupquote{3.0004698361476696}}}, and a figure: \DUrole{pasted-inline}{\sphinxincludegraphics{{overview_10_0}.png}}.

\begin{sphinxadmonition}{tip}{Tip:}
\sphinxAtStartPar
We recommend using wider, shorter figures when plotting in\sphinxhyphen{}line, with a ratio
around 6x2. For example, here’s an in\sphinxhyphen{}line figure of sorted means
from our bootstrap: \DUrole{pasted-inline}{\sphinxincludegraphics{{overview_10_1}.png}}.
It can be used to make a visual point that isn’t too complex! For more
ideas, check out \sphinxhref{https://en.wikipedia.org/wiki/Sparkline}{how sparklines are used}.
\end{sphinxadmonition}

\sphinxAtStartPar
Next we’ll cover some more specific pasting functionality, which gives you more
control over how the pasted outputs look in your pages.


\section{Controlling the pasted outputs}
\label{\detokenize{overview:controlling-the-pasted-outputs}}
\sphinxAtStartPar
You can control the pasted outputs by using a sub\sphinxhyphen{}command of \sphinxcode{\sphinxupquote{\{glue:\}}}. These are used like so:
\sphinxcode{\sphinxupquote{\{glue:subcommand\}`key`}}. These subcommands allow you to control more of the look, feel, and
content of the pasted output.

\begin{sphinxadmonition}{tip}{Tip:}
\sphinxAtStartPar
When you use \sphinxcode{\sphinxupquote{\{glue:\}}} you are actually using shorthand for \sphinxcode{\sphinxupquote{\{glue:any\}}}. This is a
generic command that doesn’t make many assumptions about what you are gluing.
\end{sphinxadmonition}


\subsection{The \sphinxstyleliteralintitle{\sphinxupquote{glue:text}} role}
\label{\detokenize{overview:the-glue-text-role}}
\sphinxAtStartPar
The \sphinxcode{\sphinxupquote{glue:text}} role is specific to text outputs.
For example, the following text:

\begin{sphinxVerbatim}[commandchars=\\\{\}]
The mean of the bootstrapped distribution was \PYGZob{}glue:text\PYGZcb{}`boot\PYGZus{}mean` (95\PYGZpc{} confidence interval \PYGZob{}glue:text\PYGZcb{}`boot\PYGZus{}clo`/\PYGZob{}glue:text\PYGZcb{}`boot\PYGZus{}chi`).
\end{sphinxVerbatim}

\sphinxAtStartPar
Is rendered as:
The mean of the bootstrapped distribution was \DUrole{pasted-text}{3.0004698361476696} (95\% confidence interval \DUrole{pasted-text}{2.9880501551321603}/\DUrole{pasted-text}{3.012816614734734})

\begin{sphinxadmonition}{note}{Note:}
\sphinxAtStartPar
\sphinxcode{\sphinxupquote{glue:text}} only works with glued variables that contain a \sphinxcode{\sphinxupquote{text/plain}} output.
\end{sphinxadmonition}

\sphinxAtStartPar
With \sphinxcode{\sphinxupquote{glue:text}} we can \sphinxstylestrong{add formatting to the output}.
This is particularly useful if you are displaying numbers and
want to round the results. To add formatting, use this syntax:
\begin{itemize}
\item {} 
\sphinxAtStartPar
\sphinxcode{\sphinxupquote{\{glue:text\}`mykey:formatstring`}}

\end{itemize}

\sphinxAtStartPar
For example, \sphinxcode{\sphinxupquote{My rounded mean: \{glue:text\}`boot\_mean:.2f` }} will be rendered like this: My rounded mean: \DUrole{pasted-text}{3.00} (95\% CI: \DUrole{pasted-text}{2.99}/\DUrole{pasted-text}{3.01}).


\subsection{The \sphinxstyleliteralintitle{\sphinxupquote{glue:figure}} directive}
\label{\detokenize{overview:the-glue-figure-directive}}
\sphinxAtStartPar
With \sphinxcode{\sphinxupquote{glue:figure}} you can apply more formatting to figure\sphinxhyphen{}like objects,
such as giving them a caption and referenceable label. For example,

\begin{sphinxVerbatim}[commandchars=\\\{\}]
```\PYGZob{}glue:figure\PYGZcb{} boot\PYGZus{}fig
:figwidth: 300px
:name: \PYGZdq{}fig\PYGZhy{}boot\PYGZdq{}

This is a \PYG{g+gs}{**caption**}, with an embedded \PYG{l+s+sb}{`\PYGZob{}glue:text\PYGZcb{}`} element: \PYGZob{}glue:text\PYGZcb{}`boot\PYGZus{}mean:.2f`!
```
\end{sphinxVerbatim}

\sphinxAtStartPar
produces the following figure:

\begin{figure}[htbp]
\centering
\capstart
\begin{sphinxVerbatimOutput}

\begin{sphinxuseclass}{cell_output}
\noindent\sphinxincludegraphics{{overview_10_0}.png}

\end{sphinxuseclass}\end{sphinxVerbatimOutput}
\caption{This is a \sphinxstylestrong{caption}, with an embedded \sphinxcode{\sphinxupquote{\{glue:text\}}} element: \DUrole{pasted-text}{3.00}!}\label{\detokenize{overview:fig-boot}}\end{figure}

\sphinxAtStartPar
Later, the code

\begin{sphinxVerbatim}[commandchars=\\\{\}]
Here is a \PYGZob{}ref\PYGZcb{}`reference to the figure \PYGZlt{}fig\PYGZhy{}boot\PYGZgt{}`
\end{sphinxVerbatim}

\sphinxAtStartPar
can be used to reference the figure.

\sphinxAtStartPar
Here is a {\hyperref[\detokenize{overview:fig-boot}]{\sphinxcrossref{\DUrole{std,std-ref}{reference to the figure}}}}

\sphinxAtStartPar
Here’s a table:

\begin{sphinxVerbatim}[commandchars=\\\{\}]
```\PYGZob{}glue:figure\PYGZcb{} df\PYGZus{}tbl
:figwidth: 300px
:name: \PYGZdq{}tbl:df\PYGZdq{}

A caption for a pandas table.
```
\end{sphinxVerbatim}

\sphinxAtStartPar
which gets rendered as

\begin{figure}[htbp]
\centering
\capstart
\begin{sphinxVerbatimOutput}

\begin{sphinxuseclass}{cell_output}
\begin{sphinxVerbatim}[commandchars=\\\{\}]
      first    second     third
0  3.386805  2.897177  3.088443
1  2.847377  3.098285  3.257944
2  2.944767  2.997578  2.790158
3  2.956163  2.911256  3.332858
4  3.069492  2.979205  3.340269
\end{sphinxVerbatim}

\end{sphinxuseclass}\end{sphinxVerbatimOutput}
\caption{A caption for a pandas table.}\label{\detokenize{overview:tbl-df}}\end{figure}


\subsection{The \sphinxstyleliteralintitle{\sphinxupquote{glue:math}} directive}
\label{\detokenize{overview:the-glue-math-directive}}
\sphinxAtStartPar
The \sphinxcode{\sphinxupquote{glue:math}} directive is specific to LaTeX math outputs
(glued variables that contain a \sphinxcode{\sphinxupquote{text/latex}} MIME type),
and works similarly to the \sphinxhref{https://www.sphinx-doc.org/en/1.8/usage/restructuredtext/directives.html\#math}{Sphinx math directive}.
For example, with this code we glue an equation:

\begin{sphinxuseclass}{cell}\begin{sphinxVerbatimInput}

\begin{sphinxuseclass}{cell_input}
\begin{sphinxVerbatim}[commandchars=\\\{\}]
\PYG{k+kn}{import} \PYG{n+nn}{sympy} \PYG{k}{as} \PYG{n+nn}{sym}
\PYG{n}{f} \PYG{o}{=} \PYG{n}{sym}\PYG{o}{.}\PYG{n}{Function}\PYG{p}{(}\PYG{l+s+s1}{\PYGZsq{}}\PYG{l+s+s1}{f}\PYG{l+s+s1}{\PYGZsq{}}\PYG{p}{)}
\PYG{n}{y} \PYG{o}{=} \PYG{n}{sym}\PYG{o}{.}\PYG{n}{Function}\PYG{p}{(}\PYG{l+s+s1}{\PYGZsq{}}\PYG{l+s+s1}{y}\PYG{l+s+s1}{\PYGZsq{}}\PYG{p}{)}
\PYG{n}{n} \PYG{o}{=} \PYG{n}{sym}\PYG{o}{.}\PYG{n}{symbols}\PYG{p}{(}\PYG{l+s+sa}{r}\PYG{l+s+s1}{\PYGZsq{}}\PYG{l+s+s1}{\PYGZbs{}}\PYG{l+s+s1}{alpha}\PYG{l+s+s1}{\PYGZsq{}}\PYG{p}{)}
\PYG{n}{f} \PYG{o}{=} \PYG{n}{y}\PYG{p}{(}\PYG{n}{n}\PYG{p}{)}\PYG{o}{\PYGZhy{}}\PYG{l+m+mi}{2}\PYG{o}{*}\PYG{n}{y}\PYG{p}{(}\PYG{n}{n}\PYG{o}{\PYGZhy{}}\PYG{l+m+mi}{1}\PYG{o}{/}\PYG{n}{sym}\PYG{o}{.}\PYG{n}{pi}\PYG{p}{)}\PYG{o}{\PYGZhy{}}\PYG{l+m+mi}{5}\PYG{o}{*}\PYG{n}{y}\PYG{p}{(}\PYG{n}{n}\PYG{o}{\PYGZhy{}}\PYG{l+m+mi}{2}\PYG{p}{)}
\PYG{n}{glue}\PYG{p}{(}\PYG{l+s+s2}{\PYGZdq{}}\PYG{l+s+s2}{sym\PYGZus{}eq}\PYG{l+s+s2}{\PYGZdq{}}\PYG{p}{,} \PYG{n}{sym}\PYG{o}{.}\PYG{n}{rsolve}\PYG{p}{(}\PYG{n}{f}\PYG{p}{,}\PYG{n}{y}\PYG{p}{(}\PYG{n}{n}\PYG{p}{)}\PYG{p}{,}\PYG{p}{[}\PYG{l+m+mi}{1}\PYG{p}{,}\PYG{l+m+mi}{4}\PYG{p}{]}\PYG{p}{)}\PYG{p}{)}
\end{sphinxVerbatim}

\end{sphinxuseclass}\end{sphinxVerbatimInput}
\begin{sphinxVerbatimOutput}

\begin{sphinxuseclass}{cell_output}\begin{equation*}
\begin{split}\displaystyle \left(\sqrt{5} i\right)^{\alpha} \left(\frac{1}{2} - \frac{2 \sqrt{5} i}{5}\right) + \left(- \sqrt{5} i\right)^{\alpha} \left(\frac{1}{2} + \frac{2 \sqrt{5} i}{5}\right)\end{split}
\end{equation*}
\end{sphinxuseclass}\end{sphinxVerbatimOutput}

\end{sphinxuseclass}
\sphinxAtStartPar
and now we can use the following code:

\begin{sphinxVerbatim}[commandchars=\\\{\}]
```\PYGZob{}glue:math\PYGZcb{} sym\PYGZus{}eq
:label: eq\PYGZhy{}sym
```
\end{sphinxVerbatim}

\sphinxAtStartPar
to insert the equation here:
\begin{equation}\label{equation:index:eq-sym}
\begin{split}\displaystyle \left(\sqrt{5} i\right)^{\alpha} \left(\frac{1}{2} - \frac{2 \sqrt{5} i}{5}\right) + \left(- \sqrt{5} i\right)^{\alpha} \left(\frac{1}{2} + \frac{2 \sqrt{5} i}{5}\right)\end{split}
\end{equation}
\begin{sphinxadmonition}{note}{Note:}
\sphinxAtStartPar
\sphinxcode{\sphinxupquote{glue:math}} only works with glued variables that contain a \sphinxcode{\sphinxupquote{text/latex}} output.
\end{sphinxadmonition}


\section{Advanced \sphinxstyleliteralintitle{\sphinxupquote{glue}} use\sphinxhyphen{}cases}
\label{\detokenize{overview:advanced-glue-use-cases}}
\sphinxAtStartPar
Here are a few more specific and advanced uses of the \sphinxcode{\sphinxupquote{glue}} submodule.


\subsection{Pasting into tables}
\label{\detokenize{overview:pasting-into-tables}}
\sphinxAtStartPar
In addition to pasting blocks of outputs, or in\sphinxhyphen{}line with text, you can also paste directly
into tables. This allows you to compose complex collections of structured data using outputs
that were generated in other cells or other notebooks. For example, the following Markdown table:

\begin{sphinxVerbatim}[commandchars=\\\{\}]
| name                            |       plot                    | mean                      | ci                                                |
|:\PYGZhy{}\PYGZhy{}\PYGZhy{}\PYGZhy{}\PYGZhy{}\PYGZhy{}\PYGZhy{}\PYGZhy{}\PYGZhy{}\PYGZhy{}\PYGZhy{}\PYGZhy{}\PYGZhy{}\PYGZhy{}\PYGZhy{}\PYGZhy{}\PYGZhy{}\PYGZhy{}\PYGZhy{}\PYGZhy{}\PYGZhy{}\PYGZhy{}\PYGZhy{}\PYGZhy{}\PYGZhy{}\PYGZhy{}\PYGZhy{}\PYGZhy{}\PYGZhy{}\PYGZhy{}\PYGZhy{}:|:\PYGZhy{}\PYGZhy{}\PYGZhy{}\PYGZhy{}\PYGZhy{}\PYGZhy{}\PYGZhy{}\PYGZhy{}\PYGZhy{}\PYGZhy{}\PYGZhy{}\PYGZhy{}\PYGZhy{}\PYGZhy{}\PYGZhy{}\PYGZhy{}\PYGZhy{}\PYGZhy{}\PYGZhy{}\PYGZhy{}\PYGZhy{}\PYGZhy{}\PYGZhy{}\PYGZhy{}\PYGZhy{}\PYGZhy{}\PYGZhy{}\PYGZhy{}\PYGZhy{}:|\PYGZhy{}\PYGZhy{}\PYGZhy{}\PYGZhy{}\PYGZhy{}\PYGZhy{}\PYGZhy{}\PYGZhy{}\PYGZhy{}\PYGZhy{}\PYGZhy{}\PYGZhy{}\PYGZhy{}\PYGZhy{}\PYGZhy{}\PYGZhy{}\PYGZhy{}\PYGZhy{}\PYGZhy{}\PYGZhy{}\PYGZhy{}\PYGZhy{}\PYGZhy{}\PYGZhy{}\PYGZhy{}\PYGZhy{}\PYGZhy{}|\PYGZhy{}\PYGZhy{}\PYGZhy{}\PYGZhy{}\PYGZhy{}\PYGZhy{}\PYGZhy{}\PYGZhy{}\PYGZhy{}\PYGZhy{}\PYGZhy{}\PYGZhy{}\PYGZhy{}\PYGZhy{}\PYGZhy{}\PYGZhy{}\PYGZhy{}\PYGZhy{}\PYGZhy{}\PYGZhy{}\PYGZhy{}\PYGZhy{}\PYGZhy{}\PYGZhy{}\PYGZhy{}\PYGZhy{}\PYGZhy{}\PYGZhy{}\PYGZhy{}\PYGZhy{}\PYGZhy{}\PYGZhy{}\PYGZhy{}\PYGZhy{}\PYGZhy{}\PYGZhy{}\PYGZhy{}\PYGZhy{}\PYGZhy{}\PYGZhy{}\PYGZhy{}\PYGZhy{}\PYGZhy{}\PYGZhy{}\PYGZhy{}\PYGZhy{}\PYGZhy{}\PYGZhy{}\PYGZhy{}\PYGZhy{}\PYGZhy{}|
| histogram and raw text          | \PYGZob{}glue:\PYGZcb{}`boot\PYGZus{}fig`             | \PYGZob{}glue:\PYGZcb{}`boot\PYGZus{}mean`          | \PYGZob{}glue:\PYGZcb{}`boot\PYGZus{}clo`\PYGZhy{}\PYGZob{}glue:\PYGZcb{}`boot\PYGZus{}chi`                   |
| sorted means and formatted text | \PYGZob{}glue:\PYGZcb{}`sorted\PYGZus{}means\PYGZus{}fig`     | \PYGZob{}glue:text\PYGZcb{}`boot\PYGZus{}mean:.3f` | \PYGZob{}glue:text\PYGZcb{}`boot\PYGZus{}clo:.3f`\PYGZhy{}\PYGZob{}glue:text\PYGZcb{}`boot\PYGZus{}chi:.3f` |
\end{sphinxVerbatim}

\sphinxAtStartPar
Results in:


\begin{savenotes}\sphinxattablestart
\centering
\begin{tabulary}{\linewidth}[t]{|T|T|T|T|}
\hline
\sphinxstyletheadfamily 
\sphinxAtStartPar
name
&\sphinxstyletheadfamily 
\sphinxAtStartPar
plot
&\sphinxstyletheadfamily 
\sphinxAtStartPar
mean
&\sphinxstyletheadfamily 
\sphinxAtStartPar
ci
\\
\hline
\sphinxAtStartPar
histogram and raw text
&
\sphinxAtStartPar
\DUrole{pasted-inline}{\sphinxincludegraphics{{overview_10_0}.png}}
&
\sphinxAtStartPar
\DUrole{pasted-inline}{\sphinxcode{\sphinxupquote{3.0004698361476696}}}
&
\sphinxAtStartPar
\DUrole{pasted-inline}{\sphinxcode{\sphinxupquote{2.9880501551321603}}}\sphinxhyphen{}\DUrole{pasted-inline}{\sphinxcode{\sphinxupquote{3.012816614734734}}}
\\
\hline
\sphinxAtStartPar
sorted means and formatted text
&
\sphinxAtStartPar
\DUrole{pasted-inline}{\sphinxincludegraphics{{overview_10_1}.png}}
&
\sphinxAtStartPar
\DUrole{pasted-text}{3.000}
&
\sphinxAtStartPar
\DUrole{pasted-text}{2.988}\sphinxhyphen{}\DUrole{pasted-text}{3.013}
\\
\hline
\end{tabulary}
\par
\sphinxattableend\end{savenotes}

\sphinxstepscope


\chapter{EffektTeam Effekt}
\label{\detokenize{prereq:effektteam-effekt}}\label{\detokenize{prereq::doc}}
\sphinxAtStartPar
Some \sphinxstylestrong{Team Effekt}!

\begin{sphinxuseclass}{cell}
\begin{sphinxuseclass}{tag_mytag}\begin{sphinxVerbatimInput}

\begin{sphinxuseclass}{cell_input}
\begin{sphinxVerbatim}[commandchars=\\\{\}]
\PYG{n+nb}{print}\PYG{p}{(}\PYG{l+s+s2}{\PYGZdq{}}\PYG{l+s+s2}{A python cell}\PYG{l+s+s2}{\PYGZdq{}}\PYG{p}{)}
\end{sphinxVerbatim}

\end{sphinxuseclass}\end{sphinxVerbatimInput}
\begin{sphinxVerbatimOutput}

\begin{sphinxuseclass}{cell_output}
\begin{sphinxVerbatim}[commandchars=\\\{\}]
A python cell
\end{sphinxVerbatim}

\end{sphinxuseclass}\end{sphinxVerbatimOutput}

\end{sphinxuseclass}
\end{sphinxuseclass}

\section{A section}
\label{\detokenize{prereq:a-section}}
\sphinxAtStartPar
And some more Markdown…


\chapter{Prerequisites}
\label{\detokenize{prereq:prerequisites}}
\sphinxAtStartPar
\sphinxhref{https://jupyterbook.org}{jb}


\section{openml}
\label{\detokenize{prereq:openml}}
\sphinxstepscope


\section{nb}
\label{\detokenize{prenb:nb}}\label{\detokenize{prenb::doc}}
\begin{sphinxuseclass}{cell}\begin{sphinxVerbatimInput}

\begin{sphinxuseclass}{cell_input}
\begin{sphinxVerbatim}[commandchars=\\\{\}]
\PYG{k+kn}{import} \PYG{n+nn}{plotly}\PYG{n+nn}{.}\PYG{n+nn}{express} \PYG{k}{as} \PYG{n+nn}{px}
\PYG{n}{data} \PYG{o}{=} \PYG{n}{px}\PYG{o}{.}\PYG{n}{data}\PYG{o}{.}\PYG{n}{iris}\PYG{p}{(}\PYG{p}{)}
\PYG{n}{data}\PYG{o}{.}\PYG{n}{head}\PYG{p}{(}\PYG{p}{)}
\end{sphinxVerbatim}

\end{sphinxuseclass}\end{sphinxVerbatimInput}
\begin{sphinxVerbatimOutput}

\begin{sphinxuseclass}{cell_output}
\begin{sphinxVerbatim}[commandchars=\\\{\}]
   sepal\PYGZus{}length  sepal\PYGZus{}width  petal\PYGZus{}length  petal\PYGZus{}width species  species\PYGZus{}id
0           5.1          3.5           1.4          0.2  setosa           1
1           4.9          3.0           1.4          0.2  setosa           1
2           4.7          3.2           1.3          0.2  setosa           1
3           4.6          3.1           1.5          0.2  setosa           1
4           5.0          3.6           1.4          0.2  setosa           1
\end{sphinxVerbatim}

\end{sphinxuseclass}\end{sphinxVerbatimOutput}

\end{sphinxuseclass}
\begin{sphinxuseclass}{cell}\begin{sphinxVerbatimInput}

\begin{sphinxuseclass}{cell_input}
\begin{sphinxVerbatim}[commandchars=\\\{\}]
\PYG{k+kn}{import} \PYG{n+nn}{altair} \PYG{k}{as} \PYG{n+nn}{alt}
\PYG{n}{alt}\PYG{o}{.}\PYG{n}{Chart}\PYG{p}{(}\PYG{n}{data}\PYG{o}{=}\PYG{n}{data}\PYG{p}{)}\PYG{o}{.}\PYG{n}{mark\PYGZus{}point}\PYG{p}{(}\PYG{p}{)}\PYG{o}{.}\PYG{n}{encode}\PYG{p}{(}
    \PYG{n}{x}\PYG{o}{=}\PYG{l+s+s2}{\PYGZdq{}}\PYG{l+s+s2}{sepal\PYGZus{}width}\PYG{l+s+s2}{\PYGZdq{}}\PYG{p}{,}
    \PYG{n}{y}\PYG{o}{=}\PYG{l+s+s2}{\PYGZdq{}}\PYG{l+s+s2}{sepal\PYGZus{}length}\PYG{l+s+s2}{\PYGZdq{}}\PYG{p}{,}
    \PYG{n}{color}\PYG{o}{=}\PYG{l+s+s2}{\PYGZdq{}}\PYG{l+s+s2}{species}\PYG{l+s+s2}{\PYGZdq{}}\PYG{p}{,}
    \PYG{n}{size}\PYG{o}{=}\PYG{l+s+s1}{\PYGZsq{}}\PYG{l+s+s1}{sepal\PYGZus{}length}\PYG{l+s+s1}{\PYGZsq{}}
\PYG{p}{)}
\end{sphinxVerbatim}

\end{sphinxuseclass}\end{sphinxVerbatimInput}
\begin{sphinxVerbatimOutput}

\begin{sphinxuseclass}{cell_output}
\begin{sphinxVerbatim}[commandchars=\\\{\}]
alt.Chart(...)
\end{sphinxVerbatim}

\end{sphinxuseclass}\end{sphinxVerbatimOutput}

\end{sphinxuseclass}
\sphinxstepscope


\section{PH}
\label{\detokenize{ph:ph}}\label{\detokenize{ph::doc}}
\sphinxstepscope


\part{Configurations}

\sphinxstepscope


\chapter{Introduksjon}
\label{\detokenize{src/intro/0intro:introduksjon}}\label{\detokenize{src/intro/0intro::doc}}
\sphinxstepscope


\chapter{Setup}
\label{\detokenize{src/intro/setup:setup}}\label{\detokenize{src/intro/setup::doc}}
\sphinxstepscope


\section{Installations}
\label{\detokenize{src/intro/install:installations}}\label{\detokenize{src/intro/install::doc}}
\sphinxstepscope


\section{Environment, Dependency and Package Management}
\label{\detokenize{src/intro/devman:environment-dependency-and-package-management}}\label{\detokenize{src/intro/devman::doc}}
\sphinxstepscope


\part{Oppfriskning \& Oppslag}

\sphinxstepscope


\chapter{Statistics}
\label{\detokenize{src/opp/stats:statistics}}\label{\detokenize{src/opp/stats::doc}}
\sphinxstepscope


\section{Placeholder 0}
\label{\detokenize{src/opp/ph0:placeholder-0}}\label{\detokenize{src/opp/ph0::doc}}
\sphinxstepscope


\chapter{Data Handling}
\label{\detokenize{src/opp/data:data-handling}}\label{\detokenize{src/opp/data::doc}}
\sphinxstepscope


\chapter{Machine Learning}
\label{\detokenize{src/opp/ml:machine-learning}}\label{\detokenize{src/opp/ml::doc}}
\sphinxstepscope


\chapter{Visualisation Tools \& Techniques}
\label{\detokenize{src/opp/viz:visualisation-tools-techniques}}\label{\detokenize{src/opp/viz::doc}}

\chapter{Dashboarding}
\label{\detokenize{src/opp/viz:dashboarding}}

\chapter{MLOps}
\label{\detokenize{src/opp/viz:mlops}}

\chapter{Exploratory Data Analysis (EDA) Tools}
\label{\detokenize{src/opp/viz:exploratory-data-analysis-eda-tools}}
\sphinxstepscope


\chapter{Best Practices}
\label{\detokenize{src/opp/bp:best-practices}}\label{\detokenize{src/opp/bp::doc}}
\sphinxstepscope


\part{Data Science Forum}

\sphinxstepscope


\chapter{Intro Data Science Forum \sphinxhyphen{} Fake \& Synthetic Data}
\label{\detokenize{src/dsforum/fake_synth/DSF_intro:intro-data-science-forum-fake-synthetic-data}}\label{\detokenize{src/dsforum/fake_synth/DSF_intro::doc}}

\section{Data Masking (Data Maskering)}
\label{\detokenize{src/dsforum/fake_synth/DSF_intro:data-masking-data-maskering}}
\sphinxAtStartPar
Gitt de økende cybertruslene og implementeringen av personvernlovgivning som GDPR i EU eller CCPA i USA, må bedrifter sørge for at private data brukes så lite som mulig. Datamaskering gir en måte å begrense private data på, samtidig som det lar bedrifter teste systemene sine med data så nærme reelle data som mulig.

\sphinxAtStartPar
Den gjennomsnittlige kostnaden for et datainnbrudd ble estimert til 4,24 millioner dollar i 2020, noe som skaper sterke insentiver for bedrifter til å investere i informasjonssikkerhetsløsninger, inkludert datamaskering for å beskytte sensitive data. Datamaskering er en må\sphinxhyphen{}ha\sphinxhyphen{}løsning for organisasjoner som ønsker å overholde GDPR eller bruke realistiske data i et testmiljø.

\sphinxAtStartPar
I denne artikkelen forklarer vi datamaskering og gir en liste over de beste datamaskeringsteknikkene.

\sphinxAtStartPar
Hva er datamaskering?
Datamaskering blir også referert til som dataobfuskering, dataanonymisering eller pseudonymisering. Det er prosessen med å erstatte konfidensielle data ved å bruke funksjonelle fiktive data som tegn eller andre data. Hovedformålet med datamaskering er å beskytte sensitiv, privat informasjon i situasjoner der virksomheten deler data med tredjeparter.

\sphinxAtStartPar
Hvorfor er datamaskering viktig nå?
Antallet datainnbrudd øker hvert år (Sammenlignet med midten av 2018, var antallet registrerte brudd opp 54 \% i 2019) Derfor må organisasjoner forbedre sine datasikkerhetssystemer. Behovet for datamaskering øker på grunn av følgende årsaker:

\sphinxAtStartPar
Organisasjoner trenger en kopi av produksjonsdata når de bestemmer seg for å bruke dem for ikke\sphinxhyphen{}produksjonsmessige årsaker, for eksempel applikasjonstesting eller forretningsanalysemodellering.
Foretakets retningslinjer for personvern er også truet av innsidere. Derfor bør organisasjoner fortsatt være forsiktige mens de gir tilgang til innsidemedarbeidere.
GDPR og CCPA tvinger bedrifter til å styrke databeskyttelsessystemene sine, ellers må organisasjoner betale høye bøter.
Hvordan fungerer datamaskering?
Datamaskeringsprosessen er enkel, men den har forskjellige teknikker og typer. Generelt begynner organisasjoner med å identifisere alle sensitive data bedriften din har. Deretter bruker de algoritmer for å maskere sensitive data og erstatte dem med strukturelt identiske, men numerisk forskjellige data. Hva mener vi med strukturelt identiske? For eksempel er passnumre 9 sifre i USA, og enkeltpersoner må vanligvis dele passinformasjonen sin med flyselskaper. Når et flyselskap bygger en modell for å analysere og teste forretningsmiljøet, oppretter de en annen 9\sphinxhyphen{}sifret lang pass\sphinxhyphen{}ID eller erstatter noen sifre med tegn.

\sphinxAtStartPar
Her er et eksempel på hvordan datamaskering fungerer:



\sphinxAtStartPar
Egnet for testdatahåndtering
Substitusjon
I substitusjonstilnærmingen, som navnet refererer til, erstatter virksomheter de originale dataene med tilfeldige data fra levert eller tilpasset oppslagsfil. Dette er en effektiv måte å skjule data på siden bedrifter bevarer det autentiske utseendet til data.

\sphinxAtStartPar
Blander
Blanding er en annen vanlig datamaskeringsmetode. I stokkingsmetoden, akkurat som substitusjon, erstatter bedrifter originaldata med andre data som ser autentisk ut, men de blander enhetene i samme kolonne tilfeldig.

\sphinxAtStartPar
Antall og datoavvik
For økonomiske og datodrevne datasett endres ikke nøyaktigheten til datasettet når du bruker samme varians for å opprette et nytt datasett mens data maskeres. Å bruke varians for å lage et nytt datasett er også ofte brukt i syntetisk datagenerering. Hvis du planlegger å beskytte personvernet med denne teknikken, anbefaler vi deg å lese vår omfattende veiledning for generering av syntetiske data.

\sphinxAtStartPar
Kryptering
Kryptering er den mest komplekse datamaskeringsalgoritmen. Brukere kan bare få tilgang til data hvis de har dekrypteringsnøkkelen

\noindent\sphinxincludegraphics{{Encryption-data-masking}.png}

\sphinxAtStartPar
Karakterkryptering
Denne metoden innebærer å omorganisere rekkefølgen av tegn tilfeldig. Denne prosessen er irreversibel slik at de originale dataene ikke kan hentes fra de krypterte dataene.

\sphinxAtStartPar
Egnet for å dele data med uautoriserte brukere
Nuller ut eller sletting
Å erstatte sensitive data med nullverdi er også en tilnærming bedrifter kan foretrekke i deres datamaskeringsarbeid. Selv om det reduserer nøyaktigheten av testresultater som for det meste opprettholdes i andre tilnærminger, er det en enklere tilnærming når virksomheten ikke maskerer på grunn av modellvalideringsformål.

\sphinxAtStartPar
Maskering ut
I maskeringsmetoden er bare en del av de originale dataene maskert. Det ligner på nulling ut siden det ikke er effektivt i testmiljøet. For eksempel, i netthandel, vises kun de siste 4 sifrene i kredittkortnummeret til kundene for å forhindre svindel.

\noindent\sphinxincludegraphics{{/Users/m0/Documents/GitHub/MoNAV/_build/.doctrees/images/f8a8e2ccf2495d7132e5842a108e6e7a2d52ad54/data-masking.png}.webp}

\begin{figure}[htbp]
\centering
\capstart

\noindent\sphinxincludegraphics[width=200\sphinxpxdimen]{{src/dsforum/images/sVSm}.png}
\caption{This is a caption in \sphinxstylestrong{Markdown}!}\label{\detokenize{src/dsforum/fake_synth/DSF_intro:markdown-fig}}\end{figure}


\bigskip\hrule\bigskip


\noindent\sphinxincludegraphics{{Encryption-data-masking}.png}

\sphinxAtStartPar
Personvern blir stadig viktigere i dagens samfunn, og med GDPR har betydningen blitt enda større. Dette har ført til et økt behov for generering av syntetisk data for bruk i testing av nye løsninger. Men hva innebærer det egentlig å generere representativ syntetisk data som testere kan bruke, og som er i henhold til personlovgivningen? Kom og hør på vår lyntale om hvordan vi løste denne problemstillingen iÅ jobbe med data er vanskelig. Rådata byr vanligvis på flere utfordringer som må løses før du faktisk kan jobbe produktivt med dem. Noen ganger har du ikke nok data eller dataene har hull som må fylles. I mange tilfeller er det dyrt eller vanskelig å skaffe data på grunn av ytre forhold. I tillegg påvirker personvernbestemmelser måtene du kan bruke eller distribuere et datasett på. Av alle disse grunnene er bruk av syntetiske data et godt alternativ, siden det kan oppfylle de samme behovene med liten innsats.

\sphinxAtStartPar
Hva er syntetiske data?
I henhold til definisjonen satt av Storbritannias Office for National Statistics (ONS):

\sphinxAtStartPar
“Syntetiske data er mikrodataposter opprettet for å forbedre dataverktøyet og samtidig forhindre avsløring av konfidensiell respondentinformasjon. Syntetiske data lages ved å statistisk modellere originaldata, og deretter bruke disse modellene til å generere nye dataverdier som reproduserer de originale dataenes statistiske egenskaper. Brukere er ikke i stand til å identifisere informasjonen til enhetene som ga de originale dataene.”

\begin{sphinxadmonition}{note}{Definition}

\sphinxAtStartPar
“Syntetiske data er mikrodataposter opprettet for å forbedre dataverktøyet og samtidig forhindre avsløring av konfidensiell respondentinformasjon. Syntetiske data lages ved å statistisk modellere originaldata og deretter bruke disse modellene til å generere nye dataverdier som reproduserer de originale dataenes statistiske egenskaper. Brukere er ikke i stand til å identifisere informasjonen til enhetene som ga de originale dataene.”
\end{sphinxadmonition}

\sphinxAtStartPar
Således har syntetiske data følgende tre viktige egenskaper:
\begin{itemize}
\item {} 
\sphinxAtStartPar
Syntetiske data lages fra en statistisk modell.

\item {} 
\sphinxAtStartPar
De statistiske egenskapene til syntetiske data bør være lik de originale dataene.

\item {} 
\sphinxAtStartPar
Syntetiske data skal anonymiseres.

\end{itemize}

\sphinxAtStartPar
ONS\sphinxhyphen{}metodikken gir også en skala for å evaluere modenheten til et syntetisk datasett. Denne skalaen vurderer hvor mye de syntetiske dataene ligner de originale dataene, formålet og risikoen for avsløring. Metodikken inkluderer:
\begin{itemize}
\item {} 
\sphinxAtStartPar
Syntetisk strukturell: bevarer strukturen til de originale dataene, noe som er nyttig for å teste kode.

\item {} 
\sphinxAtStartPar
Syntetisk gyldig: bevarer ikke bare strukturen, men returnerer også verdier som er plausible i konteksten til datasettet. Du bør introdusere manglende verdikoder, feil og inkonsekvenser for å replikere de originale dataene.

\item {} 
\sphinxAtStartPar
Syntetisk utvidet plausibel: replikerer distribusjonene til hvert datautvalg der det er mulig uten å ta hensyn til forholdet mellom forskjellige kolonner (univariat).

\item {} 
\sphinxAtStartPar
Syntetisk utvidet multivariat plausibel: replikerer relasjoner på høyt nivå med plausible fordelinger (multivariat).

\item {} 
\sphinxAtStartPar
Syntetisk utvidet multivariat detaljert: replikerer detaljerte forhold. For denne må du utføre evaluering av informasjonskontroll fra sak til sak.

\item {} 
\sphinxAtStartPar
Syntetisk forsterket replika: gir nærmest mulig replikering. Det er avgjørende å utføre evaluering av avsløringskontroll fra sak til sak.

\end{itemize}

\sphinxAtStartPar
Hvert av de følgende bibliotekene bruker forskjellige tilnærminger til å generere syntetiske data. Noen fokuserer på å gi kun de syntetiske dataene i seg selv, men andre gir et komplett sett med verktøy som tar sikte på å oppnå den syntetisk utvidede replikaen beskrevet ovenfor.

\begin{sphinxadmonition}{note}{Note:}
\sphinxAtStartPar
The next info should be nested

\begin{sphinxadmonition}{warning}{Warning:}
\sphinxAtStartPar
Here’s my warning
\end{sphinxadmonition}
\end{sphinxadmonition}

\sphinxAtStartPar
Således har syntetiske data tre viktige egenskaper:
\begin{itemize}
\item {} 
\sphinxAtStartPar
Syntetiske data lages fra en statistisk modell.

\item {} 
\sphinxAtStartPar
De statistiske egenskapene til syntetiske data bør være lik de originale dataene.

\item {} 
\sphinxAtStartPar
Syntetiske data skal anonymiseres.

\end{itemize}

\begin{sphinxadmonition}{note}{Note:}
\sphinxAtStartPar
The warning block will be properly\sphinxhyphen{}parsed

\begin{sphinxadmonition}{warning}{Warning:}
\sphinxAtStartPar
Here’s my warning
\end{sphinxadmonition}

\sphinxAtStartPar
But the next block will be parsed as raw text

\begin{sphinxVerbatim}[commandchars=\\\{\}]
```\PYGZob{}warning\PYGZcb{}
Here\PYGZsq{}s my raw text warning that isn\PYGZsq{}t parsed...
```
\end{sphinxVerbatim}
\end{sphinxadmonition}

\begin{sphinxadmonition}{note}{My markdown link}

\sphinxAtStartPar
Here is \sphinxhref{https://jupyter.org}{markdown link syntax}
\end{sphinxadmonition}

\begin{sphinxadmonition}{note}{My markdown link}

\sphinxAtStartPar
Here is \sphinxhref{https://jupyter.org}{markdown link syntax}
\end{sphinxadmonition}
\begin{equation}\label{equation:src/dsforum/fake_synth/DSF_intro:euler}
\begin{split}e^{i\pi} + 1 = 0\end{split}
\end{equation}
\sphinxAtStartPar
Euler’s identity, equation \eqref{equation:src/dsforum/fake_synth/DSF_intro:euler}, was elected one of the
most beautiful mathematical formulas.

\sphinxAtStartPar
ONS\sphinxhyphen{}metodikken gir også en skala for å evaluere modenheten til et syntetisk datasett. Denne skalaen vurderer hvor mye de syntetiske dataene ligner de originale dataene, formålet og risikoen for avsløring. Metodikken inkluderer:

\sphinxAtStartPar
Syntetisk strukturell: bevarer strukturen til de originale dataene, noe som er nyttig for å teste kode.
Syntetisk gyldig: bevarer ikke bare strukturen, men returnerer også verdier som er plausible i konteksten til datasettet. Du bør introdusere manglende verdikoder, feil og inkonsekvenser for å replikere de originale dataene.
Syntetisk utvidet plausibel: replikerer distribusjonene til hvert datautvalg der det er mulig uten å ta hensyn til forholdet mellom forskjellige kolonner (univariat).
Syntetisk utvidet multivariat plausibel: replikerer relasjoner på høyt nivå med plausible fordelinger (multivariat).
Syntetisk utvidet multivariat detaljert: replikerer detaljerte forhold. For denne må du utføre evaluering av informasjonskontroll fra sak til sak.
Syntetisk forsterket replika: gir nærmest mulig replikering. Det er avgjørende å utføre evaluering av avsløringskontroll fra sak til sak.
Hvert av de følgende bibliotekene bruker forskjellige tilnærminger til å generere syntetiske data. Noen fokuserer på å gi kun de syntetiske dataene i seg selv, men andre gir et komplett sett med verktøy som tar sikte på å oppnå den syntetisk utvidede replikaen beskrevet ovenfor.

\sphinxAtStartPar
\sphinxhref{https://mostly.ai/blog/synthetic-data-generator-for-healthy-test-data/}{Synthetic}
\sphinxstylestrong{Synthetic test data}  is generated by AI, that is trained on real data. It is structurally representative, referential integer data with support for relational structures. AI\sphinxhyphen{}generated synthetic data is not mock data or fake data. It is as much a representation of the behavior of your customers as production data. It’s not generated manually, but by a powerful AI engine that is capable of learning all the qualities of the dataset it is trained on, providing 100\% test coverage. A good quality synthetic data generator can automate test data generation with high efficiency and without privacy concerns. Customer data should always be used in its synthetic form to protect privacy and to retain business rules embedded in the data. For example, mobile banking apps should be tested with synthetic transaction data, that is based on real customer transactions.

\sphinxAtStartPar
TL;DR
Synthetic data generation methods changed significantly with the advance of AI
AI\sphinxhyphen{}generated, sample\sphinxhyphen{}based synthetic data is an entirely different beast than random or mock data
Types of AI\sphinxhyphen{}generated synthetic data include synthetic images, synthetic text, synthetic geolocation data, categorical, numerical and time\sphinxhyphen{}series data.
Stochastic processes are still useful if you care about data structure but not content
Rule\sphinxhyphen{}based systems can be used for simple use cases with low, fixed requirements toward complexity
Use deep generative models to automatically retain structure as well as information of data at scale to unlock private data and reduce model\sphinxhyphen{}to\sphinxhyphen{}market time
What the most common synthetic data types and their most common use cases are.
An overview of data generation methods
Not all synthetic dataset is created equal and in particular, synthetic data generation today is very different from what it was 5 years ago. Let’s take a look at different methods of synthetic data generation from the most rudimental forms to the state\sphinxhyphen{}of\sphinxhyphen{}the\sphinxhyphen{}art methods to see how far the technology has advanced! In this post we will distinguish between three major methods:

\sphinxAtStartPar
The stochastic process: random data is generated, only mimicking the structure of real data.
Rule\sphinxhyphen{}based data generation: mock data is generated following specific rules defined by humans.
Deep generative models: rich and realistic synthetic data is generated by a machine learning model trained on real data, replicating its structure and the information it contains.


\chapter{Definitions}
\label{\detokenize{src/dsforum/fake_synth/DSF_intro:definitions}}
\sphinxAtStartPar
SMOTE (Synthetic Minority Oversampling Technique) to deal with imbalanced classes in a dataset.

\sphinxAtStartPar
Another well\sphinxhyphen{}known technique is called SMOTE, which involves data augmentation (i.e., synthesizing new data samples) well before you use a clas\sphinxhyphen{} sification algorithm. SMOTE was initially developed by means of the kNN algorithm (other options are available), and it can be an effective technique for handling imbalanced classes.
Yet another option to consider is the Python package imbalanced\sphinxhyphen{}learn in the scikit\sphinxhyphen{}learn\sphinxhyphen{}contrib project. This project provides various re\sphinxhyphen{}sampling techniques for datasets that exhibit class imbalance. More details are available online:
https://github.com/scikit\sphinxhyphen{}learn\sphinxhyphen{}contrib/imbalanced\sphinxhyphen{}learn.
https://www.datprof.com/solutions/synthetic\sphinxhyphen{}test\sphinxhyphen{}data\sphinxhyphen{}generation/

\sphinxAtStartPar
https://mostly.ai/wp\sphinxhyphen{}content/uploads/2021/09/MOSTLY\sphinxhyphen{}AI\_comparison\_of\_synthetic\_data\_types\_2\sphinxhyphen{}1\sphinxhyphen{}1024x724.png

\sphinxAtStartPar
WHAT IS SMOTE?
SMOTE is a technique for synthesizing new samples for a dataset. This tech\sphinxhyphen{} nique is based on linear interpolation:
Step 1: Select samples that are close in the feature space.
Step 2: Draw a line between the samples in the feature space.
Step 3: Draw a new sample at a point along that line.
A more detailed explanation of the SMOTE algorithm is as follows:
Select a random sample “a” from the minority class.
Find k nearest neighbors for that example.
Select a random neighbor “b” from the nearest neighbors.
Create a line “L” that connects “a” and “b.”
Randomly select one or more points “c” on line L.
If need be, you can repeat this process for the other (k\sphinxhyphen{}1) nearest neigh\sphinxhyphen{} bors to distribute the synthetic values more evenly among the nearest neighbors.
SMOTE Extensions
The initial SMOTE algorithm is based on the kNN classification algorithm, which has been extended in various ways, such as replacing kNN with SVM. A list of SMOTE extensions is shown as follows:
selective synthetic sample generation
Borderline\sphinxhyphen{}SMOTE (kNN)
Borderline\sphinxhyphen{}SMOTE (SVM)
Adaptive Synthetic Sampling (ADASYN)

\sphinxstepscope


\section{Anonymizers}
\label{\detokenize{src/dsforum/fake_synth/ano:anonymizers}}\label{\detokenize{src/dsforum/fake_synth/ano::doc}}
\sphinxAtStartPar
\sphinxcode{\sphinxupquote{pip install presidio\_analyzer}}

\sphinxstepscope


\section{Fake}
\label{\detokenize{src/dsforum/fake_synth/fake:fake}}\label{\detokenize{src/dsforum/fake_synth/fake::doc}}
\sphinxstepscope


\section{Syntetisk}
\label{\detokenize{src/dsforum/fake_synth/synthetic:syntetisk}}\label{\detokenize{src/dsforum/fake_synth/synthetic::doc}}
\begin{sphinxuseclass}{cell}\begin{sphinxVerbatimInput}

\begin{sphinxuseclass}{cell_input}
\begin{sphinxVerbatim}[commandchars=\\\{\}]
\PYG{k+kn}{import} \PYG{n+nn}{zipfile} \PYG{k}{as} \PYG{n+nn}{zf}
\PYG{n}{files} \PYG{o}{=} \PYG{n}{zf}\PYG{o}{.}\PYG{n}{ZipFile}\PYG{p}{(}\PYG{l+s+s2}{\PYGZdq{}}\PYG{l+s+s2}{/home/jupyter/SyntheticNAV/dsData.zip}\PYG{l+s+s2}{\PYGZdq{}}\PYG{p}{,} \PYG{l+s+s1}{\PYGZsq{}}\PYG{l+s+s1}{r}\PYG{l+s+s1}{\PYGZsq{}}\PYG{p}{)}
\PYG{n}{files}\PYG{o}{.}\PYG{n}{extractall}\PYG{p}{(}\PYG{l+s+s1}{\PYGZsq{}}\PYG{l+s+s1}{/home/jupyter/SyntheticNAV/Data}\PYG{l+s+s1}{\PYGZsq{}}\PYG{p}{)}
\PYG{n}{files}\PYG{o}{.}\PYG{n}{close}\PYG{p}{(}\PYG{p}{)}
\end{sphinxVerbatim}

\end{sphinxuseclass}\end{sphinxVerbatimInput}

\end{sphinxuseclass}
\begin{sphinxuseclass}{cell}\begin{sphinxVerbatimInput}

\begin{sphinxuseclass}{cell_input}
\begin{sphinxVerbatim}[commandchars=\\\{\}]
\PYG{k+kn}{import} \PYG{n+nn}{pandas} \PYG{k}{as} \PYG{n+nn}{pd}
\PYG{k+kn}{from} \PYG{n+nn}{DataSynthesizer}\PYG{n+nn}{.}\PYG{n+nn}{DataDescriber} \PYG{k+kn}{import} \PYG{n}{DataDescriber}
\PYG{k+kn}{from} \PYG{n+nn}{DataSynthesizer}\PYG{n+nn}{.}\PYG{n+nn}{DataGenerator} \PYG{k+kn}{import} \PYG{n}{DataGenerator}
\PYG{k+kn}{from} \PYG{n+nn}{DataSynthesizer}\PYG{n+nn}{.}\PYG{n+nn}{ModelInspector} \PYG{k+kn}{import} \PYG{n}{ModelInspector}
\PYG{k+kn}{from} \PYG{n+nn}{DataSynthesizer}\PYG{n+nn}{.}\PYG{n+nn}{lib}\PYG{n+nn}{.}\PYG{n+nn}{utils} \PYG{k+kn}{import} \PYG{n}{read\PYGZus{}json\PYGZus{}file}\PYG{p}{,} \PYG{n}{display\PYGZus{}bayesian\PYGZus{}network}
\end{sphinxVerbatim}

\end{sphinxuseclass}\end{sphinxVerbatimInput}

\end{sphinxuseclass}
\begin{sphinxuseclass}{cell}\begin{sphinxVerbatimInput}

\begin{sphinxuseclass}{cell_input}
\begin{sphinxVerbatim}[commandchars=\\\{\}]
\PYG{k}{class} \PYG{n+nc}{DataSynthesizer}\PYG{p}{:}
    \PYG{k}{def} \PYG{n+nf+fm}{\PYGZus{}\PYGZus{}init\PYGZus{}\PYGZus{}}\PYG{p}{(}\PYG{n+nb+bp}{self}\PYG{p}{,} \PYG{n}{input\PYGZus{}data}\PYG{p}{,} \PYG{n}{description\PYGZus{}file}\PYG{p}{,} \PYG{n}{synthetic\PYGZus{}data}\PYG{p}{,} \PYG{n}{mode}\PYG{p}{,}
                 \PYG{n}{threshold\PYGZus{}value}\PYG{o}{=}\PYG{l+m+mi}{20}\PYG{p}{,} \PYG{n}{num\PYGZus{}tuples\PYGZus{}to\PYGZus{}generate}\PYG{o}{=}\PYG{l+m+mi}{32561}\PYG{p}{)}\PYG{p}{:}
        \PYG{n+nb+bp}{self}\PYG{o}{.}\PYG{n}{input\PYGZus{}data} \PYG{o}{=} \PYG{n}{input\PYGZus{}data}
        \PYG{n+nb+bp}{self}\PYG{o}{.}\PYG{n}{description\PYGZus{}file} \PYG{o}{=} \PYG{n}{description\PYGZus{}file}
        \PYG{n+nb+bp}{self}\PYG{o}{.}\PYG{n}{synthetic\PYGZus{}data} \PYG{o}{=} \PYG{n}{synthetic\PYGZus{}data}
        \PYG{n+nb+bp}{self}\PYG{o}{.}\PYG{n}{mode} \PYG{o}{=} \PYG{n}{mode}
        \PYG{n+nb+bp}{self}\PYG{o}{.}\PYG{n}{threshold\PYGZus{}value} \PYG{o}{=} \PYG{n}{threshold\PYGZus{}value}
        \PYG{n+nb+bp}{self}\PYG{o}{.}\PYG{n}{num\PYGZus{}tuples\PYGZus{}to\PYGZus{}generate} \PYG{o}{=} \PYG{n}{num\PYGZus{}tuples\PYGZus{}to\PYGZus{}generate}
        
    \PYG{k}{def} \PYG{n+nf}{random\PYGZus{}data\PYGZus{}describer}\PYG{p}{(}\PYG{n+nb+bp}{self}\PYG{p}{)}\PYG{p}{:}
        \PYG{n}{describer} \PYG{o}{=} \PYG{n}{DataDescriber}\PYG{p}{(}\PYG{n}{category\PYGZus{}threshold}\PYG{o}{=}\PYG{n+nb+bp}{self}\PYG{o}{.}\PYG{n}{threshold\PYGZus{}value}\PYG{p}{)}
        \PYG{n}{describer}\PYG{o}{.}\PYG{n}{describe\PYGZus{}dataset\PYGZus{}in\PYGZus{}random\PYGZus{}mode}\PYG{p}{(}\PYG{n+nb+bp}{self}\PYG{o}{.}\PYG{n}{input\PYGZus{}data}\PYG{p}{)}
        \PYG{n}{describer}\PYG{o}{.}\PYG{n}{save\PYGZus{}dataset\PYGZus{}description\PYGZus{}to\PYGZus{}file}\PYG{p}{(}\PYG{n+nb+bp}{self}\PYG{o}{.}\PYG{n}{description\PYGZus{}file}\PYG{p}{)}
    
    \PYG{k}{def} \PYG{n+nf}{generate\PYGZus{}synthetic\PYGZus{}data}\PYG{p}{(}\PYG{n+nb+bp}{self}\PYG{p}{)}\PYG{p}{:}
        \PYG{n}{generator} \PYG{o}{=} \PYG{n}{DataGenerator}\PYG{p}{(}\PYG{p}{)}
        \PYG{n}{generator}\PYG{o}{.}\PYG{n}{generate\PYGZus{}dataset\PYGZus{}in\PYGZus{}random\PYGZus{}mode}\PYG{p}{(}\PYG{n+nb+bp}{self}\PYG{o}{.}\PYG{n}{num\PYGZus{}tuples\PYGZus{}to\PYGZus{}generate}\PYG{p}{,} \PYG{n+nb+bp}{self}\PYG{o}{.}\PYG{n}{description\PYGZus{}file}\PYG{p}{)}
        \PYG{n}{generator}\PYG{o}{.}\PYG{n}{save\PYGZus{}synthetic\PYGZus{}data}\PYG{p}{(}\PYG{n+nb+bp}{self}\PYG{o}{.}\PYG{n}{synthetic\PYGZus{}data}\PYG{p}{)}
        
    \PYG{k}{def} \PYG{n+nf}{independent\PYGZus{}data\PYGZus{}describer}\PYG{p}{(}\PYG{n+nb+bp}{self}\PYG{p}{,} \PYG{n}{categorical\PYGZus{}attributes}\PYG{p}{,} \PYG{n}{candidate\PYGZus{}keys}\PYG{p}{)}\PYG{p}{:}
        \PYG{n}{describer} \PYG{o}{=} \PYG{n}{DataDescriber}\PYG{p}{(}\PYG{n}{category\PYGZus{}threshold}\PYG{o}{=}\PYG{n+nb+bp}{self}\PYG{o}{.}\PYG{n}{threshold\PYGZus{}value}\PYG{p}{)}
        \PYG{n}{describer}\PYG{o}{.}\PYG{n}{describe\PYGZus{}dataset\PYGZus{}in\PYGZus{}independent\PYGZus{}attribute\PYGZus{}mode}\PYG{p}{(}\PYG{n}{dataset\PYGZus{}file}\PYG{o}{=}\PYG{n+nb+bp}{self}\PYG{o}{.}\PYG{n}{input\PYGZus{}data}\PYG{p}{,}
                                                         \PYG{n}{attribute\PYGZus{}to\PYGZus{}is\PYGZus{}categorical}\PYG{o}{=}\PYG{n}{categorical\PYGZus{}attributes}\PYG{p}{,}
                                                         \PYG{n}{attribute\PYGZus{}to\PYGZus{}is\PYGZus{}candidate\PYGZus{}key}\PYG{o}{=}\PYG{n}{candidate\PYGZus{}keys}\PYG{p}{)}
        \PYG{n}{describer}\PYG{o}{.}\PYG{n}{save\PYGZus{}dataset\PYGZus{}description\PYGZus{}to\PYGZus{}file}\PYG{p}{(}\PYG{n+nb+bp}{self}\PYG{o}{.}\PYG{n}{description\PYGZus{}file}\PYG{p}{)}
    
    
    \PYG{k}{def} \PYG{n+nf}{correlated\PYGZus{}data\PYGZus{}describer}\PYG{p}{(}\PYG{n+nb+bp}{self}\PYG{p}{,} \PYG{n}{categorical\PYGZus{}attributes}\PYG{p}{,} \PYG{n}{candidate\PYGZus{}keys}\PYG{p}{,} \PYG{n}{epsilon}\PYG{p}{,} \PYG{n}{degree\PYGZus{}of\PYGZus{}bayesian\PYGZus{}network}\PYG{p}{)}\PYG{p}{:}
        \PYG{n}{describer} \PYG{o}{=} \PYG{n}{DataDescriber}\PYG{p}{(}\PYG{n}{category\PYGZus{}threshold}\PYG{o}{=}\PYG{n+nb+bp}{self}\PYG{o}{.}\PYG{n}{threshold\PYGZus{}value}\PYG{p}{)}
        \PYG{n}{describer}\PYG{o}{.}\PYG{n}{describe\PYGZus{}dataset\PYGZus{}in\PYGZus{}correlated\PYGZus{}attribute\PYGZus{}mode}\PYG{p}{(}\PYG{n}{dataset\PYGZus{}file}\PYG{o}{=}\PYG{n+nb+bp}{self}\PYG{o}{.}\PYG{n}{input\PYGZus{}data}\PYG{p}{,} 
                                                        \PYG{n}{epsilon}\PYG{o}{=}\PYG{n}{epsilon}\PYG{p}{,} 
                                                        \PYG{n}{k}\PYG{o}{=}\PYG{n}{degree\PYGZus{}of\PYGZus{}bayesian\PYGZus{}network}\PYG{p}{,}
                                                        \PYG{n}{attribute\PYGZus{}to\PYGZus{}is\PYGZus{}categorical}\PYG{o}{=}\PYG{n}{categorical\PYGZus{}attributes}\PYG{p}{,}
                                                        \PYG{n}{attribute\PYGZus{}to\PYGZus{}is\PYGZus{}candidate\PYGZus{}key}\PYG{o}{=}\PYG{n}{candidate\PYGZus{}keys}\PYG{p}{)}
        \PYG{n}{describer}\PYG{o}{.}\PYG{n}{save\PYGZus{}dataset\PYGZus{}description\PYGZus{}to\PYGZus{}file}\PYG{p}{(}\PYG{n+nb+bp}{self}\PYG{o}{.}\PYG{n}{description\PYGZus{}file}\PYG{p}{)}
        
    \PYG{k}{def} \PYG{n+nf}{plot\PYGZus{}comparision}\PYG{p}{(}\PYG{n+nb+bp}{self}\PYG{p}{)}\PYG{p}{:}
        \PYG{n}{input\PYGZus{}df} \PYG{o}{=} \PYG{n}{pd}\PYG{o}{.}\PYG{n}{read\PYGZus{}csv}\PYG{p}{(}\PYG{n+nb+bp}{self}\PYG{o}{.}\PYG{n}{input\PYGZus{}data}\PYG{p}{,} \PYG{n}{skipinitialspace}\PYG{o}{=}\PYG{k+kc}{True}\PYG{p}{)}
        \PYG{n}{synthetic\PYGZus{}df} \PYG{o}{=} \PYG{n}{pd}\PYG{o}{.}\PYG{n}{read\PYGZus{}csv}\PYG{p}{(}\PYG{n+nb+bp}{self}\PYG{o}{.}\PYG{n}{synthetic\PYGZus{}data}\PYG{p}{)}
        \PYG{n}{attribute\PYGZus{}description} \PYG{o}{=} \PYG{n}{read\PYGZus{}json\PYGZus{}file}\PYG{p}{(}\PYG{n+nb+bp}{self}\PYG{o}{.}\PYG{n}{description\PYGZus{}file}\PYG{p}{)}\PYG{p}{[}\PYG{l+s+s1}{\PYGZsq{}}\PYG{l+s+s1}{attribute\PYGZus{}description}\PYG{l+s+s1}{\PYGZsq{}}\PYG{p}{]}
        \PYG{n}{inspector} \PYG{o}{=} \PYG{n}{ModelInspector}\PYG{p}{(}\PYG{n}{input\PYGZus{}df}\PYG{p}{,} \PYG{n}{synthetic\PYGZus{}df}\PYG{p}{,} \PYG{n}{attribute\PYGZus{}description}\PYG{p}{)}
        \PYG{k}{for} \PYG{n}{attribute} \PYG{o+ow}{in} \PYG{n}{synthetic\PYGZus{}df}\PYG{o}{.}\PYG{n}{columns}\PYG{p}{:}
            \PYG{n}{inspector}\PYG{o}{.}\PYG{n}{compare\PYGZus{}histograms}\PYG{p}{(}\PYG{n}{attribute}\PYG{p}{)}
        \PYG{n}{inspector}\PYG{o}{.}\PYG{n}{mutual\PYGZus{}information\PYGZus{}heatmap}\PYG{p}{(}\PYG{p}{)}
\end{sphinxVerbatim}

\end{sphinxuseclass}\end{sphinxVerbatimInput}

\end{sphinxuseclass}
\begin{sphinxuseclass}{cell}\begin{sphinxVerbatimInput}

\begin{sphinxuseclass}{cell_input}
\begin{sphinxVerbatim}[commandchars=\\\{\}]
\PYG{n}{random\PYGZus{}synthesizer} \PYG{o}{=} \PYG{n}{DataSynthesizer}\PYG{p}{(}\PYG{n}{input\PYGZus{}data}\PYG{o}{=}\PYG{l+s+s1}{\PYGZsq{}}\PYG{l+s+s1}{/home/jupyter/SyntheticNAV/Data/data/adult\PYGZus{}ssn.csv}\PYG{l+s+s1}{\PYGZsq{}}\PYG{p}{,} 
                                 \PYG{n}{description\PYGZus{}file}\PYG{o}{=}\PYG{l+s+s1}{\PYGZsq{}}\PYG{l+s+s1}{/home/jupyter/SyntheticNAV/Data/out/random\PYGZus{}mode/description.json}\PYG{l+s+s1}{\PYGZsq{}}\PYG{p}{,} 
                                 \PYG{n}{synthetic\PYGZus{}data}\PYG{o}{=}\PYG{l+s+s1}{\PYGZsq{}}\PYG{l+s+s1}{/home/jupyter/SyntheticNAV/Data/out/random\PYGZus{}mode/sythetic\PYGZus{}data.csv}\PYG{l+s+s1}{\PYGZsq{}}\PYG{p}{,} 
                                 \PYG{n}{mode}\PYG{o}{=}\PYG{l+s+s1}{\PYGZsq{}}\PYG{l+s+s1}{random\PYGZus{}mode}\PYG{l+s+s1}{\PYGZsq{}}\PYG{p}{)}
\PYG{n}{random\PYGZus{}synthesizer}\PYG{o}{.}\PYG{n}{random\PYGZus{}data\PYGZus{}describer}\PYG{p}{(}\PYG{p}{)}
\PYG{n}{random\PYGZus{}synthesizer}\PYG{o}{.}\PYG{n}{generate\PYGZus{}synthetic\PYGZus{}data}\PYG{p}{(}\PYG{p}{)}
\PYG{n}{random\PYGZus{}synthesizer}\PYG{o}{.}\PYG{n}{plot\PYGZus{}comparision}\PYG{p}{(}\PYG{p}{)}
\end{sphinxVerbatim}

\end{sphinxuseclass}\end{sphinxVerbatimInput}
\begin{sphinxVerbatimOutput}

\begin{sphinxuseclass}{cell_output}
\noindent\sphinxincludegraphics{{synthetic_4_0}.png}

\noindent\sphinxincludegraphics{{synthetic_4_1}.png}

\noindent\sphinxincludegraphics{{synthetic_4_2}.png}

\noindent\sphinxincludegraphics{{synthetic_4_3}.png}

\noindent\sphinxincludegraphics{{synthetic_4_4}.png}

\noindent\sphinxincludegraphics{{synthetic_4_5}.png}

\noindent\sphinxincludegraphics{{synthetic_4_6}.png}

\end{sphinxuseclass}\end{sphinxVerbatimOutput}

\end{sphinxuseclass}
\begin{sphinxuseclass}{cell}\begin{sphinxVerbatimInput}

\begin{sphinxuseclass}{cell_input}
\begin{sphinxVerbatim}[commandchars=\\\{\}]
\PYG{n}{independent\PYGZus{}synthesizer} \PYG{o}{=} \PYG{n}{DataSynthesizer}\PYG{p}{(}\PYG{n}{input\PYGZus{}data}\PYG{o}{=}\PYG{l+s+s1}{\PYGZsq{}}\PYG{l+s+s1}{/home/jupyter/SyntheticNAV/Data/data/adult\PYGZus{}ssn.csv}\PYG{l+s+s1}{\PYGZsq{}}\PYG{p}{,} 
                                 \PYG{n}{description\PYGZus{}file}\PYG{o}{=}\PYG{l+s+s1}{\PYGZsq{}}\PYG{l+s+s1}{/home/jupyter/SyntheticNAV/Data/out/independent\PYGZus{}attribute\PYGZus{}mode/description.json}\PYG{l+s+s1}{\PYGZsq{}}\PYG{p}{,} 
                                 \PYG{n}{synthetic\PYGZus{}data}\PYG{o}{=}\PYG{l+s+s1}{\PYGZsq{}}\PYG{l+s+s1}{/home/jupyter/SyntheticNAV/Data/out/independent\PYGZus{}attribute\PYGZus{}mode/sythetic\PYGZus{}data.csv}\PYG{l+s+s1}{\PYGZsq{}}\PYG{p}{,} 
                                     \PYG{n}{mode}\PYG{o}{=}\PYG{l+s+s1}{\PYGZsq{}}\PYG{l+s+s1}{independent\PYGZus{}mode}\PYG{l+s+s1}{\PYGZsq{}}\PYG{p}{)}

\PYG{n}{categorical\PYGZus{}attributes} \PYG{o}{=} \PYG{p}{\PYGZob{}}\PYG{l+s+s1}{\PYGZsq{}}\PYG{l+s+s1}{education}\PYG{l+s+s1}{\PYGZsq{}}\PYG{p}{:} \PYG{k+kc}{True}\PYG{p}{\PYGZcb{}}
\PYG{n}{candidate\PYGZus{}keys} \PYG{o}{=} \PYG{p}{\PYGZob{}}\PYG{l+s+s1}{\PYGZsq{}}\PYG{l+s+s1}{age}\PYG{l+s+s1}{\PYGZsq{}}\PYG{p}{:} \PYG{k+kc}{False}\PYG{p}{\PYGZcb{}}
\end{sphinxVerbatim}

\end{sphinxuseclass}\end{sphinxVerbatimInput}

\end{sphinxuseclass}
\begin{sphinxuseclass}{cell}\begin{sphinxVerbatimInput}

\begin{sphinxuseclass}{cell_input}
\begin{sphinxVerbatim}[commandchars=\\\{\}]
\PYG{n}{independent\PYGZus{}synthesizer}\PYG{o}{.}\PYG{n}{independent\PYGZus{}data\PYGZus{}describer}\PYG{p}{(}\PYG{n}{categorical\PYGZus{}attributes}\PYG{o}{=}\PYG{n}{categorical\PYGZus{}attributes}\PYG{p}{,}
                                                  \PYG{n}{candidate\PYGZus{}keys}\PYG{o}{=}\PYG{n}{candidate\PYGZus{}keys}\PYG{p}{)}

\PYG{n}{independent\PYGZus{}synthesizer}\PYG{o}{.}\PYG{n}{generate\PYGZus{}synthetic\PYGZus{}data}\PYG{p}{(}\PYG{p}{)}
\PYG{n}{independent\PYGZus{}synthesizer}\PYG{o}{.}\PYG{n}{plot\PYGZus{}comparision}\PYG{p}{(}\PYG{p}{)}
\end{sphinxVerbatim}

\end{sphinxuseclass}\end{sphinxVerbatimInput}
\begin{sphinxVerbatimOutput}

\begin{sphinxuseclass}{cell_output}
\noindent\sphinxincludegraphics{{synthetic_6_0}.png}

\noindent\sphinxincludegraphics{{synthetic_6_1}.png}

\noindent\sphinxincludegraphics{{synthetic_6_2}.png}

\noindent\sphinxincludegraphics{{synthetic_6_3}.png}

\noindent\sphinxincludegraphics{{synthetic_6_4}.png}

\noindent\sphinxincludegraphics{{synthetic_6_5}.png}

\noindent\sphinxincludegraphics{{synthetic_6_6}.png}

\end{sphinxuseclass}\end{sphinxVerbatimOutput}

\end{sphinxuseclass}
\begin{sphinxuseclass}{cell}\begin{sphinxVerbatimInput}

\begin{sphinxuseclass}{cell_input}
\begin{sphinxVerbatim}[commandchars=\\\{\}]
\PYG{n}{correlated\PYGZus{}synthesizer} \PYG{o}{=} \PYG{n}{DataSynthesizer}\PYG{p}{(}\PYG{n}{input\PYGZus{}data}\PYG{o}{=}\PYG{l+s+s1}{\PYGZsq{}}\PYG{l+s+s1}{/home/jupyter/SyntheticNAV/Data/data/adult\PYGZus{}ssn.csv}\PYG{l+s+s1}{\PYGZsq{}}\PYG{p}{,} 
                                 \PYG{n}{description\PYGZus{}file}\PYG{o}{=}\PYG{l+s+s1}{\PYGZsq{}}\PYG{l+s+s1}{/home/jupyter/SyntheticNAV/Data/out/correlated\PYGZus{}attribute\PYGZus{}mode/description.json}\PYG{l+s+s1}{\PYGZsq{}}\PYG{p}{,} 
                                 \PYG{n}{synthetic\PYGZus{}data}\PYG{o}{=}\PYG{l+s+s1}{\PYGZsq{}}\PYG{l+s+s1}{/home/jupyter/SyntheticNAV/Data/out/correlated\PYGZus{}attribute\PYGZus{}mode/sythetic\PYGZus{}data.csv}\PYG{l+s+s1}{\PYGZsq{}}\PYG{p}{,} 
                                     \PYG{n}{mode}\PYG{o}{=}\PYG{l+s+s1}{\PYGZsq{}}\PYG{l+s+s1}{correlated\PYGZus{}attribute\PYGZus{}mode}\PYG{l+s+s1}{\PYGZsq{}}\PYG{p}{)}

\PYG{n}{categorical\PYGZus{}attributes} \PYG{o}{=} \PYG{p}{\PYGZob{}}\PYG{l+s+s1}{\PYGZsq{}}\PYG{l+s+s1}{education}\PYG{l+s+s1}{\PYGZsq{}}\PYG{p}{:} \PYG{k+kc}{True}\PYG{p}{\PYGZcb{}}
\PYG{n}{candidate\PYGZus{}keys} \PYG{o}{=} \PYG{p}{\PYGZob{}}\PYG{l+s+s1}{\PYGZsq{}}\PYG{l+s+s1}{ssn}\PYG{l+s+s1}{\PYGZsq{}}\PYG{p}{:} \PYG{k+kc}{True}\PYG{p}{\PYGZcb{}}
\PYG{n}{epsilon} \PYG{o}{=} \PYG{l+m+mi}{1}
\PYG{n}{degree\PYGZus{}of\PYGZus{}bayesian\PYGZus{}network} \PYG{o}{=} \PYG{l+m+mi}{2}

\PYG{n}{correlated\PYGZus{}synthesizer}\PYG{o}{.}\PYG{n}{correlated\PYGZus{}data\PYGZus{}describer}\PYG{p}{(}\PYG{n}{categorical\PYGZus{}attributes}\PYG{o}{=}\PYG{n}{categorical\PYGZus{}attributes}\PYG{p}{,}
                                               \PYG{n}{candidate\PYGZus{}keys}\PYG{o}{=}\PYG{n}{candidate\PYGZus{}keys}\PYG{p}{,}
                                               \PYG{n}{epsilon}\PYG{o}{=}\PYG{n}{epsilon}\PYG{p}{,}
                                               \PYG{n}{degree\PYGZus{}of\PYGZus{}bayesian\PYGZus{}network}\PYG{o}{=}\PYG{n}{degree\PYGZus{}of\PYGZus{}bayesian\PYGZus{}network}\PYG{p}{)}

\PYG{n}{correlated\PYGZus{}synthesizer}\PYG{o}{.}\PYG{n}{generate\PYGZus{}synthetic\PYGZus{}data}\PYG{p}{(}\PYG{p}{)}
\PYG{n}{correlated\PYGZus{}synthesizer}\PYG{o}{.}\PYG{n}{plot\PYGZus{}comparision}\PYG{p}{(}\PYG{p}{)}
\end{sphinxVerbatim}

\end{sphinxuseclass}\end{sphinxVerbatimInput}
\begin{sphinxVerbatimOutput}

\begin{sphinxuseclass}{cell_output}
\begin{sphinxVerbatim}[commandchars=\\\{\}]
================ Constructing Bayesian Network (BN) ================
Adding ROOT relationship
Adding attribute marital\PYGZhy{}status
Adding attribute age
Adding attribute sex
Adding attribute education
Adding attribute income
========================== BN constructed ==========================
\end{sphinxVerbatim}

\noindent\sphinxincludegraphics{{synthetic_7_1}.png}

\noindent\sphinxincludegraphics{{synthetic_7_2}.png}

\noindent\sphinxincludegraphics{{synthetic_7_3}.png}

\noindent\sphinxincludegraphics{{synthetic_7_4}.png}

\noindent\sphinxincludegraphics{{synthetic_7_5}.png}

\noindent\sphinxincludegraphics{{synthetic_7_6}.png}

\noindent\sphinxincludegraphics{{synthetic_7_7}.png}

\end{sphinxuseclass}\end{sphinxVerbatimOutput}

\end{sphinxuseclass}
\begin{sphinxuseclass}{cell}\begin{sphinxVerbatimInput}

\begin{sphinxuseclass}{cell_input}
\begin{sphinxVerbatim}[commandchars=\\\{\}]
\PYG{k}{class} \PYG{n+nc}{DataSynthesizer}\PYG{p}{:}
    
\PYG{k}{class} \PYG{n+nc}{Mode}\PYG{p}{:}
    
\PYG{k}{class} \PYG{n+nc}{DataSynthesizer}\PYG{p}{:}
    
\PYG{k}{class} \PYG{n+nc}{Mode}\PYG{p}{:}
    
\end{sphinxVerbatim}

\end{sphinxuseclass}\end{sphinxVerbatimInput}

\end{sphinxuseclass}
\sphinxstepscope


\section{Simulasjoner}
\label{\detokenize{src/dsforum/fake_synth/simulations:simulasjoner}}\label{\detokenize{src/dsforum/fake_synth/simulations::doc}}
\sphinxstepscope


\part{Team Effekt}

\sphinxstepscope


\chapter{Intro POAO \sphinxhyphen{} Team Effekt}
\label{\detokenize{src/effect/intro:intro-poao-team-effekt}}\label{\detokenize{src/effect/intro::doc}}

\chapter{Referanser}
\label{\detokenize{src/effect/intro:referanser}}
\sphinxAtStartPar
https://www.sjweh.fi/show\_abstract.php?abstract\_id=3823\&fullText=1
https://www.sjweh.fi/article/3169
https://www.sjweh.fi/article/3562
https://www.sjweh.fi/article/3664
https://www.sjweh.fi/article/3780
https://www.sjweh.fi/article/1266
https://www.sjweh.fi/article/3258
https://www.sjweh.fi/article/3664

\sphinxstepscope


\chapter{Ventetid}
\label{\detokenize{src/effect/ventetid:ventetid}}\label{\detokenize{src/effect/ventetid::doc}}
\sphinxstepscope


\part{Testing}

\sphinxstepscope


\chapter{Notebooks with MyST Markdown}
\label{\detokenize{src/test/markdown-notebooks:notebooks-with-myst-markdown}}\label{\detokenize{src/test/markdown-notebooks::doc}}
\sphinxAtStartPar
Jupyter Book also lets you write text\sphinxhyphen{}based notebooks using MyST Markdown.
See \sphinxhref{https://jupyterbook.org/file-types/myst-notebooks.html}{the Notebooks with MyST Markdown documentation} for more detailed instructions.
This page shows off a notebook written in MyST Markdown.


\section{An example cell}
\label{\detokenize{src/test/markdown-notebooks:an-example-cell}}
\sphinxAtStartPar
With MyST Markdown, you can define code cells with a directive like so:

\begin{sphinxuseclass}{cell}\begin{sphinxVerbatimInput}

\begin{sphinxuseclass}{cell_input}
\begin{sphinxVerbatim}[commandchars=\\\{\}]
\PYG{n+nb}{print}\PYG{p}{(}\PYG{l+m+mi}{2} \PYG{o}{+} \PYG{l+m+mi}{2}\PYG{p}{)}
\end{sphinxVerbatim}

\end{sphinxuseclass}\end{sphinxVerbatimInput}
\begin{sphinxVerbatimOutput}

\begin{sphinxuseclass}{cell_output}
\begin{sphinxVerbatim}[commandchars=\\\{\}]
4
\end{sphinxVerbatim}

\end{sphinxuseclass}\end{sphinxVerbatimOutput}

\end{sphinxuseclass}
\sphinxAtStartPar
When your book is built, the contents of any \sphinxcode{\sphinxupquote{\{code\sphinxhyphen{}cell\}}} blocks will be
executed with your default Jupyter kernel, and their outputs will be displayed
in\sphinxhyphen{}line with the rest of your content.


\sphinxstrong{See also:}
\nopagebreak


\sphinxAtStartPar
Jupyter Book uses \sphinxhref{https://jupytext.readthedocs.io/en/latest/}{Jupytext} to convert text\sphinxhyphen{}based files to notebooks, and can support \sphinxhref{https://jupyterbook.org/file-types/jupytext.html}{many other text\sphinxhyphen{}based notebook files}.




\section{Create a notebook with MyST Markdown}
\label{\detokenize{src/test/markdown-notebooks:create-a-notebook-with-myst-markdown}}
\sphinxAtStartPar
MyST Markdown notebooks are defined by two things:
\begin{enumerate}
\sphinxsetlistlabels{\arabic}{enumi}{enumii}{}{.}%
\item {} 
\sphinxAtStartPar
YAML metadata that is needed to understand if / how it should convert text files to notebooks (including information about the kernel needed).
See the YAML at the top of this page for example.

\item {} 
\sphinxAtStartPar
The presence of \sphinxcode{\sphinxupquote{\{code\sphinxhyphen{}cell\}}} directives, which will be executed with your book.

\end{enumerate}

\sphinxAtStartPar
That’s all that is needed to get started!


\section{Quickly add YAML metadata for MyST Notebooks}
\label{\detokenize{src/test/markdown-notebooks:quickly-add-yaml-metadata-for-myst-notebooks}}
\sphinxAtStartPar
If you have a markdown file and you’d like to quickly add YAML metadata to it, so that Jupyter Book will treat it as a MyST Markdown Notebook, run the following command:

\begin{sphinxVerbatim}[commandchars=\\\{\}]
\PYG{n}{jupyter}\PYG{o}{\PYGZhy{}}\PYG{n}{book} \PYG{n}{myst} \PYG{n}{init} \PYG{n}{path}\PYG{o}{/}\PYG{n}{to}\PYG{o}{/}\PYG{n}{markdownfile}\PYG{o}{.}\PYG{n}{md}
\end{sphinxVerbatim}


\bigskip\hrule\bigskip



\section{substitutions:
key1: “I’m a \sphinxstylestrong{substitution}”
key2: |
\sphinxstyleliteralintitle{\sphinxupquote{\{note\}     \{\{ key1 \}\}     }}
fishy: |
\sphinxstyleliteralintitle{\sphinxupquote{\{image\} img/fun\sphinxhyphen{}fish.png     :alt: fishy     :width: 200px     }}}
\label{\detokenize{src/test/markdown-notebooks:substitutions-key1-im-a-substitution-key2-note-key1-fishy-image-img-fun-fish-png-alt-fishy-width-200px}}
\sphinxstepscope


\section{Markdown Files}
\label{\detokenize{src/test/markdown:markdown-files}}\label{\detokenize{src/test/markdown::doc}}
\sphinxAtStartPar
Whether you write your book’s content in Jupyter Notebooks (\sphinxcode{\sphinxupquote{.ipynb}}) or
in regular markdown files (\sphinxcode{\sphinxupquote{.md}}), you’ll write in the same flavor of markdown
called \sphinxstylestrong{MyST Markdown}.
This is a simple file to help you get started and show off some syntax.


\subsection{What is MyST?}
\label{\detokenize{src/test/markdown:what-is-myst}}
\sphinxAtStartPar
MyST stands for “Markedly Structured Text”. It
is a slight variation on a flavor of markdown called “CommonMark” markdown,
with small syntax extensions to allow you to write \sphinxstylestrong{roles} and \sphinxstylestrong{directives}
in the Sphinx ecosystem.

\sphinxAtStartPar
For more about MyST, see \sphinxhref{https://jupyterbook.org/content/myst.html}{the MyST Markdown Overview}.


\subsection{Sample Roles and Directives}
\label{\detokenize{src/test/markdown:sample-roles-and-directives}}
\sphinxAtStartPar
Roles and directives are two of the most powerful tools in Jupyter Book. They
are kind of like functions, but written in a markup language. They both
serve a similar purpose, but \sphinxstylestrong{roles are written in one line}, whereas
\sphinxstylestrong{directives span many lines}. They both accept different kinds of inputs,
and what they do with those inputs depends on the specific role or directive
that is being called.

\sphinxAtStartPar
Here is a “note” directive:

\begin{sphinxadmonition}{note}{Note:}
\sphinxAtStartPar
Here is a note
\end{sphinxadmonition}

\sphinxAtStartPar
It will be rendered in a special box when you build your book.

\sphinxAtStartPar
Here is an inline directive to refer to a document: {\hyperref[\detokenize{src/test/markdown-notebooks::doc}]{\sphinxcrossref{\DUrole{doc}{Notebooks with MyST Markdown}}}}.


\subsection{Citations}
\label{\detokenize{src/test/markdown:citations}}
\sphinxAtStartPar
You can also cite references that are stored in a \sphinxcode{\sphinxupquote{bibtex}} file. For example,
the following syntax: \sphinxcode{\sphinxupquote{\{cite\}`holdgraf\_evidence\_2014`}} will render like
this: {[}\hyperlink{cite.src/test/markdown:id3}{HdHPK14}{]}.

\sphinxAtStartPar
Moreover, you can insert a bibliography into your page with this syntax:
The \sphinxcode{\sphinxupquote{\{bibliography\}}} directive must be used for all the \sphinxcode{\sphinxupquote{\{cite\}}} roles to
render properly.
For example, if the references for your book are stored in \sphinxcode{\sphinxupquote{references.bib}},
then the bibliography is inserted with:


\subsection{Learn more}
\label{\detokenize{src/test/markdown:learn-more}}
\sphinxAtStartPar
This is just a simple starter to get you started.
You can learn a lot more at \sphinxhref{https://jupyterbook.org}{jupyterbook.org}.

\sphinxstepscope


\section{Content with notebooks}
\label{\detokenize{src/test/notebooks:content-with-notebooks}}\label{\detokenize{src/test/notebooks::doc}}
\sphinxAtStartPar
You can also create content with Jupyter Notebooks. This means that you can include
code blocks and their outputs in your book.


\subsection{Markdown + notebooks}
\label{\detokenize{src/test/notebooks:markdown-notebooks}}
\sphinxAtStartPar
As it is markdown, you can embed images, HTML, etc into your posts!

\sphinxAtStartPar
\sphinxincludegraphics{{/Users/m0/Documents/GitHub/MoNAV/_build/.doctrees/images/86f925bcfc7b65561516d54179f9c6a9096d9072/logo-wide}.svg}

\sphinxAtStartPar
You can also \(add_{math}\) and
\begin{equation*}
\begin{split}
math^{blocks}
\end{split}
\end{equation*}
\sphinxAtStartPar
or
\begin{equation*}
\begin{split}
\begin{aligned}
\mbox{mean} la_{tex} \\ \\
math blocks
\end{aligned}
\end{split}
\end{equation*}
\sphinxAtStartPar
But make sure you \$Escape \$your \$dollar signs \$you want to keep!


\subsection{MyST markdown}
\label{\detokenize{src/test/notebooks:myst-markdown}}
\sphinxAtStartPar
MyST markdown works in Jupyter Notebooks as well. For more information about MyST markdown, check
out \sphinxhref{https://jupyterbook.org/content/myst.html}{the MyST guide in Jupyter Book},
or see \sphinxhref{https://myst-parser.readthedocs.io/en/latest/}{the MyST markdown documentation}.


\subsection{Code blocks and outputs}
\label{\detokenize{src/test/notebooks:code-blocks-and-outputs}}
\sphinxAtStartPar
Jupyter Book will also embed your code blocks and output in your book.
For example, here’s some sample Matplotlib code:

\begin{sphinxuseclass}{cell}\begin{sphinxVerbatimInput}

\begin{sphinxuseclass}{cell_input}
\begin{sphinxVerbatim}[commandchars=\\\{\}]
\PYG{k+kn}{from} \PYG{n+nn}{matplotlib} \PYG{k+kn}{import} \PYG{n}{rcParams}\PYG{p}{,} \PYG{n}{cycler}
\PYG{k+kn}{import} \PYG{n+nn}{matplotlib}\PYG{n+nn}{.}\PYG{n+nn}{pyplot} \PYG{k}{as} \PYG{n+nn}{plt}
\PYG{k+kn}{import} \PYG{n+nn}{numpy} \PYG{k}{as} \PYG{n+nn}{np}
\PYG{n}{plt}\PYG{o}{.}\PYG{n}{ion}\PYG{p}{(}\PYG{p}{)}
\end{sphinxVerbatim}

\end{sphinxuseclass}\end{sphinxVerbatimInput}
\begin{sphinxVerbatimOutput}

\begin{sphinxuseclass}{cell_output}
\begin{sphinxVerbatim}[commandchars=\\\{\}]
\PYGZlt{}matplotlib.pyplot.\PYGZus{}IonContext at 0x7f15af527e10\PYGZgt{}
\end{sphinxVerbatim}

\end{sphinxuseclass}\end{sphinxVerbatimOutput}

\end{sphinxuseclass}
\begin{sphinxuseclass}{cell}\begin{sphinxVerbatimInput}

\begin{sphinxuseclass}{cell_input}
\begin{sphinxVerbatim}[commandchars=\\\{\}]
\PYG{c+c1}{\PYGZsh{} Fixing random state for reproducibility}
\PYG{n}{np}\PYG{o}{.}\PYG{n}{random}\PYG{o}{.}\PYG{n}{seed}\PYG{p}{(}\PYG{l+m+mi}{19680801}\PYG{p}{)}

\PYG{n}{N} \PYG{o}{=} \PYG{l+m+mi}{10}
\PYG{n}{data} \PYG{o}{=} \PYG{p}{[}\PYG{n}{np}\PYG{o}{.}\PYG{n}{logspace}\PYG{p}{(}\PYG{l+m+mi}{0}\PYG{p}{,} \PYG{l+m+mi}{1}\PYG{p}{,} \PYG{l+m+mi}{100}\PYG{p}{)} \PYG{o}{+} \PYG{n}{np}\PYG{o}{.}\PYG{n}{random}\PYG{o}{.}\PYG{n}{randn}\PYG{p}{(}\PYG{l+m+mi}{100}\PYG{p}{)} \PYG{o}{+} \PYG{n}{ii} \PYG{k}{for} \PYG{n}{ii} \PYG{o+ow}{in} \PYG{n+nb}{range}\PYG{p}{(}\PYG{n}{N}\PYG{p}{)}\PYG{p}{]}
\PYG{n}{data} \PYG{o}{=} \PYG{n}{np}\PYG{o}{.}\PYG{n}{array}\PYG{p}{(}\PYG{n}{data}\PYG{p}{)}\PYG{o}{.}\PYG{n}{T}
\PYG{n}{cmap} \PYG{o}{=} \PYG{n}{plt}\PYG{o}{.}\PYG{n}{cm}\PYG{o}{.}\PYG{n}{coolwarm}
\PYG{n}{rcParams}\PYG{p}{[}\PYG{l+s+s1}{\PYGZsq{}}\PYG{l+s+s1}{axes.prop\PYGZus{}cycle}\PYG{l+s+s1}{\PYGZsq{}}\PYG{p}{]} \PYG{o}{=} \PYG{n}{cycler}\PYG{p}{(}\PYG{n}{color}\PYG{o}{=}\PYG{n}{cmap}\PYG{p}{(}\PYG{n}{np}\PYG{o}{.}\PYG{n}{linspace}\PYG{p}{(}\PYG{l+m+mi}{0}\PYG{p}{,} \PYG{l+m+mi}{1}\PYG{p}{,} \PYG{n}{N}\PYG{p}{)}\PYG{p}{)}\PYG{p}{)}


\PYG{k+kn}{from} \PYG{n+nn}{matplotlib}\PYG{n+nn}{.}\PYG{n+nn}{lines} \PYG{k+kn}{import} \PYG{n}{Line2D}
\PYG{n}{custom\PYGZus{}lines} \PYG{o}{=} \PYG{p}{[}\PYG{n}{Line2D}\PYG{p}{(}\PYG{p}{[}\PYG{l+m+mi}{0}\PYG{p}{]}\PYG{p}{,} \PYG{p}{[}\PYG{l+m+mi}{0}\PYG{p}{]}\PYG{p}{,} \PYG{n}{color}\PYG{o}{=}\PYG{n}{cmap}\PYG{p}{(}\PYG{l+m+mf}{0.}\PYG{p}{)}\PYG{p}{,} \PYG{n}{lw}\PYG{o}{=}\PYG{l+m+mi}{4}\PYG{p}{)}\PYG{p}{,}
                \PYG{n}{Line2D}\PYG{p}{(}\PYG{p}{[}\PYG{l+m+mi}{0}\PYG{p}{]}\PYG{p}{,} \PYG{p}{[}\PYG{l+m+mi}{0}\PYG{p}{]}\PYG{p}{,} \PYG{n}{color}\PYG{o}{=}\PYG{n}{cmap}\PYG{p}{(}\PYG{l+m+mf}{.5}\PYG{p}{)}\PYG{p}{,} \PYG{n}{lw}\PYG{o}{=}\PYG{l+m+mi}{4}\PYG{p}{)}\PYG{p}{,}
                \PYG{n}{Line2D}\PYG{p}{(}\PYG{p}{[}\PYG{l+m+mi}{0}\PYG{p}{]}\PYG{p}{,} \PYG{p}{[}\PYG{l+m+mi}{0}\PYG{p}{]}\PYG{p}{,} \PYG{n}{color}\PYG{o}{=}\PYG{n}{cmap}\PYG{p}{(}\PYG{l+m+mf}{1.}\PYG{p}{)}\PYG{p}{,} \PYG{n}{lw}\PYG{o}{=}\PYG{l+m+mi}{4}\PYG{p}{)}\PYG{p}{]}

\PYG{n}{fig}\PYG{p}{,} \PYG{n}{ax} \PYG{o}{=} \PYG{n}{plt}\PYG{o}{.}\PYG{n}{subplots}\PYG{p}{(}\PYG{n}{figsize}\PYG{o}{=}\PYG{p}{(}\PYG{l+m+mi}{10}\PYG{p}{,} \PYG{l+m+mi}{5}\PYG{p}{)}\PYG{p}{)}
\PYG{n}{lines} \PYG{o}{=} \PYG{n}{ax}\PYG{o}{.}\PYG{n}{plot}\PYG{p}{(}\PYG{n}{data}\PYG{p}{)}
\PYG{n}{ax}\PYG{o}{.}\PYG{n}{legend}\PYG{p}{(}\PYG{n}{custom\PYGZus{}lines}\PYG{p}{,} \PYG{p}{[}\PYG{l+s+s1}{\PYGZsq{}}\PYG{l+s+s1}{Cold}\PYG{l+s+s1}{\PYGZsq{}}\PYG{p}{,} \PYG{l+s+s1}{\PYGZsq{}}\PYG{l+s+s1}{Medium}\PYG{l+s+s1}{\PYGZsq{}}\PYG{p}{,} \PYG{l+s+s1}{\PYGZsq{}}\PYG{l+s+s1}{Hot}\PYG{l+s+s1}{\PYGZsq{}}\PYG{p}{]}\PYG{p}{)}\PYG{p}{;}
\end{sphinxVerbatim}

\end{sphinxuseclass}\end{sphinxVerbatimInput}
\begin{sphinxVerbatimOutput}

\begin{sphinxuseclass}{cell_output}
\noindent\sphinxincludegraphics{{notebooks_3_0}.png}

\end{sphinxuseclass}\end{sphinxVerbatimOutput}

\end{sphinxuseclass}
\sphinxAtStartPar
There is a lot more that you can do with outputs (such as including interactive outputs)
with your book. For more information about this, see \sphinxhref{https://jupyterbook.org}{the Jupyter Book documentation}

\begin{sphinxuseclass}{cell}\begin{sphinxVerbatimInput}

\begin{sphinxuseclass}{cell_input}
\begin{sphinxVerbatim}[commandchars=\\\{\}]
\PYG{k+kn}{from} \PYG{n+nn}{google}\PYG{n+nn}{.}\PYG{n+nn}{cloud} \PYG{k+kn}{import} \PYG{n}{bigquery}

\PYG{c+c1}{\PYGZsh{} Construct a BigQuery client object.}
\PYG{n}{client} \PYG{o}{=} \PYG{n}{bigquery}\PYG{o}{.}\PYG{n}{Client}\PYG{p}{(}\PYG{p}{)}

\PYG{c+c1}{\PYGZsh{} TODO(developer): Set table\PYGZus{}id to the fully\PYGZhy{}qualified table ID in standard}
\PYG{c+c1}{\PYGZsh{} SQL format, including the project ID and dataset ID.}
\PYG{n}{table\PYGZus{}id} \PYG{o}{=} \PYG{l+s+s2}{\PYGZdq{}}\PYG{l+s+s2}{bigquery\PYGZhy{}public\PYGZhy{}data.usa\PYGZus{}names.usa\PYGZus{}1910\PYGZus{}current}\PYG{l+s+s2}{\PYGZdq{}}

\PYG{c+c1}{\PYGZsh{} Use the BigQuery Storage API to speed\PYGZhy{}up downloads of large tables.}
\PYG{n}{dataframe} \PYG{o}{=} \PYG{n}{client}\PYG{o}{.}\PYG{n}{list\PYGZus{}rows}\PYG{p}{(}\PYG{n}{table\PYGZus{}id}\PYG{p}{)}\PYG{o}{.}\PYG{n}{to\PYGZus{}dataframe}\PYG{p}{(}\PYG{n}{create\PYGZus{}bqstorage\PYGZus{}client}\PYG{o}{=}\PYG{k+kc}{True}\PYG{p}{)}

\PYG{n+nb}{print}\PYG{p}{(}\PYG{n}{dataframe}\PYG{o}{.}\PYG{n}{info}\PYG{p}{(}\PYG{p}{)}\PYG{p}{)}
\end{sphinxVerbatim}

\end{sphinxuseclass}\end{sphinxVerbatimInput}
\begin{sphinxVerbatimOutput}

\begin{sphinxuseclass}{cell_output}
\begin{sphinxVerbatim}[commandchars=\\\{\}]
\PYGZlt{}class \PYGZsq{}pandas.core.frame.DataFrame\PYGZsq{}\PYGZgt{}
RangeIndex: 6311504 entries, 0 to 6311503
Data columns (total 5 columns):
 \PYGZsh{}   Column  Dtype 
\PYGZhy{}\PYGZhy{}\PYGZhy{}  \PYGZhy{}\PYGZhy{}\PYGZhy{}\PYGZhy{}\PYGZhy{}\PYGZhy{}  \PYGZhy{}\PYGZhy{}\PYGZhy{}\PYGZhy{}\PYGZhy{} 
 0   state   object
 1   gender  object
 2   year    int64 
 3   name    object
 4   number  int64 
dtypes: int64(2), object(3)
memory usage: 240.8+ MB
None
\end{sphinxVerbatim}

\end{sphinxuseclass}\end{sphinxVerbatimOutput}

\end{sphinxuseclass}
\sphinxstepscope


\chapter{DataSynthesizer (random mode)}
\label{\detokenize{src/test/SynthNAV0:datasynthesizer-random-mode}}\label{\detokenize{src/test/SynthNAV0::doc}}
\sphinxAtStartPar
DataSynthesizer generates synthetic data that simulates a given dataset.

\sphinxAtStartPar
It aims to facilitate the collaborations between data scientists and owners of sensitive data. It applies Differential Privacy techniques to achieve strong privacy guarantee.
\#https://github.com/DataResponsibly/DataSynthesizer/blob/master/docs/cr\sphinxhyphen{}datasynthesizer\sphinxhyphen{}privacy.pdf


\section{Step 1 import packages}
\label{\detokenize{src/test/SynthNAV0:step-1-import-packages}}
\begin{sphinxuseclass}{cell}\begin{sphinxVerbatimInput}

\begin{sphinxuseclass}{cell_input}
\begin{sphinxVerbatim}[commandchars=\\\{\}]
\PYG{k+kn}{from} \PYG{n+nn}{DataSynthesizer}\PYG{n+nn}{.}\PYG{n+nn}{DataDescriber} \PYG{k+kn}{import} \PYG{n}{DataDescriber}
\PYG{k+kn}{from} \PYG{n+nn}{DataSynthesizer}\PYG{n+nn}{.}\PYG{n+nn}{DataGenerator} \PYG{k+kn}{import} \PYG{n}{DataGenerator}
\PYG{k+kn}{from} \PYG{n+nn}{DataSynthesizer}\PYG{n+nn}{.}\PYG{n+nn}{ModelInspector} \PYG{k+kn}{import} \PYG{n}{ModelInspector}
\PYG{k+kn}{from} \PYG{n+nn}{DataSynthesizer}\PYG{n+nn}{.}\PYG{n+nn}{lib}\PYG{n+nn}{.}\PYG{n+nn}{utils} \PYG{k+kn}{import} \PYG{n}{read\PYGZus{}json\PYGZus{}file}\PYG{p}{,} \PYG{n}{display\PYGZus{}bayesian\PYGZus{}network}

\PYG{k+kn}{import} \PYG{n+nn}{pandas} \PYG{k}{as} \PYG{n+nn}{pd}
\end{sphinxVerbatim}

\end{sphinxuseclass}\end{sphinxVerbatimInput}

\end{sphinxuseclass}
\begin{sphinxuseclass}{cell}\begin{sphinxVerbatimInput}

\begin{sphinxuseclass}{cell_input}
\begin{sphinxVerbatim}[commandchars=\\\{\}]
\PYG{c+c1}{\PYGZsh{} help(DataDescriber)}
\PYG{c+c1}{\PYGZsh{}print(dir(DataDescriber))}
\PYG{n+nb}{print}\PYG{p}{(}\PYG{n+nb}{id}\PYG{p}{(}\PYG{n}{DataDescriber}\PYG{p}{)}\PYG{p}{)}
\end{sphinxVerbatim}

\end{sphinxuseclass}\end{sphinxVerbatimInput}
\begin{sphinxVerbatimOutput}

\begin{sphinxuseclass}{cell_output}
\begin{sphinxVerbatim}[commandchars=\\\{\}]
5303097408
\end{sphinxVerbatim}

\end{sphinxuseclass}\end{sphinxVerbatimOutput}

\end{sphinxuseclass}
\begin{sphinxuseclass}{cell}\begin{sphinxVerbatimInput}

\begin{sphinxuseclass}{cell_input}
\begin{sphinxVerbatim}[commandchars=\\\{\}]
\PYG{c+c1}{\PYGZsh{} hasattr(obj, name)}
\PYG{c+c1}{\PYGZsh{} obj is the object under inspection.}
\PYG{c+c1}{\PYGZsh{} name is the name (as a string) of the possible attribute.}
\PYG{n+nb}{hasattr}\PYG{p}{(}\PYG{n}{DataDescriber}\PYG{p}{,} \PYG{l+s+s2}{\PYGZdq{}}\PYG{l+s+s2}{type(int)}\PYG{l+s+s2}{\PYGZdq{}}\PYG{p}{)}
\end{sphinxVerbatim}

\end{sphinxuseclass}\end{sphinxVerbatimInput}
\begin{sphinxVerbatimOutput}

\begin{sphinxuseclass}{cell_output}
\begin{sphinxVerbatim}[commandchars=\\\{\}]
False
\end{sphinxVerbatim}

\end{sphinxuseclass}\end{sphinxVerbatimOutput}

\end{sphinxuseclass}

\section{Step 2 user\sphinxhyphen{}defined parameteres}
\label{\detokenize{src/test/SynthNAV0:step-2-user-defined-parameteres}}
\begin{sphinxuseclass}{cell}\begin{sphinxVerbatimInput}

\begin{sphinxuseclass}{cell_input}
\begin{sphinxVerbatim}[commandchars=\\\{\}]
\PYG{c+c1}{\PYGZsh{} input dataset}
\PYG{c+c1}{\PYGZsh{}input\PYGZus{}data = \PYGZsq{}./data/adult\PYGZus{}ssn.csv\PYGZsq{}}
\PYG{n}{input\PYGZus{}data} \PYG{o}{=} \PYG{l+s+s1}{\PYGZsq{}}\PYG{l+s+s1}{/Users/m0/Documents/GitHub/MoHoushmand.github.io/DATA/data/adult\PYGZus{}ssn.csv}\PYG{l+s+s1}{\PYGZsq{}}
\PYG{c+c1}{\PYGZsh{} location of two output files}
\PYG{n}{mode} \PYG{o}{=} \PYG{l+s+s1}{\PYGZsq{}}\PYG{l+s+s1}{random\PYGZus{}mode}\PYG{l+s+s1}{\PYGZsq{}}
\PYG{c+c1}{\PYGZsh{}description\PYGZus{}file =  f\PYGZsq{}.DATA/out/\PYGZob{}mode\PYGZcb{}/description.json\PYGZsq{}}
\PYG{n}{description\PYGZus{}file} \PYG{o}{=} \PYG{l+s+sa}{f}\PYG{l+s+s1}{\PYGZsq{}}\PYG{l+s+s1}{/Users/m0/Documents/GitHub/MoHoushmand.github.io/DATA/out/}\PYG{l+s+si}{\PYGZob{}}\PYG{n}{mode}\PYG{l+s+si}{\PYGZcb{}}\PYG{l+s+s1}{/description.json}\PYG{l+s+s1}{\PYGZsq{}}
\PYG{c+c1}{\PYGZsh{}synthetic\PYGZus{}data = f\PYGZsq{}.DATA/out/\PYGZob{}mode\PYGZcb{}/sythetic\PYGZus{}data.csv\PYGZsq{}}
\PYG{n}{synthetic\PYGZus{}data} \PYG{o}{=} \PYG{l+s+sa}{f}\PYG{l+s+s1}{\PYGZsq{}}\PYG{l+s+s1}{/Users/m0/Documents/GitHub/MoHoushmand.github.io/DATA/out/}\PYG{l+s+si}{\PYGZob{}}\PYG{n}{mode}\PYG{l+s+si}{\PYGZcb{}}\PYG{l+s+s1}{/sythetic\PYGZus{}data.csv}\PYG{l+s+s1}{\PYGZsq{}}
\end{sphinxVerbatim}

\end{sphinxuseclass}\end{sphinxVerbatimInput}

\end{sphinxuseclass}
\begin{sphinxuseclass}{cell}\begin{sphinxVerbatimInput}

\begin{sphinxuseclass}{cell_input}
\begin{sphinxVerbatim}[commandchars=\\\{\}]
\PYG{c+c1}{\PYGZsh{} An attribute is categorical if its domain size is less than this threshold.}
\PYG{c+c1}{\PYGZsh{} Here modify the threshold to adapt to the domain size of \PYGZdq{}education\PYGZdq{} (which is 14 in input dataset).}
\PYG{n}{threshold\PYGZus{}value} \PYG{o}{=} \PYG{l+m+mi}{20} 

\PYG{c+c1}{\PYGZsh{} Number of tuples generated in synthetic dataset.}
\PYG{n}{num\PYGZus{}tuples\PYGZus{}to\PYGZus{}generate} \PYG{o}{=} \PYG{l+m+mi}{32561} \PYG{c+c1}{\PYGZsh{} Here 32561 is the same as input dataset, but it can be set to another number.}
\end{sphinxVerbatim}

\end{sphinxuseclass}\end{sphinxVerbatimInput}

\end{sphinxuseclass}

\section{Step 3 DataDescriber}
\label{\detokenize{src/test/SynthNAV0:step-3-datadescriber}}\begin{enumerate}
\sphinxsetlistlabels{\arabic}{enumi}{enumii}{}{.}%
\item {} 
\sphinxAtStartPar
Instantiate a DataDescriber.

\item {} 
\sphinxAtStartPar
Compute the statistics of the dataset.

\item {} 
\sphinxAtStartPar
Save dataset description to a file on local machine.

\end{enumerate}

\begin{sphinxuseclass}{cell}\begin{sphinxVerbatimInput}

\begin{sphinxuseclass}{cell_input}
\begin{sphinxVerbatim}[commandchars=\\\{\}]
\PYG{n}{describer} \PYG{o}{=} \PYG{n}{DataDescriber}\PYG{p}{(}\PYG{n}{category\PYGZus{}threshold}\PYG{o}{=}\PYG{n}{threshold\PYGZus{}value}\PYG{p}{)}
\PYG{n}{describer}\PYG{o}{.}\PYG{n}{describe\PYGZus{}dataset\PYGZus{}in\PYGZus{}random\PYGZus{}mode}\PYG{p}{(}\PYG{n}{input\PYGZus{}data}\PYG{p}{)}
\PYG{n}{describer}\PYG{o}{.}\PYG{n}{save\PYGZus{}dataset\PYGZus{}description\PYGZus{}to\PYGZus{}file}\PYG{p}{(}\PYG{n}{description\PYGZus{}file}\PYG{p}{)}
\end{sphinxVerbatim}

\end{sphinxuseclass}\end{sphinxVerbatimInput}

\end{sphinxuseclass}

\section{Step 4 generate synthetic dataset}
\label{\detokenize{src/test/SynthNAV0:step-4-generate-synthetic-dataset}}\begin{enumerate}
\sphinxsetlistlabels{\arabic}{enumi}{enumii}{}{.}%
\item {} 
\sphinxAtStartPar
Instantiate a DataGenerator.

\item {} 
\sphinxAtStartPar
Generate a synthetic dataset.

\item {} 
\sphinxAtStartPar
Save it to local machine.

\end{enumerate}

\begin{sphinxuseclass}{cell}\begin{sphinxVerbatimInput}

\begin{sphinxuseclass}{cell_input}
\begin{sphinxVerbatim}[commandchars=\\\{\}]
\PYG{n}{generator} \PYG{o}{=} \PYG{n}{DataGenerator}\PYG{p}{(}\PYG{p}{)}
\PYG{n}{generator}\PYG{o}{.}\PYG{n}{generate\PYGZus{}dataset\PYGZus{}in\PYGZus{}random\PYGZus{}mode}\PYG{p}{(}\PYG{n}{num\PYGZus{}tuples\PYGZus{}to\PYGZus{}generate}\PYG{p}{,} \PYG{n}{description\PYGZus{}file}\PYG{p}{)}
\PYG{n}{generator}\PYG{o}{.}\PYG{n}{save\PYGZus{}synthetic\PYGZus{}data}\PYG{p}{(}\PYG{n}{synthetic\PYGZus{}data}\PYG{p}{)}
\end{sphinxVerbatim}

\end{sphinxuseclass}\end{sphinxVerbatimInput}

\end{sphinxuseclass}

\section{Step 5 compare the statistics of input and sythetic data (optional)}
\label{\detokenize{src/test/SynthNAV0:step-5-compare-the-statistics-of-input-and-sythetic-data-optional}}
\sphinxAtStartPar
The synthetic data is already saved in a file by step 4. The ModelInspector is for a quick test on the similarity between input and synthetic datasets.


\subsection{5.1 instantiate a ModelInspector.}
\label{\detokenize{src/test/SynthNAV0:instantiate-a-modelinspector}}
\sphinxAtStartPar
It needs input dataset, synthetic dataset, and attribute description.

\begin{sphinxuseclass}{cell}\begin{sphinxVerbatimInput}

\begin{sphinxuseclass}{cell_input}
\begin{sphinxVerbatim}[commandchars=\\\{\}]
\PYG{c+c1}{\PYGZsh{} Read both datasets using Pandas.}
\PYG{n}{input\PYGZus{}df} \PYG{o}{=} \PYG{n}{pd}\PYG{o}{.}\PYG{n}{read\PYGZus{}csv}\PYG{p}{(}\PYG{n}{input\PYGZus{}data}\PYG{p}{,} \PYG{n}{skipinitialspace}\PYG{o}{=}\PYG{k+kc}{True}\PYG{p}{)}
\PYG{n}{synthetic\PYGZus{}df} \PYG{o}{=} \PYG{n}{pd}\PYG{o}{.}\PYG{n}{read\PYGZus{}csv}\PYG{p}{(}\PYG{n}{synthetic\PYGZus{}data}\PYG{p}{)}
\PYG{c+c1}{\PYGZsh{} Read attribute description from the dataset description file.}
\PYG{n}{attribute\PYGZus{}description} \PYG{o}{=} \PYG{n}{read\PYGZus{}json\PYGZus{}file}\PYG{p}{(}\PYG{n}{description\PYGZus{}file}\PYG{p}{)}\PYG{p}{[}\PYG{l+s+s1}{\PYGZsq{}}\PYG{l+s+s1}{attribute\PYGZus{}description}\PYG{l+s+s1}{\PYGZsq{}}\PYG{p}{]}

\PYG{n}{inspector} \PYG{o}{=} \PYG{n}{ModelInspector}\PYG{p}{(}\PYG{n}{input\PYGZus{}df}\PYG{p}{,} \PYG{n}{synthetic\PYGZus{}df}\PYG{p}{,} \PYG{n}{attribute\PYGZus{}description}\PYG{p}{)}
\end{sphinxVerbatim}

\end{sphinxuseclass}\end{sphinxVerbatimInput}

\end{sphinxuseclass}

\subsection{5.2 compare histograms between input and synthetic datasets.}
\label{\detokenize{src/test/SynthNAV0:compare-histograms-between-input-and-synthetic-datasets}}
\begin{sphinxuseclass}{cell}\begin{sphinxVerbatimInput}

\begin{sphinxuseclass}{cell_input}
\begin{sphinxVerbatim}[commandchars=\\\{\}]
\PYG{k}{for} \PYG{n}{attribute} \PYG{o+ow}{in} \PYG{n}{synthetic\PYGZus{}df}\PYG{o}{.}\PYG{n}{columns}\PYG{p}{:}
    \PYG{n}{inspector}\PYG{o}{.}\PYG{n}{compare\PYGZus{}histograms}\PYG{p}{(}\PYG{n}{attribute}\PYG{p}{)}
\end{sphinxVerbatim}

\end{sphinxuseclass}\end{sphinxVerbatimInput}
\begin{sphinxVerbatimOutput}

\begin{sphinxuseclass}{cell_output}
\begin{sphinxVerbatim}[commandchars=\\\{\}]
/Users/m0/miniforge3/envs/mambamojb/lib/python3.9/site\PYGZhy{}packages/DataSynthesizer/ModelInspector.py:82: FutureWarning: iteritems is deprecated and will be removed in a future version. Use .items instead.
  for idx, number in dist\PYGZus{}priv.iteritems():
/Users/m0/miniforge3/envs/mambamojb/lib/python3.9/site\PYGZhy{}packages/DataSynthesizer/ModelInspector.py:85: FutureWarning: iteritems is deprecated and will be removed in a future version. Use .items instead.
  for idx, number in dist\PYGZus{}synt.iteritems():
/Users/m0/miniforge3/envs/mambamojb/lib/python3.9/site\PYGZhy{}packages/DataSynthesizer/ModelInspector.py:82: FutureWarning: iteritems is deprecated and will be removed in a future version. Use .items instead.
  for idx, number in dist\PYGZus{}priv.iteritems():
/Users/m0/miniforge3/envs/mambamojb/lib/python3.9/site\PYGZhy{}packages/DataSynthesizer/ModelInspector.py:85: FutureWarning: iteritems is deprecated and will be removed in a future version. Use .items instead.
  for idx, number in dist\PYGZus{}synt.iteritems():
/Users/m0/miniforge3/envs/mambamojb/lib/python3.9/site\PYGZhy{}packages/DataSynthesizer/ModelInspector.py:82: FutureWarning: iteritems is deprecated and will be removed in a future version. Use .items instead.
  for idx, number in dist\PYGZus{}priv.iteritems():
/Users/m0/miniforge3/envs/mambamojb/lib/python3.9/site\PYGZhy{}packages/DataSynthesizer/ModelInspector.py:85: FutureWarning: iteritems is deprecated and will be removed in a future version. Use .items instead.
  for idx, number in dist\PYGZus{}synt.iteritems():
/Users/m0/miniforge3/envs/mambamojb/lib/python3.9/site\PYGZhy{}packages/DataSynthesizer/ModelInspector.py:82: FutureWarning: iteritems is deprecated and will be removed in a future version. Use .items instead.
  for idx, number in dist\PYGZus{}priv.iteritems():
/Users/m0/miniforge3/envs/mambamojb/lib/python3.9/site\PYGZhy{}packages/DataSynthesizer/ModelInspector.py:85: FutureWarning: iteritems is deprecated and will be removed in a future version. Use .items instead.
  for idx, number in dist\PYGZus{}synt.iteritems():
/Users/m0/miniforge3/envs/mambamojb/lib/python3.9/site\PYGZhy{}packages/DataSynthesizer/ModelInspector.py:82: FutureWarning: iteritems is deprecated and will be removed in a future version. Use .items instead.
  for idx, number in dist\PYGZus{}priv.iteritems():
/Users/m0/miniforge3/envs/mambamojb/lib/python3.9/site\PYGZhy{}packages/DataSynthesizer/ModelInspector.py:85: FutureWarning: iteritems is deprecated and will be removed in a future version. Use .items instead.
  for idx, number in dist\PYGZus{}synt.iteritems():
\end{sphinxVerbatim}

\noindent\sphinxincludegraphics{{SynthNAV0_17_1}.png}

\noindent\sphinxincludegraphics{{SynthNAV0_17_2}.png}

\noindent\sphinxincludegraphics{{SynthNAV0_17_3}.png}

\noindent\sphinxincludegraphics{{SynthNAV0_17_4}.png}

\noindent\sphinxincludegraphics{{SynthNAV0_17_5}.png}

\noindent\sphinxincludegraphics{{SynthNAV0_17_6}.png}

\end{sphinxuseclass}\end{sphinxVerbatimOutput}

\end{sphinxuseclass}
\begin{sphinxuseclass}{cell}\begin{sphinxVerbatimInput}

\begin{sphinxuseclass}{cell_input}
\begin{sphinxVerbatim}[commandchars=\\\{\}]
\PYG{n}{inspector}\PYG{o}{.}\PYG{n}{mutual\PYGZus{}information\PYGZus{}heatmap}\PYG{p}{(}\PYG{p}{)}
\end{sphinxVerbatim}

\end{sphinxuseclass}\end{sphinxVerbatimInput}
\begin{sphinxVerbatimOutput}

\begin{sphinxuseclass}{cell_output}
\noindent\sphinxincludegraphics{{SynthNAV0_18_0}.png}

\end{sphinxuseclass}\end{sphinxVerbatimOutput}

\end{sphinxuseclass}

\chapter{DataSynthesizer Usage (independent attribute mode)}
\label{\detokenize{src/test/SynthNAV0:datasynthesizer-usage-independent-attribute-mode}}

\section{Step 1 import packages}
\label{\detokenize{src/test/SynthNAV0:id1}}

\section{Step 2 user\sphinxhyphen{}defined parameteres}
\label{\detokenize{src/test/SynthNAV0:id2}}
\begin{sphinxuseclass}{cell}\begin{sphinxVerbatimInput}

\begin{sphinxuseclass}{cell_input}
\begin{sphinxVerbatim}[commandchars=\\\{\}]
\PYG{c+c1}{\PYGZsh{} location of two output files}
\PYG{n}{mode} \PYG{o}{=} \PYG{l+s+s1}{\PYGZsq{}}\PYG{l+s+s1}{independent\PYGZus{}attribute\PYGZus{}mode}\PYG{l+s+s1}{\PYGZsq{}}
\PYG{n}{description\PYGZus{}file} \PYG{o}{=} \PYG{l+s+sa}{f}\PYG{l+s+s1}{\PYGZsq{}}\PYG{l+s+s1}{/Users/m0/Documents/GitHub/MoHoushmand.github.io/DATA/out/}\PYG{l+s+si}{\PYGZob{}}\PYG{n}{mode}\PYG{l+s+si}{\PYGZcb{}}\PYG{l+s+s1}{/description.json}\PYG{l+s+s1}{\PYGZsq{}}
\PYG{n}{synthetic\PYGZus{}data} \PYG{o}{=} \PYG{l+s+sa}{f}\PYG{l+s+s1}{\PYGZsq{}}\PYG{l+s+s1}{/Users/m0/Documents/GitHub/MoHoushmand.github.io/DATA/out/}\PYG{l+s+si}{\PYGZob{}}\PYG{n}{mode}\PYG{l+s+si}{\PYGZcb{}}\PYG{l+s+s1}{/sythetic\PYGZus{}data.csv}\PYG{l+s+s1}{\PYGZsq{}}
\end{sphinxVerbatim}

\end{sphinxuseclass}\end{sphinxVerbatimInput}

\end{sphinxuseclass}
\begin{sphinxuseclass}{cell}\begin{sphinxVerbatimInput}

\begin{sphinxuseclass}{cell_input}
\begin{sphinxVerbatim}[commandchars=\\\{\}]
\PYG{c+c1}{\PYGZsh{} An attribute is categorical if its domain size is less than this threshold.}
\PYG{c+c1}{\PYGZsh{} Here modify the threshold to adapt to the domain size of \PYGZdq{}education\PYGZdq{} (which is 14 in input dataset).}
\PYG{n}{threshold\PYGZus{}value} \PYG{o}{=} \PYG{l+m+mi}{20} 

\PYG{c+c1}{\PYGZsh{} specify categorical attributes}
\PYG{n}{categorical\PYGZus{}attributes} \PYG{o}{=} \PYG{p}{\PYGZob{}}\PYG{l+s+s1}{\PYGZsq{}}\PYG{l+s+s1}{education}\PYG{l+s+s1}{\PYGZsq{}}\PYG{p}{:} \PYG{k+kc}{True}\PYG{p}{\PYGZcb{}}

\PYG{c+c1}{\PYGZsh{} specify which attributes are candidate keys of input dataset.}
\PYG{n}{candidate\PYGZus{}keys} \PYG{o}{=} \PYG{p}{\PYGZob{}}\PYG{l+s+s1}{\PYGZsq{}}\PYG{l+s+s1}{age}\PYG{l+s+s1}{\PYGZsq{}}\PYG{p}{:} \PYG{k+kc}{False}\PYG{p}{\PYGZcb{}}

\PYG{c+c1}{\PYGZsh{} Number of tuples generated in synthetic dataset.}
\PYG{n}{num\PYGZus{}tuples\PYGZus{}to\PYGZus{}generate} \PYG{o}{=} \PYG{l+m+mi}{32561} \PYG{c+c1}{\PYGZsh{} Here 32561 is the same as input dataset, but it can be set to another number.}
\end{sphinxVerbatim}

\end{sphinxuseclass}\end{sphinxVerbatimInput}

\end{sphinxuseclass}

\section{Step 3 DataDescriber}
\label{\detokenize{src/test/SynthNAV0:id3}}\begin{enumerate}
\sphinxsetlistlabels{\arabic}{enumi}{enumii}{}{.}%
\item {} 
\sphinxAtStartPar
Instantiate a DataDescriber.

\item {} 
\sphinxAtStartPar
Compute the statistics of the dataset.

\item {} 
\sphinxAtStartPar
Save dataset description to a file on local machine.

\end{enumerate}

\begin{sphinxuseclass}{cell}\begin{sphinxVerbatimInput}

\begin{sphinxuseclass}{cell_input}
\begin{sphinxVerbatim}[commandchars=\\\{\}]
\PYG{n}{describer} \PYG{o}{=} \PYG{n}{DataDescriber}\PYG{p}{(}\PYG{n}{category\PYGZus{}threshold}\PYG{o}{=}\PYG{n}{threshold\PYGZus{}value}\PYG{p}{)}
\PYG{n}{describer}\PYG{o}{.}\PYG{n}{describe\PYGZus{}dataset\PYGZus{}in\PYGZus{}independent\PYGZus{}attribute\PYGZus{}mode}\PYG{p}{(}\PYG{n}{dataset\PYGZus{}file}\PYG{o}{=}\PYG{n}{input\PYGZus{}data}\PYG{p}{,}
                                                         \PYG{n}{attribute\PYGZus{}to\PYGZus{}is\PYGZus{}categorical}\PYG{o}{=}\PYG{n}{categorical\PYGZus{}attributes}\PYG{p}{,}
                                                         \PYG{n}{attribute\PYGZus{}to\PYGZus{}is\PYGZus{}candidate\PYGZus{}key}\PYG{o}{=}\PYG{n}{candidate\PYGZus{}keys}\PYG{p}{)}
\PYG{n}{describer}\PYG{o}{.}\PYG{n}{save\PYGZus{}dataset\PYGZus{}description\PYGZus{}to\PYGZus{}file}\PYG{p}{(}\PYG{n}{description\PYGZus{}file}\PYG{p}{)}
\end{sphinxVerbatim}

\end{sphinxuseclass}\end{sphinxVerbatimInput}

\end{sphinxuseclass}

\section{Step 4 generate synthetic dataset}
\label{\detokenize{src/test/SynthNAV0:id4}}\begin{enumerate}
\sphinxsetlistlabels{\arabic}{enumi}{enumii}{}{.}%
\item {} 
\sphinxAtStartPar
Instantiate a DataGenerator.

\item {} 
\sphinxAtStartPar
Generate a synthetic dataset.

\item {} 
\sphinxAtStartPar
Save it to local machine.

\end{enumerate}

\begin{sphinxuseclass}{cell}\begin{sphinxVerbatimInput}

\begin{sphinxuseclass}{cell_input}
\begin{sphinxVerbatim}[commandchars=\\\{\}]
\PYG{n}{generator} \PYG{o}{=} \PYG{n}{DataGenerator}\PYG{p}{(}\PYG{p}{)}
\PYG{n}{generator}\PYG{o}{.}\PYG{n}{generate\PYGZus{}dataset\PYGZus{}in\PYGZus{}independent\PYGZus{}mode}\PYG{p}{(}\PYG{n}{num\PYGZus{}tuples\PYGZus{}to\PYGZus{}generate}\PYG{p}{,} \PYG{n}{description\PYGZus{}file}\PYG{p}{)}
\PYG{n}{generator}\PYG{o}{.}\PYG{n}{save\PYGZus{}synthetic\PYGZus{}data}\PYG{p}{(}\PYG{n}{synthetic\PYGZus{}data}\PYG{p}{)}
\end{sphinxVerbatim}

\end{sphinxuseclass}\end{sphinxVerbatimInput}

\end{sphinxuseclass}

\section{Step 5 compare the statistics of input and sythetic data (optional)}
\label{\detokenize{src/test/SynthNAV0:id5}}
\sphinxAtStartPar
The synthetic data is already saved in a file by step 4. The ModelInspector is for a quick test on the similarity between input and synthetic datasets.


\section{5.1 instantiate a ModelInspector.}
\label{\detokenize{src/test/SynthNAV0:id6}}
\sphinxAtStartPar
It needs input dataset, synthetic dataset, and attribute description.

\begin{sphinxuseclass}{cell}\begin{sphinxVerbatimInput}

\begin{sphinxuseclass}{cell_input}
\begin{sphinxVerbatim}[commandchars=\\\{\}]
\PYG{c+c1}{\PYGZsh{} Read both datasets using Pandas.}
\PYG{n}{input\PYGZus{}df} \PYG{o}{=} \PYG{n}{pd}\PYG{o}{.}\PYG{n}{read\PYGZus{}csv}\PYG{p}{(}\PYG{n}{input\PYGZus{}data}\PYG{p}{,} \PYG{n}{skipinitialspace}\PYG{o}{=}\PYG{k+kc}{True}\PYG{p}{)}
\PYG{n}{synthetic\PYGZus{}df} \PYG{o}{=} \PYG{n}{pd}\PYG{o}{.}\PYG{n}{read\PYGZus{}csv}\PYG{p}{(}\PYG{n}{synthetic\PYGZus{}data}\PYG{p}{)}
\PYG{c+c1}{\PYGZsh{} Read attribute description from the dataset description file.}
\PYG{n}{attribute\PYGZus{}description} \PYG{o}{=} \PYG{n}{read\PYGZus{}json\PYGZus{}file}\PYG{p}{(}\PYG{n}{description\PYGZus{}file}\PYG{p}{)}\PYG{p}{[}\PYG{l+s+s1}{\PYGZsq{}}\PYG{l+s+s1}{attribute\PYGZus{}description}\PYG{l+s+s1}{\PYGZsq{}}\PYG{p}{]}

\PYG{n}{inspector} \PYG{o}{=} \PYG{n}{ModelInspector}\PYG{p}{(}\PYG{n}{input\PYGZus{}df}\PYG{p}{,} \PYG{n}{synthetic\PYGZus{}df}\PYG{p}{,} \PYG{n}{attribute\PYGZus{}description}\PYG{p}{)}
\end{sphinxVerbatim}

\end{sphinxuseclass}\end{sphinxVerbatimInput}

\end{sphinxuseclass}

\section{5.2 compare histograms between input and synthetic datasets.}
\label{\detokenize{src/test/SynthNAV0:id7}}
\begin{sphinxuseclass}{cell}\begin{sphinxVerbatimInput}

\begin{sphinxuseclass}{cell_input}
\begin{sphinxVerbatim}[commandchars=\\\{\}]
\PYG{k}{for} \PYG{n}{attribute} \PYG{o+ow}{in} \PYG{n}{synthetic\PYGZus{}df}\PYG{o}{.}\PYG{n}{columns}\PYG{p}{:}
    \PYG{n}{inspector}\PYG{o}{.}\PYG{n}{compare\PYGZus{}histograms}\PYG{p}{(}\PYG{n}{attribute}\PYG{p}{)}
\end{sphinxVerbatim}

\end{sphinxuseclass}\end{sphinxVerbatimInput}
\begin{sphinxVerbatimOutput}

\begin{sphinxuseclass}{cell_output}
\begin{sphinxVerbatim}[commandchars=\\\{\}]
/Users/m0/miniforge3/envs/mambamojb/lib/python3.9/site\PYGZhy{}packages/DataSynthesizer/ModelInspector.py:82: FutureWarning: iteritems is deprecated and will be removed in a future version. Use .items instead.
  for idx, number in dist\PYGZus{}priv.iteritems():
/Users/m0/miniforge3/envs/mambamojb/lib/python3.9/site\PYGZhy{}packages/DataSynthesizer/ModelInspector.py:85: FutureWarning: iteritems is deprecated and will be removed in a future version. Use .items instead.
  for idx, number in dist\PYGZus{}synt.iteritems():
/Users/m0/miniforge3/envs/mambamojb/lib/python3.9/site\PYGZhy{}packages/DataSynthesizer/ModelInspector.py:82: FutureWarning: iteritems is deprecated and will be removed in a future version. Use .items instead.
  for idx, number in dist\PYGZus{}priv.iteritems():
/Users/m0/miniforge3/envs/mambamojb/lib/python3.9/site\PYGZhy{}packages/DataSynthesizer/ModelInspector.py:85: FutureWarning: iteritems is deprecated and will be removed in a future version. Use .items instead.
  for idx, number in dist\PYGZus{}synt.iteritems():
/Users/m0/miniforge3/envs/mambamojb/lib/python3.9/site\PYGZhy{}packages/DataSynthesizer/ModelInspector.py:82: FutureWarning: iteritems is deprecated and will be removed in a future version. Use .items instead.
  for idx, number in dist\PYGZus{}priv.iteritems():
/Users/m0/miniforge3/envs/mambamojb/lib/python3.9/site\PYGZhy{}packages/DataSynthesizer/ModelInspector.py:85: FutureWarning: iteritems is deprecated and will be removed in a future version. Use .items instead.
  for idx, number in dist\PYGZus{}synt.iteritems():
/Users/m0/miniforge3/envs/mambamojb/lib/python3.9/site\PYGZhy{}packages/DataSynthesizer/ModelInspector.py:82: FutureWarning: iteritems is deprecated and will be removed in a future version. Use .items instead.
  for idx, number in dist\PYGZus{}priv.iteritems():
/Users/m0/miniforge3/envs/mambamojb/lib/python3.9/site\PYGZhy{}packages/DataSynthesizer/ModelInspector.py:85: FutureWarning: iteritems is deprecated and will be removed in a future version. Use .items instead.
  for idx, number in dist\PYGZus{}synt.iteritems():
/Users/m0/miniforge3/envs/mambamojb/lib/python3.9/site\PYGZhy{}packages/DataSynthesizer/ModelInspector.py:82: FutureWarning: iteritems is deprecated and will be removed in a future version. Use .items instead.
  for idx, number in dist\PYGZus{}priv.iteritems():
/Users/m0/miniforge3/envs/mambamojb/lib/python3.9/site\PYGZhy{}packages/DataSynthesizer/ModelInspector.py:85: FutureWarning: iteritems is deprecated and will be removed in a future version. Use .items instead.
  for idx, number in dist\PYGZus{}synt.iteritems():
\end{sphinxVerbatim}

\noindent\sphinxincludegraphics{{SynthNAV0_31_1}.png}

\noindent\sphinxincludegraphics{{SynthNAV0_31_2}.png}

\noindent\sphinxincludegraphics{{SynthNAV0_31_3}.png}

\noindent\sphinxincludegraphics{{SynthNAV0_31_4}.png}

\noindent\sphinxincludegraphics{{SynthNAV0_31_5}.png}

\noindent\sphinxincludegraphics{{SynthNAV0_31_6}.png}

\end{sphinxuseclass}\end{sphinxVerbatimOutput}

\end{sphinxuseclass}

\section{5.3 compare pairwise mutual information}
\label{\detokenize{src/test/SynthNAV0:compare-pairwise-mutual-information}}
\begin{sphinxuseclass}{cell}\begin{sphinxVerbatimInput}

\begin{sphinxuseclass}{cell_input}
\begin{sphinxVerbatim}[commandchars=\\\{\}]
\PYG{n}{inspector}\PYG{o}{.}\PYG{n}{mutual\PYGZus{}information\PYGZus{}heatmap}\PYG{p}{(}\PYG{p}{)}
\end{sphinxVerbatim}

\end{sphinxuseclass}\end{sphinxVerbatimInput}
\begin{sphinxVerbatimOutput}

\begin{sphinxuseclass}{cell_output}
\noindent\sphinxincludegraphics{{SynthNAV0_33_0}.png}

\end{sphinxuseclass}\end{sphinxVerbatimOutput}

\end{sphinxuseclass}

\chapter{DataSynthesizer Usage (correlated attribute mode)}
\label{\detokenize{src/test/SynthNAV0:datasynthesizer-usage-correlated-attribute-mode}}

\section{Step 1 import packages}
\label{\detokenize{src/test/SynthNAV0:id8}}

\section{Step 2 user\sphinxhyphen{}defined parameteres}
\label{\detokenize{src/test/SynthNAV0:id9}}
\begin{sphinxuseclass}{cell}\begin{sphinxVerbatimInput}

\begin{sphinxuseclass}{cell_input}
\begin{sphinxVerbatim}[commandchars=\\\{\}]
\PYG{c+c1}{\PYGZsh{} location of two output files}
\PYG{n}{mode} \PYG{o}{=} \PYG{l+s+s1}{\PYGZsq{}}\PYG{l+s+s1}{correlated\PYGZus{}attribute\PYGZus{}mode}\PYG{l+s+s1}{\PYGZsq{}}
\PYG{n}{description\PYGZus{}file} \PYG{o}{=} \PYG{l+s+sa}{f}\PYG{l+s+s1}{\PYGZsq{}}\PYG{l+s+s1}{/Users/m0/Documents/GitHub/MoHoushmand.github.io/DATA/out/}\PYG{l+s+si}{\PYGZob{}}\PYG{n}{mode}\PYG{l+s+si}{\PYGZcb{}}\PYG{l+s+s1}{/description.json}\PYG{l+s+s1}{\PYGZsq{}}
\PYG{n}{synthetic\PYGZus{}data} \PYG{o}{=} \PYG{l+s+sa}{f}\PYG{l+s+s1}{\PYGZsq{}}\PYG{l+s+s1}{/Users/m0/Documents/GitHub/MoHoushmand.github.io/DATA/out/}\PYG{l+s+si}{\PYGZob{}}\PYG{n}{mode}\PYG{l+s+si}{\PYGZcb{}}\PYG{l+s+s1}{/sythetic\PYGZus{}data.csv}\PYG{l+s+s1}{\PYGZsq{}}
\end{sphinxVerbatim}

\end{sphinxuseclass}\end{sphinxVerbatimInput}

\end{sphinxuseclass}
\begin{sphinxuseclass}{cell}\begin{sphinxVerbatimInput}

\begin{sphinxuseclass}{cell_input}
\begin{sphinxVerbatim}[commandchars=\\\{\}]
\PYG{c+c1}{\PYGZsh{} An attribute is categorical if its domain size is less than this threshold.}
\PYG{c+c1}{\PYGZsh{} Here modify the threshold to adapt to the domain size of \PYGZdq{}education\PYGZdq{} (which is 14 in input dataset).}
\PYG{n}{threshold\PYGZus{}value} \PYG{o}{=} \PYG{l+m+mi}{20}

\PYG{c+c1}{\PYGZsh{} specify categorical attributes}
\PYG{n}{categorical\PYGZus{}attributes} \PYG{o}{=} \PYG{p}{\PYGZob{}}\PYG{l+s+s1}{\PYGZsq{}}\PYG{l+s+s1}{education}\PYG{l+s+s1}{\PYGZsq{}}\PYG{p}{:} \PYG{k+kc}{True}\PYG{p}{\PYGZcb{}}

\PYG{c+c1}{\PYGZsh{} specify which attributes are candidate keys of input dataset.}
\PYG{n}{candidate\PYGZus{}keys} \PYG{o}{=} \PYG{p}{\PYGZob{}}\PYG{l+s+s1}{\PYGZsq{}}\PYG{l+s+s1}{ssn}\PYG{l+s+s1}{\PYGZsq{}}\PYG{p}{:} \PYG{k+kc}{True}\PYG{p}{\PYGZcb{}}

\PYG{c+c1}{\PYGZsh{} A parameter in Differential Privacy. It roughly means that removing a row in the input dataset will not }
\PYG{c+c1}{\PYGZsh{} change the probability of getting the same output more than a multiplicative difference of exp(epsilon).}
\PYG{c+c1}{\PYGZsh{} Increase epsilon value to reduce the injected noises. Set epsilon=0 to turn off differential privacy.}
\PYG{n}{epsilon} \PYG{o}{=} \PYG{l+m+mi}{1}

\PYG{c+c1}{\PYGZsh{} The maximum number of parents in Bayesian network, i.e., the maximum number of incoming edges.}
\PYG{n}{degree\PYGZus{}of\PYGZus{}bayesian\PYGZus{}network} \PYG{o}{=} \PYG{l+m+mi}{2}

\PYG{c+c1}{\PYGZsh{} Number of tuples generated in synthetic dataset.}
\PYG{n}{num\PYGZus{}tuples\PYGZus{}to\PYGZus{}generate} \PYG{o}{=} \PYG{l+m+mi}{1000} \PYG{c+c1}{\PYGZsh{} Here 32561 is the same as input dataset, but it can be set to another number.}
\end{sphinxVerbatim}

\end{sphinxuseclass}\end{sphinxVerbatimInput}

\end{sphinxuseclass}

\section{Step 3 DataDescriber}
\label{\detokenize{src/test/SynthNAV0:id10}}
\sphinxAtStartPar
Instantiate a DataDescriber.
Compute the statistics of the dataset.
Save dataset description to a file on local machine.

\begin{sphinxuseclass}{cell}\begin{sphinxVerbatimInput}

\begin{sphinxuseclass}{cell_input}
\begin{sphinxVerbatim}[commandchars=\\\{\}]
\PYG{n}{describer} \PYG{o}{=} \PYG{n}{DataDescriber}\PYG{p}{(}\PYG{n}{category\PYGZus{}threshold}\PYG{o}{=}\PYG{n}{threshold\PYGZus{}value}\PYG{p}{)}
\PYG{n}{describer}\PYG{o}{.}\PYG{n}{describe\PYGZus{}dataset\PYGZus{}in\PYGZus{}correlated\PYGZus{}attribute\PYGZus{}mode}\PYG{p}{(}\PYG{n}{dataset\PYGZus{}file}\PYG{o}{=}\PYG{n}{input\PYGZus{}data}\PYG{p}{,} 
                                                        \PYG{n}{epsilon}\PYG{o}{=}\PYG{n}{epsilon}\PYG{p}{,} 
                                                        \PYG{n}{k}\PYG{o}{=}\PYG{n}{degree\PYGZus{}of\PYGZus{}bayesian\PYGZus{}network}\PYG{p}{,}
                                                        \PYG{n}{attribute\PYGZus{}to\PYGZus{}is\PYGZus{}categorical}\PYG{o}{=}\PYG{n}{categorical\PYGZus{}attributes}\PYG{p}{,}
                                                        \PYG{n}{attribute\PYGZus{}to\PYGZus{}is\PYGZus{}candidate\PYGZus{}key}\PYG{o}{=}\PYG{n}{candidate\PYGZus{}keys}\PYG{p}{)}
\PYG{n}{describer}\PYG{o}{.}\PYG{n}{save\PYGZus{}dataset\PYGZus{}description\PYGZus{}to\PYGZus{}file}\PYG{p}{(}\PYG{n}{description\PYGZus{}file}\PYG{p}{)}
\end{sphinxVerbatim}

\end{sphinxuseclass}\end{sphinxVerbatimInput}
\begin{sphinxVerbatimOutput}

\begin{sphinxuseclass}{cell_output}
\begin{sphinxVerbatim}[commandchars=\\\{\}]
================ Constructing Bayesian Network (BN) ================
Adding ROOT relationship
Adding attribute marital\PYGZhy{}status
Adding attribute age
Adding attribute sex
Adding attribute education
Adding attribute income
========================== BN constructed ==========================
\end{sphinxVerbatim}

\begin{sphinxVerbatim}[commandchars=\\\{\}]
/Users/m0/miniforge3/envs/mambamojb/lib/python3.9/site\PYGZhy{}packages/DataSynthesizer/lib/PrivBayes.py:275: FutureWarning: In a future version of pandas, a length 1 tuple will be returned when iterating over a groupby with a grouper equal to a list of length 1. Don\PYGZsq{}t supply a list with a single grouper to avoid this warning.
  for parents\PYGZus{}instance, stats\PYGZus{}sub in stats.groupby(parents):
\end{sphinxVerbatim}

\end{sphinxuseclass}\end{sphinxVerbatimOutput}

\end{sphinxuseclass}

\section{Step 4 generate synthetic dataset}
\label{\detokenize{src/test/SynthNAV0:id11}}\begin{enumerate}
\sphinxsetlistlabels{\arabic}{enumi}{enumii}{}{.}%
\item {} 
\sphinxAtStartPar
Instantiate a DataGenerator.

\item {} 
\sphinxAtStartPar
Generate a synthetic dataset.

\item {} 
\sphinxAtStartPar
Save it to local machine.

\end{enumerate}

\begin{sphinxuseclass}{cell}\begin{sphinxVerbatimInput}

\begin{sphinxuseclass}{cell_input}
\begin{sphinxVerbatim}[commandchars=\\\{\}]
\PYG{n}{generator} \PYG{o}{=} \PYG{n}{DataGenerator}\PYG{p}{(}\PYG{p}{)}
\PYG{n}{generator}\PYG{o}{.}\PYG{n}{generate\PYGZus{}dataset\PYGZus{}in\PYGZus{}correlated\PYGZus{}attribute\PYGZus{}mode}\PYG{p}{(}\PYG{n}{num\PYGZus{}tuples\PYGZus{}to\PYGZus{}generate}\PYG{p}{,} \PYG{n}{description\PYGZus{}file}\PYG{p}{)}
\PYG{n}{generator}\PYG{o}{.}\PYG{n}{save\PYGZus{}synthetic\PYGZus{}data}\PYG{p}{(}\PYG{n}{synthetic\PYGZus{}data}\PYG{p}{)}
\end{sphinxVerbatim}

\end{sphinxuseclass}\end{sphinxVerbatimInput}

\end{sphinxuseclass}

\section{Step 5 compare the statistics of input and sythetic data (optional)}
\label{\detokenize{src/test/SynthNAV0:id12}}
\sphinxAtStartPar
The synthetic data is already saved in a file by step 4. The ModelInspector is for a quick test on the similarity between input and synthetic datasets.


\section{5.1 instantiate a ModelInspector.}
\label{\detokenize{src/test/SynthNAV0:id13}}
\sphinxAtStartPar
It needs input dataset, synthetic dataset, and attribute description

\begin{sphinxuseclass}{cell}\begin{sphinxVerbatimInput}

\begin{sphinxuseclass}{cell_input}
\begin{sphinxVerbatim}[commandchars=\\\{\}]
\PYG{c+c1}{\PYGZsh{} Read both datasets using Pandas.}
\PYG{n}{input\PYGZus{}df} \PYG{o}{=} \PYG{n}{pd}\PYG{o}{.}\PYG{n}{read\PYGZus{}csv}\PYG{p}{(}\PYG{n}{input\PYGZus{}data}\PYG{p}{,} \PYG{n}{skipinitialspace}\PYG{o}{=}\PYG{k+kc}{True}\PYG{p}{)}
\PYG{n}{synthetic\PYGZus{}df} \PYG{o}{=} \PYG{n}{pd}\PYG{o}{.}\PYG{n}{read\PYGZus{}csv}\PYG{p}{(}\PYG{n}{synthetic\PYGZus{}data}\PYG{p}{)}
\PYG{c+c1}{\PYGZsh{} Read attribute description from the dataset description file.}
\PYG{n}{attribute\PYGZus{}description} \PYG{o}{=} \PYG{n}{read\PYGZus{}json\PYGZus{}file}\PYG{p}{(}\PYG{n}{description\PYGZus{}file}\PYG{p}{)}\PYG{p}{[}\PYG{l+s+s1}{\PYGZsq{}}\PYG{l+s+s1}{attribute\PYGZus{}description}\PYG{l+s+s1}{\PYGZsq{}}\PYG{p}{]}

\PYG{n}{inspector} \PYG{o}{=} \PYG{n}{ModelInspector}\PYG{p}{(}\PYG{n}{input\PYGZus{}df}\PYG{p}{,} \PYG{n}{synthetic\PYGZus{}df}\PYG{p}{,} \PYG{n}{attribute\PYGZus{}description}\PYG{p}{)}
\end{sphinxVerbatim}

\end{sphinxuseclass}\end{sphinxVerbatimInput}

\end{sphinxuseclass}

\section{5.2 compare histograms between input and synthetic datasets.}
\label{\detokenize{src/test/SynthNAV0:id14}}
\begin{sphinxuseclass}{cell}\begin{sphinxVerbatimInput}

\begin{sphinxuseclass}{cell_input}
\begin{sphinxVerbatim}[commandchars=\\\{\}]
\PYG{k}{for} \PYG{n}{attribute} \PYG{o+ow}{in} \PYG{n}{synthetic\PYGZus{}df}\PYG{o}{.}\PYG{n}{columns}\PYG{p}{:}
    \PYG{n}{inspector}\PYG{o}{.}\PYG{n}{compare\PYGZus{}histograms}\PYG{p}{(}\PYG{n}{attribute}\PYG{p}{)}
\end{sphinxVerbatim}

\end{sphinxuseclass}\end{sphinxVerbatimInput}
\begin{sphinxVerbatimOutput}

\begin{sphinxuseclass}{cell_output}
\begin{sphinxVerbatim}[commandchars=\\\{\}]
/Users/m0/miniforge3/envs/mambamojb/lib/python3.9/site\PYGZhy{}packages/DataSynthesizer/ModelInspector.py:82: FutureWarning: iteritems is deprecated and will be removed in a future version. Use .items instead.
  for idx, number in dist\PYGZus{}priv.iteritems():
/Users/m0/miniforge3/envs/mambamojb/lib/python3.9/site\PYGZhy{}packages/DataSynthesizer/ModelInspector.py:85: FutureWarning: iteritems is deprecated and will be removed in a future version. Use .items instead.
  for idx, number in dist\PYGZus{}synt.iteritems():
/Users/m0/miniforge3/envs/mambamojb/lib/python3.9/site\PYGZhy{}packages/DataSynthesizer/ModelInspector.py:82: FutureWarning: iteritems is deprecated and will be removed in a future version. Use .items instead.
  for idx, number in dist\PYGZus{}priv.iteritems():
/Users/m0/miniforge3/envs/mambamojb/lib/python3.9/site\PYGZhy{}packages/DataSynthesizer/ModelInspector.py:85: FutureWarning: iteritems is deprecated and will be removed in a future version. Use .items instead.
  for idx, number in dist\PYGZus{}synt.iteritems():
/Users/m0/miniforge3/envs/mambamojb/lib/python3.9/site\PYGZhy{}packages/DataSynthesizer/ModelInspector.py:82: FutureWarning: iteritems is deprecated and will be removed in a future version. Use .items instead.
  for idx, number in dist\PYGZus{}priv.iteritems():
/Users/m0/miniforge3/envs/mambamojb/lib/python3.9/site\PYGZhy{}packages/DataSynthesizer/ModelInspector.py:85: FutureWarning: iteritems is deprecated and will be removed in a future version. Use .items instead.
  for idx, number in dist\PYGZus{}synt.iteritems():
/Users/m0/miniforge3/envs/mambamojb/lib/python3.9/site\PYGZhy{}packages/DataSynthesizer/ModelInspector.py:82: FutureWarning: iteritems is deprecated and will be removed in a future version. Use .items instead.
  for idx, number in dist\PYGZus{}priv.iteritems():
/Users/m0/miniforge3/envs/mambamojb/lib/python3.9/site\PYGZhy{}packages/DataSynthesizer/ModelInspector.py:85: FutureWarning: iteritems is deprecated and will be removed in a future version. Use .items instead.
  for idx, number in dist\PYGZus{}synt.iteritems():
/Users/m0/miniforge3/envs/mambamojb/lib/python3.9/site\PYGZhy{}packages/DataSynthesizer/ModelInspector.py:82: FutureWarning: iteritems is deprecated and will be removed in a future version. Use .items instead.
  for idx, number in dist\PYGZus{}priv.iteritems():
/Users/m0/miniforge3/envs/mambamojb/lib/python3.9/site\PYGZhy{}packages/DataSynthesizer/ModelInspector.py:85: FutureWarning: iteritems is deprecated and will be removed in a future version. Use .items instead.
  for idx, number in dist\PYGZus{}synt.iteritems():
\end{sphinxVerbatim}

\noindent\sphinxincludegraphics{{SynthNAV0_46_1}.png}

\noindent\sphinxincludegraphics{{SynthNAV0_46_2}.png}

\noindent\sphinxincludegraphics{{SynthNAV0_46_3}.png}

\noindent\sphinxincludegraphics{{SynthNAV0_46_4}.png}

\noindent\sphinxincludegraphics{{SynthNAV0_46_5}.png}

\noindent\sphinxincludegraphics{{SynthNAV0_46_6}.png}

\end{sphinxuseclass}\end{sphinxVerbatimOutput}

\end{sphinxuseclass}

\section{5.3 compare pairwise mutual information}
\label{\detokenize{src/test/SynthNAV0:id15}}
\begin{sphinxuseclass}{cell}\begin{sphinxVerbatimInput}

\begin{sphinxuseclass}{cell_input}
\begin{sphinxVerbatim}[commandchars=\\\{\}]
\PYG{n}{inspector}\PYG{o}{.}\PYG{n}{mutual\PYGZus{}information\PYGZus{}heatmap}\PYG{p}{(}\PYG{p}{)}
\end{sphinxVerbatim}

\end{sphinxuseclass}\end{sphinxVerbatimInput}
\begin{sphinxVerbatimOutput}

\begin{sphinxuseclass}{cell_output}
\noindent\sphinxincludegraphics{{SynthNAV0_48_0}.png}

\end{sphinxuseclass}\end{sphinxVerbatimOutput}

\end{sphinxuseclass}
\begin{sphinxuseclass}{cell}\begin{sphinxVerbatimInput}

\begin{sphinxuseclass}{cell_input}
\begin{sphinxVerbatim}[commandchars=\\\{\}]
\PYG{n}{i} \PYG{o}{=} \PYG{n}{rand}\PYG{p}{(}\PYG{l+m+mf}{1.}\PYG{l+m+mf}{.10}\PYG{p}{)}
\PYG{n}{unless} \PYG{n}{i} \PYG{o}{\PYGZpc{}} \PYG{l+m+mi}{2} \PYG{o}{==} \PYG{l+m+mi}{0}
    \PYG{n}{puts} \PYG{l+s+s2}{\PYGZdq{}}\PYG{l+s+s2}{\PYGZsh{}}\PYG{l+s+si}{\PYGZob{}i\PYGZcb{}}\PYG{l+s+s2}{ is odd}\PYG{l+s+s2}{\PYGZdq{}}
\PYG{k}{else}
    \PYG{n}{puts} \PYG{l+s+s2}{\PYGZdq{}}\PYG{l+s+s2}{\PYGZsh{}}\PYG{l+s+si}{\PYGZob{}i\PYGZcb{}}\PYG{l+s+s2}{ is even}\PYG{l+s+s2}{\PYGZdq{}}
\PYG{n}{end}
\end{sphinxVerbatim}

\end{sphinxuseclass}\end{sphinxVerbatimInput}
\begin{sphinxVerbatimOutput}

\begin{sphinxuseclass}{cell_output}
\begin{sphinxVerbatim}[commandchars=\\\{\}]
\PYG{g+gt}{  File}\PYG{n+nn}{ \PYGZdq{}/tmp/ipykernel\PYGZus{}18899/2226457451.py\PYGZdq{}}\PYG{g+gt}{, line }\PYG{l+m+mi}{1}
    \PYG{n}{i} \PYG{o}{=} \PYG{n}{rand}\PYG{p}{(}\PYG{l+m+mf}{1.}\PYG{l+m+mf}{.10}\PYG{p}{)}
                 \PYG{o}{\PYGZca{}}
\PYG{n+ne}{SyntaxError}: invalid syntax
\end{sphinxVerbatim}

\end{sphinxuseclass}\end{sphinxVerbatimOutput}

\end{sphinxuseclass}
\sphinxstepscope


\chapter{SyntheticNAV}
\label{\detokenize{src/test/SyntheticNAV:syntheticnav}}\label{\detokenize{src/test/SyntheticNAV::doc}}

\section{gebe}
\label{\detokenize{src/test/SyntheticNAV:gebe}}

\chapter{Introduction}
\label{\detokenize{src/test/SyntheticNAV:introduction}}

\section{What is SyntheticNAV?}
\label{\detokenize{src/test/SyntheticNAV:what-is-syntheticnav}}

\subsection{ffs}
\label{\detokenize{src/test/SyntheticNAV:ffs}}

\section{How to use SyntheticNAV?}
\label{\detokenize{src/test/SyntheticNAV:how-to-use-syntheticnav}}

\subsection{ffs}
\label{\detokenize{src/test/SyntheticNAV:id1}}
\begin{sphinxthebibliography}{HdHPK14}
\bibitem[HdHPK14]{index:id3}
\sphinxAtStartPar
Christopher Ramsay Holdgraf, Wendy de Heer, Brian N. Pasley, and Robert T. Knight. Evidence for Predictive Coding in Human Auditory Cortex. In \sphinxstyleemphasis{International Conference on Cognitive Neuroscience}. Brisbane, Australia, Australia, 2014. Frontiers in Neuroscience.
\bibitem[HdHPK14]{src/test/markdown:id3}
\sphinxAtStartPar
Christopher Ramsay Holdgraf, Wendy de Heer, Brian N. Pasley, and Robert T. Knight. Evidence for Predictive Coding in Human Auditory Cortex. In \sphinxstyleemphasis{International Conference on Cognitive Neuroscience}. Brisbane, Australia, Australia, 2014. Frontiers in Neuroscience.
\end{sphinxthebibliography}







\renewcommand{\indexname}{Index}
\printindex
\end{document}